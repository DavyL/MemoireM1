\documentclass{report}
\usepackage[utf8]{inputenc}
\usepackage{amsmath, amssymb, amsthm}
\usepackage{geometry}
\usepackage{mathabx}
\usepackage{tikz}
\usepackage{pgfplots}
\usepackage{standalone}
\usepackage{graphicx}

\usepackage[backend = bibtex]{biblatex}
\addbibresource{bibliography.bib}

\newgeometry{vmargin={30mm}, hmargin={20mm,20mm}}


\usetikzlibrary{external}
\tikzexternalize[shell escape=-enable-write18]

\pgfplotsset{compat=1.16}
\usepgfplotslibrary{external}
\theoremstyle{plain}
\newtheorem{theoreme}{Theoreme}[section]
\newtheorem{proposition}[theoreme]{Proposition}
\newtheorem{lemme}[theoreme]{Lemme}

\theoremstyle{remark}
\newtheorem{remarque}[theoreme]{Remarque}
\newtheorem{exemple}[theoreme]{Exemple}
\newtheorem{preuve}[theoreme]{Preuve}

\theoremstyle{definition}
\newtheorem{definition}[theoreme]{Definition}



\newcommand{\notimplies}{
	\mathrel{{\ooalign{\hidewidth$\not\phantom{=}$\hidewidth\cr$\implies$}}}}

%\title{Different approaches to random graphs: Erd\H{o}s-R\'enyi and Branching Processes}

%\author{Leo \textsc{Davy} }
%\date{2019\\ March-May}
\usepackage{hyperref}
\begin{document}
\renewcommand{\proofname}{Preuve}
\begin{titlepage}
	\begin{center}
		\vspace*{1cm}
		\huge
		\textbf{Reconstruction parcimonieuse de signal\\}
		\vspace*{0.5cm}
	        \LARGE
		Introduction à la reconstruction de signal et au compressed sensing\\ 
		\vspace{1.5cm}
		\textbf{Leo Davy}
		\vfill				      
	        supervisé par\par
		\textbf{Jean-François \textsc{Crouzet}}
							    
		\vspace{0.8cm}
		%\begin{wrapfigure}{r}{0.1\textwidth}
		%	\centering
		%	\includegraphics[width=0.4\textwidth]{logo_uob}
		%\end{wrapfigure}
		
		\Large					  
		Institut de Mathématiques Alexander Grothendieck\\
		Université de Montpellier\\
	        Faculté des sciences\\
	        Février-Mai
				      
	\end{center}
\end{titlepage}

%\maketitle

\tableofcontents{}

\chapter{Cadre du problème}
\section{Problématique de la reconstruction de signal}
La reconstruction du signal est un problème que l'on considère dans le cadre du traitement du signal, c'est à dire que l'on considère qu'à un signal, on peut appliquer une transformation, et de cette transformation on obtient un nouveau signal qui aura certaines caractéristiques permettant de mieux comprendre ce signal.
D'un point de vue plus formel, on considère une famille de signaux $\mathcal{F}$, chaque élément de cette famille étant une application $f : X \longrightarrow Y$, et on considère un opérateur $A : \mathcal{F} \longrightarrow \mathcal{G}$, où $\mathcal{G}$ est une autre famille de signaux.
Donnons ici quelques exemples de familles de fonctions que l'on rencontrera dans ce mémoire. Commencons avec les fonctions à temps continu (c'est à dire avec $X = \mathbb{R}^d$ pour un certain $d>0$).
%%TODO : remplacer d par N pour être cohérent avec la suite
\begin{exemple}
	Signaux à énergie finie :
	\begin{enumerate}
		\item $\mathcal{F} = L^2(\mathbb{R}^d) := \{f : \mathbb{R}^d \longrightarrow \mathbb{R} | \int_\mathbb{R}^d |f(t)|^2dt < \infty\}$.
		\item $\mathcal{F} = L^p(\mathbb{R}^d) := \{f : \mathbb{R}^d \longrightarrow \mathbb{R} | \int_\mathbb{R}^d |f(t)|^pdt < \infty\}$, pour $0 < p \leq 1$.
		\item $\mathcal{F} = L^\infty(\mathbb{R}^d) := \{f : \mathbb{R}^d \longrightarrow \mathbb{R} | \sup_{t\in \mathbb{R}^d} |f(t)| < \infty\}$.
	\end{enumerate}
	Signaux avec une régularité :
	\begin{enumerate}
		\item $\mathcal{F} = \mathcal{C}^0(\mathbb{R}^d) =\{f : \mathbb{R}^d \longrightarrow \mathbb{R} \text{avec } f \text{ continue}\}$.
		\item $\mathcal{F} = \mathcal{C}^r(\mathbb{R}^d) =\{f : \mathbb{R}^d \longrightarrow \mathbb{R} \text{avec } \forall t \in \mathbb{R}^d \sup_{0\leq k \leq r} | f^{(k)}(t)| < \infty$.
	\end{enumerate}
\end{exemple}
on pourra aussi considérer pour chacun des espaces ci-dessus, leur version à support compact, notée avec l'indice $_\text{loc}$ (par exemple $L^p_\text{loc}$ ou $\mathcal{C}^p_\text{loc}$ ) pour indiquer que pour tout $f\in \mathcal{F}_\text{loc}$, il existe un $C \geq 0$ tel que $\lvert t \rvert \geq C \implies f(t) = 0$. On considérera également des càs où $X$ est un ouvert ou un fermé de $\mathbb{R}^d$
Un autre espace qui sera peut être utilisé est l'espace de Sobolev $W^{r, p}$ qui représente des signaux à énergie finie pour $||\cdot|| _p$, mais dont chaque dérivéé (définie faiblement) d'ordre inférieur ou égal à $r$ est elle aussi à énergie finie.
\newline
Une autre classe d'intérêt de signaux majeur est celle des signaux à temps discret (c'est à dire avec $X = \mathbb{N}^d$).
\begin{exemple}
	Signaux à énergie finie :
	\begin{enumerate}
		\item $\mathcal{F} = l^2(\mathbb{N}^d) := \{f : \mathbb{N}^d \longrightarrow \mathbb{R} | \sum_\mathbb{N}^d |f(t)|^2 < \infty\}$.
		\item $\mathcal{F} = l^p(\mathbb{N}^d) := \{f : \mathbb{N}^d \longrightarrow \mathbb{R} | \sum_\mathbb{N}^d |f(t)|^p < \infty\}$, pour $0 < p \leq 1$.
		\item $\mathcal{F} = l^\infty(\mathbb{N}^d) := \{f : \mathbb{N}^d \longrightarrow \mathbb{R} | \sup_{t\in \mathbb{N}^d} |f(t)| < \infty\}$.
	\end{enumerate}
\end{exemple}
On verra plus loin que l'on peut également définir une notion de régularité intéressante pour les signaux à temps discret.
\begin{remarque}
	Dans les exemples ci-dessus les signaux sont à valeur dans $Y = \mathbb{R}$, cependant toutes ces exemples peuvent être considérées avec $\mathbb{C}$ comme espace d'arrivée. 
\end{remarque}

\subsection{Exemples de signaux étudiés (image, sons, tomographie, ...) et formalisation mathématique de leur description}
Considérons maintenant de façon plus concrète des exemples de signaux étudiés afin d'introduire l'opérateur $A$.
\newline
TODO : Pour chacun des exemples ci-dessous, formaliser le problème et poser sa solution comme un problème de minimisation "$argmin$"

\begin{exemple} L'exemple le plus simple pour introduire le sujet est celui d'un signal à une dimension, on pourra par exemple penser à un signal décrivant un son ou bien un signal electrique, d'une durée finie, et à chaque instant on peut associer une amplitude. D'un point de vue formel on pourra ainsi considérer que ce signal est $f:[0, 1] \longrightarrow \mathbb{R}^+$. On pourra cherche à faire différentes opérations sur ce signal :
	\begin{itemize}
		\item \it{Echantillonage}
		\item \it{Seuil}
		\item \it{Décomposition harmonique (Fourier)}
		\item \it{Filtrage}
	\end{itemize}
\end{exemple}
\begin{exemple}
	Après avoir considéré le signal à une dimension, un autre type de signal est celui des signaux en deux ou trois dimensions  dimensions dont l'exemple type est celui des vidéos ou des images :
	\begin{itemize}
		\item \it{Débruitage}
		\item \it{Super-résolution}
		\item \it{Compression}
		\item \it{Détection / Reconnaissance}
	\end{itemize}
\end{exemple}
\begin{exemple}
	Une autre problématique essentielle que l'on considérera en profondeur dans ce mémoire est celui des problèmes inverses dans lesquels à partir d'un signal mesuré, on cherchera à reconstruire ce qui a émis ce signal.
	\begin{itemize}
		\item \it{Tomographie (transformée de Radon)} 
		\item \it{Géologie/IRM}
	\end{itemize}  
\end{exemple}
\begin{exemple}
	Récemment, des problèmes avec des signaux en grande dimension sont aussi apparus, notamment dans des problématiques de type big-data.
\end{exemple}

Cependant sur ces problèmes il y a une ambiguité sur la définition de la dimension qui est considérée comme la dimension de l'espace de départ du signal considéré, mais cependant, chaque signal appartient à un espace qui n'a à priori aucune raison d'être fini. 
De plus, chacun de ces problèmes commence par une mesure qui est toujours un processus discret et le reste du traitement est réalisé sur un ordinateur qui est lui aussi un processus discret.
Ainsi chacun de ces problèmes est discretisé et alors la dimension\footnote{On considère ici la "dimension" comme étant le nombre de degré de libertés du signal étudié.} du signal augmente de façon considérable, ainsi, une image photographie a typiquement une dimension $d >> 10^6$.

\section{Exactitude, échantillonage et bruit}
\subsection{Lien entre l'exactitude et l'échantillonage}
Ainsi il est nécessaire d'adapter la stratégie d'échantillonage, un échantillonage insuffisant ou inaproprié ne permettra pas avec certitude de pouvoir récupérer l'information sous-jacente au signal.
Un échantillonage qui prendrait trop de mesures pose aussi des problèmes, premièrement car si la taille de ces mesures est trop importante, il sera difficile de faire des opérations dessus et cela compliquera la résolution du problème. 
Mais aussi, car augmenter le nombre de mesures risque de ne pas apporter davantage d'informations pour la résolution du problème \footnote{On peut ici penser aux problématiques d'\it{overfitting} du Machine Learning dans lesquels un système qui est trop entrainé sur un ensemble de données devient inefficace dès qu'il est testé sur des données sur lesquelles il n'a pas été entrainé}
, et dans certains cas, ces mesures superflues risquent seulement de mesurer du bruit et donc de diminuer l'efficacité de la résolution.
Il est donc nécessaire pour une famille de signaux donnée d'avoir des conditions nécessaires sur l'échantillonage, pour veiller à être certain d'étudier au moins le signal, mais aussi des conditions suffisantes pour ne pas étudier trop au delà du signal.
\subsection{L'importance du bruit dans les problèmes}
Ainsi il est nécessaire de prendre en compte le fait qu'il y ait des sources de bruit dans les problèmes considérés et on cherchera donc à vérifier que les constructions qui viendront seront stables face au bruit.
Une remarque importante à faire est que le bruit est généralement constitué de modifications très locales (que l'on considérera ainsi comme "hautes-fréquences").

\section{Bases orthonormales et frames}
\subsection{Intérêt des bases orthonormales et description des outils mathématiques disponibles}
Une approche classique et pratique pour l'analyse de signaux est l'utilisation d'une base orthonormale pour représenter un signal. 
En effet l'intérêt est multiple, si l'on connait une base orthonormale de décomposition d'un signal, alors il y aura une unique façon d'écrire ce signal dans cette base, mais surtout, l'espace est alors naturellement muni d'un produit scalaire qui permettra d'utiliser tout l'outillage des espaces de Hilbert pour résoudre le problème.
\newline
On verra ainsi dans cette section tout d'abord des définitions et propriétés classiques des espaces de Hilbert. 
Ensuite on verra progressivement comment relacher certaines des définitions initiales afin de pouvoir conserver une formule de reconstruction.
Afin d'expliciter l'intérêt de ces définitions on verra deux exemples de frames.
Tout d'abord le frame de Fourier, dont la compréhension sera utile pour le troisième chapitre.
Finalement, nous introduirons les ondelettes par l'analyse multi-résolution, les formules que nous obtiendrons seront utilisées dans le chapitre suivant.
\subsection{Lien entre frames et base orthonormale}
Rappelons tout d'abord les définitions et propriétés d'une base orthonormale. On considère ici un espace $H$ muni d'un produit scalaire et une famille $\{e_i\}_I$, avec $I$ un ensemble.
\begin{definition}
	On dira qu'une famille $\{e_i\}_I$ est :
	\begin{itemize}
		\item \it{libre} si pour n'importe quelle suite finie de coefficients $(\lambda_i)_I$ telle que $\sum_I \lambda_i e_i = 0$, on a $\lambda_i = 0$ pour n'importe quel $i \in I$.
		\item \it{orthogonale}\footnote{Dans un espace vectoriel muni d'un produit scalaire et d'une base, une famille est libre si et seulement si elle est orthogonale (cela découle des propriétés du produit scalaire, notamment qu'il est défini).} si pour n'importe quels $i$ et $j$ différents on a $\langle e_i, e_j \rangle = 0$
		\item \it{génératrice} si quel que soit $f \in H$ tel que pour tout $i\in I$ on a $\langle f, e_i \rangle = 0$, alors $f =0$.
		\item une \it{base} si la famille est libre et génératrice.
	\end{itemize}
\end{definition}
Donc, pour tout $h \in H$, si la famille $\{e_i\}_I$ est libre et génératrice, il existe une unique suite $(\lambda_i)_{i \in I}$ de scalaires, telle que $ h = \sum_{i \in I} \lambda_i e_i$.
On peut alors définir un nouveau produit scalaire sur $H$: 
\begin{align}
	\langle \cdot, \cdot \rangle :  H \times H &\longrightarrow \mathbb{R} \\
		(h_1= (\lambda_i)_I, h_2 = (\mu_i)_I) &\longmapsto \langle h_1, h_2 \rangle = \sum_I \lambda_i \mu_i^*.
\end{align}
On peut remarquer que ce produit scalaire est défini de façon unique par rapport à la base $\{e_i\}_I$, cependant, si la base est normalisée ($\langle e_i, e_i \rangle =1$), alors ce produit scalaire devient identique au produit scalaire initial de $H$. 
Ainsi, la valuation du produit scalaire est indépendante du choix de la base (tant qu'elle est normalisée)\footnote{Dans $L^2(\mathbb{R})$, on peut ainsi retrouver les égalités de Parseval ou de Plancherel en exprimant une fonction $f\in L^2(\mathbb{R})$ soit dans la base canonique de $L^2(\mathbb{R})$ par rapport à $f$ donnée par le vecteur $\frac{f}{||f||}$et une base de son orthogonal, soit en exprimant la fonction dans la base de Fourier.}.
On a alors le théorème suivant qui nous donne une condition nécessaire et suffisante pour que l'espace engendré par une famille $\{f_i\}$ soit dense dans H:
\begin{theoreme}
	Soit $\{f_i\}_I$ une suite d'éléments orthonormaux dans $H$ muni d'un produit scalaire.
	Alors $\overline{\text{Vect}(\{f_i\}_I)} = H$ si et seulement si 
	\begin{equation*}
		\sum_I |\langle f, f_i\rangle|^2 = ||f||_2 ^2 \quad, \forall f \in H.
	\end{equation*}
\end{theoreme}
Cependant, comme on le verra dans la suite, il y a des situations dans lesquelles chercher à avoir une base orthonormale est trop restrictif, on cherchera donc à relacher les conditions sur la définition d'une base.
\newline
Tout d'abord, si la famille est orthogonale, mais elle n'est pas génératrice on a 
\begin{theoreme}
	Soit $\{f_i\}_I$ une famille orthonormale de $H$.
	Alors,
	\begin{equation*}
		\sum_I |\langle f, f_i \rangle|^2 \leq ||f||_2 ^2, \forall f \in H
	\end{equation*}.
\end{theoreme}
On peut exprimer ce théorème en disant que l'analyse par une famille orthogonale n'ajoute pas d'énergie au vecteur analysé.
et si la famille est génératrice,
\begin{theoreme}
	Soit $\{f_i\}_I$ une famille génératrice normalisée.
	Alors,
	\begin{equation*}
		||f||^2 \leq \sum_I |\langle f, f_i \rangle|^2, \forall f \in H
	\end{equation*}.
\end{theoreme}
On peut exprimer ce théorème en disant que l'analyse par une famille génératrice capture au moins l'énergie du vecteur analysé.
Au vu de ces résultats, on est amenés à considérer les définitions suivantes qui correspondent à une relaxation de la condition de normalisation ou de la condition d'orthogonalité.
\begin{definition}
	Pour une famille d'éléments $\{f_i\}_I$ de $H$, alors on dit que c'est 
	\begin{enumerate}
		\item Une suite de \it{Bessel} si il existe une constante $M>0$ telle que
			\begin{equation*}
				\sum_I |\langle f, f_i \rangle|^2 \leq M||f||^2, \forall f \in H.
			\end{equation*}
		\item Un \it{frame} si il existe des constantes $M, m>0$ telles que
			\begin{equation}\label{eq:defFrame}
				m||f||^2 \leq \sum_I |\langle f, f_i\rangle|^2 \leq M||f||^2, \forall f \in H.
			\end{equation}
		\item Une \it{base de Riesz} (ou base \it{inconditionnelle}) si il existe des constantes $M, m>0$ telles que
			\begin{equation*}
				m\sum |c_k|^2 \leq ||\sum c_k f_k||^2 \leq M\sum |c_k|^2
			\end{equation*}
		pour n'importe quelle suite finie $\{c_k\}$.
	\end{enumerate}
\end{definition}
\begin{remarque}
	\begin{itemize}
		\item Une base orthonormale est une base de Riesz avec $m = M = 1$.
		\item Une base de Riesz est un frame dont les éléments sont linéairement indépendents.
		\item Un frame est une suite de Bessel dont les éléments sont générateurs.
	\end{itemize}
\end{remarque}
Ainsi, lorsque l'on dispose d'une suite $F = \{f_i\}_I$ on peut définir l'opérateur d'analyse
\begin{equation*}
	\theta_F (f) = \{\langle f, f_i \rangle\}_I 
\end{equation*}
et de synthèse
\begin{equation*}
	\theta^*_F( \{c_i\}_I) = \sum_I c_i f_i.  
\end{equation*}
Ainsi la composée des deux opérateurs nous donne un opérateur de projection dans l'espace vectoriel engendré par $F$ :
\begin{equation}\label{eq:thetaF}
	\theta^*_F \circ \theta_F (f) = \sum_I \langle f, f_i\rangle f_i. 
\end{equation}
Tout d'abord on peut remarquer que, si $F$ est une famille orthogonale, alors l'application précédente correspond presque à une projection orthogonale dans l'espace engendré par $F$\footnote{La propriété qui n'est pas vérifiée est $(\theta^*_F \circ \theta_F)^2 = Id_{Vect(F)}$}.
On va maintenant voir que si $F$ est un frame \textit{équilibré} (c'est à dire avec des constantes $m, M$ égales), alors on dispose d'une formule analogue à \ref{eq:thetaF} qui nous donne une projection orthogonale.
L'intérêt de cela étant que, si $F$ est génératrice de l'espace entier $H$, alors la projection orthogonale correspond à une formule de reconstruction.
\newline
Supposons ainsi que l'on ait $m =M$, on a d'après \ref{eq:defFrame}, 
\begin{equation*}
	\sum_I |\langle f_j, f_i \rangle |^2 = M||f_j||^2.
\end{equation*}
Posons $\pi = \frac{1}{M}\theta^*_F \circ \theta$ et vérifions que c'est une projection orthogonale, soit $f\in Vect(F)$, alors
$f = \sum_J \lambda_j f_j$ avec $J\subset I$, d'où,
\begin{equation*}
	\langle f, f_k \rangle = \sum_J \lambda_j \langle f_j, f_k \rangle = \lambda_k \langle f_k, f_k \rangle + \sum_{j \in J-\{k\}} \lambda_j \langle f_j, f_k \rangle  
\end{equation*}
\begin{equation*}
	\pi(f) = \frac{1}{M}\sum_I \langle f, f_i \rangle f_i 
	= \frac{1}{M}\sum_J \lambda_j \sum_I \langle f_j, f_i \rangle f_i  
\end{equation*}
et pour conclure, on projète $\pi(f)$, sur chaque composante $f_k$ et on obtient
\begin{align*}
	\langle \pi(f), f_k \rangle &= \frac{1}{M} \sum_J \lambda_j \sum_I \langle f_j, f_i \rangle \langle f_i, f_k \rangle 
	= \frac{1}{M}  \lambda_k \sum_I |\langle f_k, f_i\rangle|^2 + \frac{1}{M} \sum_{j \in J-\{k\}}\lambda_j \sum_I  \langle f_j, f_i \rangle \langle f_i, f_k \rangle \\
	&= \frac{1}{M} \lambda_k M \langle f_k, f_k \rangle + \frac{1}{M} \sum_{j \in J-\{k\}} \lambda_j M\langle f_j, f_k \rangle \\
	&= \langle f, f_k \rangle.
\end{align*}
On a donc, si $f$ est dans l'espace engendré par F, alors la projection ne change pas les coordonées de $f$, sinon, si $f$ est dans l'orthogonal de $F$, alors chacune de ses composantes est orthogonale à tous les $f_i$, donc $f$ est dans le noyau de $\pi$.
Ainsi, $\pi$ est bien une projection orthogonale dans $F$.
\newline
On dispose donc d'une formule de reconstruction qui est valable pour tout $f$ qui est dans l'espace engendré par $F$ 
\begin{equation}
	f = \frac{1}{M} \sum_{f_i \in F} \langle f, f_i \rangle f_i 
\end{equation}
cependant cette formule ne semble rajouter que des complications par rapport à une base orthonormale (cas $m=M=1$ d'après la combinaison des théorèmes \ref{th:orth1} et \ref{th:norm1}). 
Nous allons donc voir ci-dessous en quoi avoir un coefficient de frame équilibré $M>1$ permet d'améliorer la stabilité de la formule de reconstruction, on appelera un tel frame redondant.
Ainsi, comme aperçu par Jean Morlet dès 1986 (TODO: ajouter ref Daubechies) travailler avec des frames permet, en pratique, de pouvoir stocker des coefficients de frame avec moins de précision.
Avant de poursuivre et de prouver cette observation de Jean Morlet, voyons d'abord rapidement comment obtenir la redondance d'un frame équilibré d'un point de vue formel.
Soit $F=(f_i)_{i=1, \cdots, N}$ un frame équilibré et on considère $Vect(F)$ l'espace vectoriel engendré par $F$, qui est un espace vectoriel de dimension $d$. 
Si les éléments de $F$ ne sont pas linéairement indépendants, alors on a $d<N$ et $M>1$.
Montrons avec le lemme suivant la relation entre le nombre d'éléments dans un frame équilibré ($N$), la constante de frame ($M$) et la dimension de l'espace engendré par le frame ($d$).
\begin{lemme}
	Soit $\Phi =(\varphi_i)_{i=1, \cdots, N}$ un frame avec des constantes de frame $m=M$ qui engendre $Vect \Phi$ un espace vectoriel de dimension $d$.
	Alors
	\begin{equation}
		\frac{N}{M^2} \leq d?
	\end{equation}
\end{lemme}
\begin{proof}
	Afin de prouver ce résultat rappelons que l'on a l'opérateur d'analyse
	\begin{align}
		\mathbb{R}^d &\longrightarrow^A \ell^2(N)\\
		f \longmapsto (\langle f, \varphi_i \rangle)_{i=1,\cdots, N}
	\end{align}
	et le frame étant équilibré, la boule $B_d(0,1)$ dans $\mathbb{R}^d$ de rayon 1 est envoyée sur la boule $B_N(0,M)$ de $\ell^2(N) := \ell^2(\{0, \cdots, N\})$.
	Maintenant, on considère l'opérateur de synthèse,
	\begin{align}
		\ell^2(N) &\longrightarrow^S \mathbb{R}^d\\
		x=(x_i)_{i=1, \cdots, N} &\longmapsto \sum_{i=1}^N x_i \varphi_i
	\end{align}
	et on considère, sans perte de généralité que chaque $\varphi_i$ est normalisé. Ainsi chaque à chaque $\varphi_i$, on peut associer une suite $(\lambda_j^i)_{j=0, \cdots, d}=(\langle \varphi_i, e_j)_{j=0,\cdots, d}$ de norme 1 dans $\mathbb{R}^d$ muni de la norme euclidienne telle que $\varphi_i = \sum_{j=1}^d \lambda_j^i e_j$ où $e_j$ est une base orthonormale de $\mathbb{R}^d$. 
	L'objectif est maintenant de réécrire $S(x)$ dans la base orthonormale, on a ainsi
	\begin{equation}
		\sum_{i=1}^N x_i \varphi_i = \sum_{i=1}^N x_i \sum_{j=1}^d \lambda_j^i e_j = \sum_{j=1}^d \sum_{i=1}^N x_i \lambda_j^ie_j =\sum_{j=1}^d c_j e_j,
	\end{equation}
	où $c_j = \sum_{i=1}^Nx_i\lambda_j^i$. On va maintenant majorer les $c_j$ de façon uniforme, chaque $\lambda_j^i$ correspond à la projection de $\varphi_i$ sur $e_j$, chaque $\varphi_i$ est dans la boule unité ainsi $c_j$ correspond à la projection de tous les $\varphi_i$ sur $e_j$, ainsi on a la majoration
	\begin{equation}
		c_j \leq ||x||_2 M.
	\end{equation}
	On a maintenant la majoration,
	\begin{equation}
		||S (x)||^2_2 = ||\sum_{j=1}^d c_j e_j||_2^2= \sum_{j=1}^d c_j^2 \leq d M^2||x||_2^2. 
	\end{equation}
	On va maintenant chercher une minoration de $||S(x)||_2^2$,
	\begin{equation}
		||S(x)||_2^2 = ||\sum_{i=1}^N x_i \sum_{j=1}^d \lambda_j^i e_j||_2^2 \geq N||x||_2^2 \min_{i=1, \cdots, N} ||\sum_{j=1}^d \lambda_j^i e_j||_2^2 = N||x||_2^2 \min_{i=1, \cdots, N}\sum_{j=1}^d (\lambda_j^i)^2.
	\end{equation}
	Or,
	\begin{equation}
		\sum_{i=1}^d (\lambda_j^i)^2 = \sum_{i=1}^d \langle \varphi_j, e_i \rangle =||\varphi_j||_2^2=1.  
	\end{equation}
	On peut donc combiner les inégalités obtenues et on a
	\begin{equation}
		N ||x||_2^2 \leq ||S(x)||_2^2 \leq d M^2||x||_2^2
	\end{equation}
	et ainsi
	\begin{equation}
		\frac{N}{M^2} \leq d.
	\end{equation}
\end{proof}
Fixons maintenant un élément $f\in  Vect(F)$,
supposons tout d'abord que l'on est dans le cas classique où ${e_i}_{i=1, \cdots, d}$ est une base orthonormale de $Vect(F)$, on a ainsi la formule standard,
\begin{equation*}
	f = \sum_{i=1}^d \langle f, e_i \rangle e_i = \sum_{i=1}^d c_i e_i
\end{equation*}
Maintenant perturbons les coefficients de la façon suivante, fixons un $\epsilon >0$ qui nous permettra de contrôler la taille de la perturbation introduite, et prenons $d$ variables aléatoires réelles indépendantes $(\alpha_i)_{i=1,\cdots, d}$ de moyenne nulle et de variance égale à 1.
On peut alors perturber chaque coefficient $c_i$ en y ajoutant $\epsilon \alpha_i$, on peut alors calculer l'erreur de reconstruction moyenne
\begin{align*}
	\mathbb{E}\left(|| f- \sum_{i=1}^d (c_i + \epsilon \alpha_i)e_i||_2^2\right) &= \mathbb{E}\left(\sum_{i=1}^d \epsilon^2 \alpha_i^2\right)\\
		&= \epsilon^2 \sum_{i=1}^d \mathbb{E} (\alpha_i^2) = \epsilon^2 d.
\end{align*} 
Maintenant appliquons la même altération aux coefficients de $f$ dans le frame $F$ afin de calculer l'erreur de reconstruction moyenne.
Ainsi, on a la formule de reconstruction
\begin{equation*}
	f = \frac{1}{M}\sum_{i=1}^N \langle f, f_i \rangle f_i
\end{equation*}
et prenons $N$ variables aléatoires réelles indépendantes $(\alpha_i)_{i=1, \cdots, N}$  de moyenne nulle et de variance égale à 1.
\begin{align*}
	\mathbb{E}\left(|| f- \frac{1}{M}\sum_{i=1}^N (\langle f, f_i\rangle + \epsilon \alpha_i)e_i||_2^2\right) &= \mathbb{E}\left(||\frac{1}{M} \sum_{i=1}^N \epsilon \alpha_i||_2^2\right)\\
	&= \frac{\epsilon^2}{M^2} \mathbb{E}(\sum_{i=1}^N \alpha_i^2) = \frac{\epsilon^2 N}{M^2}.	
\end{align*}
Ainsi, d'après le lemme précédent, l'erreur de reconstruction moyenne d'un frame redondant est toujours au moins aussi petite que l'erreur de reconstruction dans une base orthonormale.
Montrons avec un exemple dans $\mathbb{R}^2$ que l'on peut faire mieux que le résultat du lemme.
Soit $\varphi_1 = (0, 1), \varphi_2 = (-\frac{\sqrt{3}}{2}, -\frac{1}{2}), \varphi_3 =(\sqrt{3}{2}, -\frac{1}{2})$ et prenons $x=(x_1, x_2)$ un élément de $\mathbb{R}^2$, alors,
\begin{align*}
	\sum_{i=1}^3 |\langle x, \varphi_i \rangle|^2 &= |x_2|^2 + |-\frac{\sqrt{3}}{2}x_1 -\frac{1}{2} x_2|^2 + |\frac{\sqrt{3}}{2}x_2 - \frac{1}{2}x_1|2 \\
	&= \frac{3}{2}(|x_1|^2 + |x_2|^2) = \frac{3}{2} ||x||_2^2.
\end{align*}
Ainsi, les vecteurs $(\varphi_1, \varphi_2, \varphi_3)$ forment un frame équilibré de constante $M=\frac{3}{2}$ de $\mathbb{R}^2$.
On a alors $\frac{N}{M^2} = \frac{3}{M^2} = \frac{4}{3} < 2 = d$.
Donc dans le cas de ce frame, l'erreur de reconstruction moyenne après perturbation des coefficients est améliorée d'un facteur $\frac{2}{3}$ par rapport à celle que l'on obtiendrait en utilisant une base orthonormale de $\mathbb{R}^2$.

\subsection{Exemples de frames (Fourier, ondelettes)}
Introduisons maintenant la décomposition d'une fonction en ondelettes.
\begin{definition}
	On dit que $\psi:\mathbb{R} \to \mathbb{R}$ est une ondelette génératrice si 
	\begin{equation}
		\{\psi_{j,k} := 2^{j/2}\psi(2^j\cdot - k)\}_{j,k \in \mathbb{Z}}
	\end{equation}
	est une famille génératrice de $L^2(\mathbb{R})$.
	On appelera\footnote{On considérera par la suite des frames d'ondelettes ou bien des bases de Riesz d'ondelettes} une base engendrée par une telle fonction $\psi$ une base d'ondelettes.
\end{definition}
\begin{figure}
	\includegraphics{Figs/shannon}
\end{figure}
\begin{remarque}
	Dans la définition des ondelettes $\psi_{j,k}$ engendrées par $\psi$, le coefficient $j$ correspond au facteur d'échelle\footnote{Du point de vue des notations, on considère que $j$ tend vers l'infini, signifie que $\psi_j$ analyse les hautes fréquences, ce choix de notation n'est pas uniforme dans la littérature, par exemple (TODO :ajouter ref) Mallat et Daubechies, utilisent $-j$ par rapport à nos notations. Par contre les notations utilisées correspondent à celles de Jaffard et Meyer. Mais cependant tous les résultats sont bien entendu équivalents.},
	en raison du facteur $2^j$ devant la variable, au fur et à mesure que $j$ augmente, l'ondelette est parcourue de plus en plus vite. Ainsi, augmenter $j$ revient à augmenter la fréquence de $\psi$, c'est à dire d'éloigner le support de $\hat{\psi}$ de l'origine\footnote{En effet, de façon plus précise et formelle, on a $\hat{\psi_{j,0}}(\omega) = 2^{-\frac{j}{2}}\hat{\psi}(\frac{\omega}{2^j})$.}.
	Le coefficient $k$ correspond à une translation de l'ondelette $\psi_j$, en ce sens, l'analyse par ondelette, permet une analyse à la fois en temps (par rapport à $k$) et en fréquence (par rapport à $j$).
\end{remarque}
Donc étant donnée une famille d'ondelettes $\{\psi_{j,k}\}_{j,k \in \mathbb{Z}}$, on peut associer à une fonction $f\in L^2(\mathbb{R})$, ses coefficients d'ondelettes
\begin{equation}
	Wf = (\langle f, \psi_{j,k} \rangle )_{j,k \in \mathbb{Z}}
\end{equation}
et on se demande alors si à partir de ces coefficients on peut reconstruire $f$.
Cela revient ainsi à déterminer si la famille d'ondelettes est un frame d'ondelette. 
Ici on ne cherchera pas a énumérer et a vérifier des frames d'ondelettes, un très grand nombre de frames d'ondelettes existent (ajouter ref.), on admet ainsi pour l'instant l'existence des frames d'ondelettes.
Un peu plus bas nous verrons et démontrerons un théorème qui permet de construire de nombreux frames d'ondelettes, ce théorème fournira également une extension des ondelettes à $L^2(\mathbb{R}^d)$ et avec une version qui montre que la présence de puissances de 2 dans la définition des ondelettes revient à un choix d'échantillonage. 
\newline
Supposons ainsi que l'on dispose d'un frame d'ondelettes et que ce frame est équilibré (c'est à dire que les bornes de frame $m$ et $M$ sont égales), alors on dispose d'une formule de reconstruction (d'après Daubechies 3.2.2)
\begin{equation}
	f = \frac{1}{M} \sum_{j,k} \langle f, \psi_{j,k}\rangle \psi_{j,k}.
\end{equation}
Cependant bien que la formule précédente permette une reconstruction elle suppose de parcourir des indices sur $\mathbb{Z}$ ce qui pourrait créer des complications concernant la convergence (d'un point de vue théorique ou pratique).
On va ici très rapidement introduire la notion d'analyse multi-échelle qui permet de simplifier la formule de reconstruction.
Construisons ici une analyse multi-échelle (ici de $L^2(\mathbb{R})$), considérons tout d'abord une suite d'espaces emboités satisfaisant 
\begin{equation*}
	\{0\}=	\lim_{j\to -\infty} \bigcap_{j}^{+\infty} V_i \subset \cdots \subset V_{-1} \subset V_0 \subset \cdots V_i \subset V_{i+1} \subset \cdots \subset \lim_{j\to \infty} \bigcup_{-\infty}^{j} V_i = L^2(\mathbb{R}).  
\end{equation*}
L'intérêt d'avoir une telle suite d'espaces emmboités est que étant donnée une fonction $f\in L^2(\mathbb{R})$, on peut considérer sa projection orthogonale dans un $V_i$, on a alors une approximation $f_i$ de $f$ dans $V_i$, si on souhaite améliorer l'approximation de $f$ il suffit alors de remonter dans ces espaces emboités pour avoir une reconstruction avec une précision arbitraire.
Introduisons maintenant la propriété qui va permettre de voir cette suite d'espaces comme une analyse multi-échelle
\begin{equation}
	f(\cdot) \in V_j \iff f(\frac{\cdot}{2^j}) \in V_0,
\end{equation}
c'est à dire que les fonctions d'un espace $V_j$ sont des versions dilatées d'un facteur $2^{-j}$ des fonctions de l'espace $W_0$.
Ajoutons maintenant la condition que $V_0$ contient toutes les translations entières de ses éléments, c'est à dire
\begin{equation}
	f \in V_0 \iff f(\cdot - n) \in V_0 \forall n \in \mathbb{Z}.
\end{equation}
Ainsi, une fonction qui appartient à $V_j$ s'écrit comme une combinaison linéaire de versions translatées et dilatées de fonctions appartenant à $V_0$.
De plus, quelque soit $j$ on peut prendre $W_j$ le complémentaire orthogonal de $V_j$ dans $V_{j+1}$, 
\begin{equation}
	f = \pi_{V_0}(f) +\sum_{i>0} \pi_{W_j}(f) 
\end{equation}
donc afin d'avoir une formule de reconstruction, il suffit de connaitre un frame de $V_0$ et de même pour chaque $W_j$. 
On peut maintenant revenir aux ondelettes, on considère que l'on connait une base $(\varphi_{0,k})_{k\in \mathbb{Z}}$ de $V_0$ et $(\psi_{j,k})_{j\in \mathbb{N}^*, k \in \mathbb{Z}}$ un frame de l'orthogonal de $V_0$ dans $L^2(\mathbb{R})$, on pose alors $W_j = W_{j-1} \bigoplus Vect(\{\psi_{j,k}\}_{k\in \mathbb{Z}})$ et on obtient ainsi que l'analyse multi-échelle ainsi construite fournit une formule de reconstruction\footnote{Dans la formule de reconstruction les deux sommes sur $\mathbb{Z}$ ne sont pas problématiques car les fonctions considérées sont dans $L^2(\mathbb{R})$ donc avec une décroissance suffisament rapide, donc seulement un nombre fini de $\langle f, \varphi_{0,k} \rangle$ sont différents de 0 si $\varphi_0$ a une décroissance suffisament rapide (et de même pour chaque $\psi_j$).}
\begin{equation}
	f = \sum_{k\in \mathbb{Z}} \langle f, \varphi_{0,k} \rangle \varphi_{0,k} + \sum_{j = 1}^{+\infty} \sum_{k\in \mathbb{Z}} \langle f, \psi_{j,k} \rangle \psi_{j,k}.
\end{equation}
On appelle l'application $\varphi_0$ ondelette d'échelle.
On a vu que les familles orthonormales forment des frames, on a donc la proposition :
\begin{proposition}
	\begin{enumerate}
		\item La transformée de Fourier discrète est un frame pour $\mathcal{F}$.
		\item Les ondelettes forment un frame pour $\mathcal{F}$.
	\end{enumerate}
\end{proposition}
\begin{preuve}
	\begin{enumerate}
		\item On utilisera la transformée de Fourier discrète écrite sous forme de matrice unitaire, qui est une matrice unitaire de Vandermonde avec les racines de l'unité en coefficients. Avec de l'algèbre on montre l'orthonormalité des colonnes.
		\item Application du théorème suivant.
	\end{enumerate}
\end{preuve}

\begin{theoreme}
	Soit $\mathcal{Q} \subset \mathcal{R}^d$ un ensemble de mesure finie, $h \in L^2(\mathbb{R}^d)$
	et $\mathcal{A} =\{A_j \in GL_d(R)\}_J$ une famille de matrices inversibles.
	\newline
	Pour tout $j \in J$, on pose $B_j =(A_j^T)^{-1}, S_j = A_j^TQ, h_j = h(B_j \cdot)$
	et soit $\mathcal{S} = \{S_j\}_J$.
	\newline
	On suppose que $\mathcal{S}$ est un recouvrement de $\mathbb{R}^d$, $\mathcal{H}$ est une partition de Riesz de l'unité avec des bornes $p$ et $P$ et que $Supp(h) \subset Q$.
	\newline
	Soit $X = \{x_{j,k} \in \mathbb{R}^d : j\in J, k \in K\}$ tel que quelque soit $j \in J$, 
	l'ensemble $\{e_{x_{j,k}}\chi_Q\}_K$ forme un frame pour $\mathcal{K}_Q$ avec des bornes $m_j$ et $M_j$. 
	\newline
	Si $m := \inf_J m_j > 0$ et $M= \sup_J M_j < \infty$, alors la collection
	\begin{equation*}
		\{|\det A_j|^{1/2} \psi(A_j x - x_{j,k})\}_{J, K}
	\end{equation*}
	forme un frame d'ondelettes de $L^2(\mathbb{R}^d)$ avec des bornes $mp$ et $MP$, 
	engendré par une seule fonction $\psi$ où $\psi$ est la transformée de Fourier inverse de $h$.
\end{theoreme}
\begin{preuve}
	Voir \cite{IrregWav} pour la preuve et les définitions, je les ajouterais ici et au dessus plus tard %%TODO 
\end{preuve}
\section{Le théorème de Shannon-Nyquist}
\subsection{L'échantillonage selon Shannon d'un signal à support compact en fréquence}
\subsection{L'échantillonage selon Shannon d'un signal k-sparse}



\chapter{Approche parcimonieuse}
\section{Résolution de (P0)}
Dans ce qui précède, nous nous sommes intéressés aux propriétés qui font qu'une famille de vecteurs permet de reconstruire une famillle de signaux.
Nous avons vu différentes bases (Fourier et ondelettes) et nous avons vu que ces bases permettent de reconstruire des signaux présentant un certain type de régularité avec des coefficients qui suivent une décroissance assez rapide.
\newline
On a par exemple vu que l'on pouvait reconstruire les fonctions Lipschitziennes avec une bonne précision en utilisant une base d'ondelettes orthonormale avec un certain nombre de moments nuls.
De plus, on a remarqué que si la fonction se comporte comme un polynôme d'un degré inférieur au nombre de moments nuls au voisinage d'un point, alors les coefficients d'ondelettes dans ce voisinage seront nuls.
De même, les seuls coefficients d'ondelette qui seront grands seront ceux au voisinage d'un point où aucune approximation par un polynôme de petit degré n'est efficace\footnote{Une analyse du théorème de Jaffard \ref{th:Jaffard} montre que les coefficients affectés par une discontinuité forment un cône dans les coefficients d'ondelette autour du point de discontinuité. Ce cône se visualise dans la représentation temps-fréquence des coefficients d'ondelette, il part du point de discontinuité et s'élargit en diminuant le coefficient d'échelle $j$. La largeur de ce cône dépend de la régularité $\alpha$ de la fonction.}.
Ainsi, la représentation avec ces ondelettes d'une fonction ne possédant que quelques points où elle est irrégulière sera approximée avec peu de coefficients.
Afin d'insister, l'intérêt de cela est que de façon naive, afin de déterminer une fonction, il faut connaitre sa valeur en chaque point, ainsi, si l'on souhaite faire un traitement par ordinateur de cette fonction, il faut stocker chacun des points de la fonction.
Avec ce que l'on a fait, on sait qu'en fait on peut reconstruire la fonction avec un plus petit nombre de coefficients que la fonction n'a de points.
En ce sens, la représentation en ondelettes d'une fonction Hölderienne est parcimonieuse (peu de coefficients non nuls), alors que la représentation par la valuation d'une fonction Hölderienne n'est pas parcimonieuse.
\newline
Nous allons maintenant nous intéresser à l'autre direction de ce problème, c'est à dire que nous allons supposer que l'on dispose d'une famille de vecteurs et que la fonction que l'on cherche à reconstruire est une somme parcimonieuse de vecteurs de cette famille.
Cependant on connait seulement la valuation de cette fonction et pas les vecteurs sous-jacents qui permettent de représenter la fonction de façon parcimonieuse.
Aussi, on n'a pas supposé que cette famille est libre donc il n'y a pas une unique façon d'obtenir cette solution, en fait il y a une infinité de solutions dès que la famille n'est pas libre.
On va voir cependant que l'hypothèse de parcimonie est cruciale et qu'elle nous permettra de récupérer exactement les coefficients qui permettent l'écriture parcimonieuse de cette fonction.
\subsection{Définition de (P0)}
Formalisons maintenant ce que nous avons dit ci-dessus. 
On considère $\mathcal{F}$ un espace vectoriel et utilisons un dictionnaire $\Phi = \Phi_1 \cup \cdots \cup \Phi_D$ de bases, où chaque $\Phi_d$ est une base de $\mathcal{F}$. 
Ainsi $\Phi$ est une concaténation de bases\footnote{Ainsi $\Phi$ est un frame équilibré d'après la première partie (TODO : ajouter ref)} et on s'intéresse aux façon d'écrire un signal $f\in \mathcal{F}$ dans $\Phi$, c'est à dire aux façon d'écrire
\begin{equation}\label{eq:defSSum}
	f = \sum_\gamma c_\gamma \phi_\gamma
\end{equation}
où l'indice $\gamma = (d, i)$ indique le dictionnaire $\Phi_d$ correspondant ainsi que le vecteur $\phi_{d, i} \in \Phi_d$.
On peut aussi écrire \ref{eq:defSSum} sous forme matricielle en posant $F_\Phi$ la matrice ayant pour lignes les vecteurs $\phi_\gamma$ et en posant $x = (c_\gamma)_\gamma$ la notation sous forme de vecteur de $x$, on utilisera aussi la notation $x = (x_d)_{d=1, \cdots, D}$
On s'intéresse ainsi aux solutions de 
\begin{equation}
	f = F_\Phi x.
\end{equation}
Comme discuté précedemment, le choix des coefficients $c_\gamma$ n'est pas unique dès que $D>1$, cependant notre objectif n'est pas simplement de reconstruire $f$ (car n'importe quelle base $\Phi_i$ permet déjà cela), mais de trouver l'écriture de $f$ avec le minimum de coefficients non nuls.
Ainsi, le problème que l'on cherche à résoudre est 
\begin{equation}\label{P0}\tag{P0}
	\min ||x||_0\quad \text{tel que } f = F_\Phi x,
\end{equation}
où $||x||_0 = \#\{\gamma : c_\gamma \neq 0\}$ est le nombre de coefficients non nuls de $x$.
Cependant, la résolution en toute généralité de ce problème n'est pas faisable, en effet résoudre ce problème nécessite de résoudre (P0) pour chaque combinaison de vecteurs du dictionnaire si $x$ est dans l'image.
Ainsi, le nombre de combinaisons possibles parmi tous les vecteurs croit bien trop vite pour être calculable en pratique, nous verrons donc comment résoudre ce problème en utilisant une autre méthode.
\newline
Il est important de noter qu'à ce stade il n'y a aucune raison de supposer que chercher une unique solution à (P0) a un sens.
En effet, quand on a choisi le dictionnaire $\Phi$ rien ne nous interdisait de prendre à chaque fois la même base et on aurait ainsi $D$ solutions identiques, ayant chacune la même parcimonie.
On a ainsi $D$ solutions, et si on prend une paire de solutions $x_1, x_2$, alors $F_\Phi(x_1 - x_2) = 0$, d'où on obtient qu'à n'importe laquelle des $D$ solutions, on peut ajouter, par exemple $x_1 - x_2$, et on obtient une nouvelle solution. 
Cependant, cette solution ne sera jamais moins parcimonieuse que l'une des $D$ solutions initiales.
Il est donc clair qu'il est nécessaire d'imposer des conditions sur les bases qui constituent le dictionnaire si l'on souhaite obtenir une solution unique.
Afin d'étudier cela commençons par un cadre simple dans lequel résoudre $P_0$ a un sens.
\begin{exemple}
	On étudie les signaux dans $\mathbb{R}^N$ et on choisi un dictionnaire constitué de la concaténation de la base de Fourier $ W = \{e_k(t) = \frac{1}{\sqrt{N}} e^{\frac{i 2\pi k t}{N}}\}_{0 \leq k \leq N -1}$ et de la base canonique de Diracs\footnote{Chaque vecteur de cette base vérifie $\delta_{k,i} = 1$ si $i = k$ et 0 sinon.} $T = \{\delta_k\}_{0 \leq k \leq N-1}$.
	Ainsi, avec ce choix $F_W$ est la matrice de Fourier discrète et $F_T$ est la matrice identité de taille $N$.
	Avec le théorème suivant, on va obtenir un principe d'incertitude, qui nous garantira qu'un signal ne peut pas être parcimonieux à la fois dans la base de Dirac, et dans la base de Fourier.
	\begin{theoreme}\label{th:Incert1}
		Soit un signal $f\in \mathbb{R^N}$ non nul, alors
		\begin{equation}
			||F_W f||_0 ||F_T f||_0 \geq N 	
		\end{equation}
		et ainsi
		\begin{equation}
			||F_W f||_0 + ||F_T f||_0 \geq 2 \sqrt{N}. 	
		\end{equation}
	\end{theoreme}
	\begin{proof}
	TODO :Ajouter Ref Tao uncertainty principle for cyclic....
		Soit $0 \leq \omega \leq N-1$ un entier, alors 
		\begin{align}
			|F_W(\omega)| &= |\hat{f}(\omega)| = \frac{1}{\sqrt{N}}|\sum_t f(t)e_\omega(t)| \\
				&\leq \frac{1}{\sqrt{N}}\sum_t |f(t)|,
		\end{align}
		d'où $\sup_\omega |F_W(\omega)| \leq \frac{1}{\sqrt{N}} \sum_t |f(t)|$.
		On pose maintenant $sign(f) = (\frac{f(t)}{|f(t)|})_t$ pour tous les $t$ tels que $f(t) \neq 0$ et 0 si $f(t) = 0$ et on a ainsi $\langle sign(f), sign(f) \rangle = ||F_T f||_0$, on va ainsi pouvoir montrer le théorème en utilisant successivement l'inégalité de Cauchy-Schwarz puis l'égalité de Parseval,
		\begin{align}
			\sup_\omega|F_W(\omega)| &\leq \frac{1}{\sqrt{N}}\sum_t |F_Tf (t)| = \frac{1}{\sqrt{N}}\langle sign(f), |f| \rangle \\
			&\leq  \frac{1}{\sqrt{N}} ||F_T||_0^{\frac{1}{2}} \langle |f|, |f| \rangle ^{\frac{1}{2}} =  \frac{1}{\sqrt{N}} ||F_T||_0^{\frac{1}{2}} \langle |F_W f|, |F_Wf| \rangle ^{\frac{1}{2}} \\
			&\leq   \frac{1}{\sqrt{N}} ||F_T||_0^{\frac{1}{2}} \langle |f|, |f| \rangle ^{\frac{1}{2}} =  \frac{1}{\sqrt{N}} ||F_T||_0^{\frac{1}{2}} \left( \sum_\omega|\hat{f}(\omega)|^2 \right)^{\frac{1}{2}} \leq \frac{1}{\sqrt{N}}||F_T||_0^{\frac{1}{2}} ||F_W f||_0^{\frac{1}{2}} \sup_\omega |F_W(\omega)|.
		\end{align}
		On a ainsi montré la première partie du théorème, la deuxième partie provient directement de l'inégalité entre la moyenne arithmétique et la moyenne géométrique.
		En effet, on a 
		\begin{equation}
			\sqrt{||F_T f||_0 ||F_W f||_0} \leq \frac{||F_T f||_0 + ||F_W f||_0}{2}
		\end{equation}
	et on a déjà montré que le terme de gauche est supérieur ou égal à $\sqrt{N}$.
	\end{proof}
	Observons que sans restrictions sur $N$, l'inégalité obtenue ne peut pas être améliorée, comme observé dans (Donoho-Stark 89 /Donoho-Huo TODO: ajouter ref), si $N$ est un carré, alors, la fonction avec des 1 seulement aux coefficients multiples de $\sqrt{N}$ et 0 ailleurs est sa propre transformée de Fourier et ainsi elle a $2\sqrt{N}$ coefficients dans le dictionnaire $(T,W)$ et ainsi l'inégalité est atteinte.
	Une conséquence de cela est qu'une condition sur la parcimonie de la forme $||F_T f||_0 + ||F_W||_0 < K$ avec $K>\sqrt{N}$ ne pourra pas garantir l'unicité de la solution de (P0).
	Montrons que si $K = \sqrt{N}$ alors on a l'unicité de la solution de (P0).
	\begin{theoreme}\label{th:Incert2}
		Soit $N$ un entier positif et un signal $f \in \mathbb{R}^N$, alors n'importe quel $x$ vérifiant $f = F_\Phi x = F_W x_W + F_T x_T$ et 
		\begin{equation}\label{eq:Incert1}
			||F_W x_W||_0 + ||F_T x_T||_0 < \sqrt{N}
		\end{equation}
		est l'unique solution de (P0).
	\end{theoreme}
	Supposons que pour $f$ donné non nul et supposons que l'on ait deux solutions de (P0), $x_1$ et $x_2$, ainsi $f = F_\Phi x_1, f = F_\Phi x_2$ et on a aussi $||x_1||_0 < \sqrt{N}, ||x_2||_0 < \sqrt{N}$.
	On a par linéarité de l'opérateur $F_\Phi$, 
	\begin{equation}
		F_\Phi( x_1 - x_2) = 0.
	\end{equation}
	Etudions ainsi les éléments du noyau de $F_\Phi$, posons $\mathcal{N} = \{\delta : F_\Phi \delta = 0\}$, et pour tout $\delta \in \mathcal{N}$, écrivons $\delta = (\delta_T, \delta_W)$, on a
	\begin{equation}
		F_T \delta_T + F_W \delta_W = 0
	\end{equation}
	ainsi, en utlilisant que les colonnes de $F_W$ forment une base, donc $F_W$ est une matrice orthogonale, on a 
	\begin{equation}\label{eq:structN}
		\delta_W = -F_W^t F_T \delta_T .
	\end{equation}
	On a donc montré que les éléments de $\mathcal{N}$ sont de la forme $\delta = (\delta_T, -F_W^t F_T \delta_T)$ et d'après le théorème \ref{th:Incert1}, on a que $\delta$ a au moins $2\sqrt{N}$ coefficients non nuls si $\delta$ est non nul. 
	En revenant a la situation initiale $\delta = x_1 - x_2$, on a une contradiction car à la fois $x_1$ et $x_2$ ont chacun moins de $\sqrt{N}$ coefficients, donc $\delta = 0$.
	Ainsi, si une solution existe avec moins de $\sqrt{N}$ coefficients, alors c'est la solution de (P0) et elle est unique.
	\newline
	Cependant, en choisissant $N = p$, où $p$ est un nombre premier\footnote{L'hypothèse $p$ premier est essentielle, la preuve reposant sur la non-existence de sous-groupes propres du groupe cyclique $\mathbb{Z}/p\mathbb{Z}$}, Tao 2005 (TODO: ajouter ref) a montré que l'on obtient l'inégalité
	\begin{equation}
		||F_T f||_0 + ||F_W f||_0 \geq p + 1
	\end{equation}
	et que l'inégalité est atteinte\footnote{L'inégalité est atteinte en ce sens que si $A\subset T$ et $B\subset W$ tels que $|A| + |B| \geq p+1$ alors il existe une fonction $f$ telle que $Supp F_T f = A$ et $Supp F_W f = B$}.
	Grâce à ce principe d'incertitude plus fort que \ref{th:Incert1}, on obtient avec le même type de preuve\footnote{Voir Candes-Romberg-Tao pour les détails, un lemme sur l'injectivité d'un opérateur similaire $F_W$ est tout de même nécessaire pour conclure la preuve.} le résultat suivant
	\begin{theoreme}
		Soit $N$ un nombre premier et un signal $f \in \mathbb{R}^N$, alors n'importe quel $x$ vérifiant $f = F_\Phi x = F_W x_W + F_T x_T$ et
		\begin{equation}\label{eq:Incert2}
		||F_T x_T||_0 \leq \frac{N}{2}
		\end{equation}
		est l'unique solution de (P0).
	\end{theoreme}
	On a ainsi vu qu'avec un dictionnaire constitué de Fourier et de Dirac la solution de (P0) est unique, on a également vu brièvement, qu'en renforçant le principe d'incertitude sur les deux familles, alors on peut certifier qu'on a bien obtenu \it{la} solution de (P0) pour des signaux avec un support plus grand.
\end{exemple}
\subsection{Solution optimale combinatoire}
\subsection{Résolution dans un dictionnaire pics/Fourier}
\subsection{Principe d'incertitude}

\section{Résolution de (P1)}
On a ainsi vu dans la section précedente que le problème (P0) de minimisation de la solution par rapport à la parcimonie admet une solution unique dès qu'une solution existe et que cette solution vérifie une condition de la forme \ref{eq:Incert1} ou \ref{eq:Incert2}.
Cependant, on a aussi vu au début de la section précédente que le problème (P0) est un problème de nature combinatoire et le nombre de combinaisons possibles augmentant très vite par rapport à $N$, sa résolution n'est pas faisable et ainsi il est nécessaire d'avoir une autre approche à ce problème.
\newline
La découverte qui a permis de rendre la résolution faisable, et par là permis par exemples les avancées du compressed sensing qui ont eu de nombreuses applications et dont la théorie sera étudiée dans le prochain chapitre, est que l'on peut résoudre un autre problème pour lequel des méthodes de résolution efficaces existaient déjà.
En effet nous allons voir que résoudre le problème \ref{P1},
\begin{equation}\label{P1}\tag{P1}
	min_x ||x||_1 \text{tel que } f = Fx 
\end{equation}
permet sous certaines conditions de résoudre \ref{P0}.
L'intéret de \ref{P1} est que c'est un problème de programmation linéaire et de nombreuses méthodes permettent de le résoudre. (TODO ajouter refs et détails).
Précisons donc ce que nous avons affirmé, dans le même cadre que précedemment, c'est à dire dans le cas d'un dictionnaire $\Phi = (T,W)$ temps fréquence composé de la base de Fourier et de Dirac dans $\mathbb{R}^N$.
\begin{theoreme}\label{th:DiracFourier}
	Soit $N$ un entier positif, $\Phi = (T, W)$ est la concaténation des bases de Dirac et de Fourier et un signal $f\in \mathbb{R}^N$, alors n'importe quel $x = (x_T, x_W)$ vérifiant $f = F_T x_T + F_W x_W$ et
	\begin{equation}\label{eq:cond1}
		||x_T||_0 < \frac{\sqrt{N}}{2} \quad \text{et} \quad   ||x_W||_0 < \frac{\sqrt{N}}{2}
	\end{equation}
	est l'unique solution de \ref{P1}, et c'est la solution de \ref{P0}.
\end{theoreme}
\begin{remarque}
	Le théorème précédent a été obtenu en cherchant une preuve alternative à la preuve qui est faite par Donoho et Huo (TODO: ajouter ref), leur preuve, comme une grande partie de la section précédente, utilise le même schéma que celle qui est faite ici.
	Leur théorème est le suivant :
\begin{theoreme}
	Soit $N$ un entier positif, $\Phi = (T, W)$ est la concaténation des bases de Dirac et de Fourier et un signal $f\in \mathbb{R}^N$, alors n'importe quel $x = (x_T, x_W)$ vérifiant $f = F_T x_T + F_W x_W$ et
	\begin{equation}
		||x_T||_0 +  ||x_W||_0 < \frac{\sqrt{N}}{2}
	\end{equation}
	est l'unique solution de \ref{P1}, et c'est la solution de \ref{P0}.
\end{theoreme}
	La preuve qui est présentée utilise un lemme qui est une version affaiblie d'un résultat présenté dans l'article.
	Dans l'article, l'inégalité plus forte qui est utilisée est obtenue à l'aide d'un principe variationnel (TODO: préciser), mais comme les auteurs le remarquent, leur résultat ne semblait pas exact au sens où même lorsque l'inégalité est atteinte il n'y avait aucun contre-exemple apparent.
	En effet, dans le cas de la base de Fourier-Dirac, le peigne de Dirac, fournit dans certains cas un exemple de signal qui est supporté sur $\sqrt{N}$ coefficients soit dans la base de Fourier, soit dans la base de Dirac, ainsi le problème \ref{P0} a plusieurs solutions et donc une condition nécessaire pour résoudre simultanément \ref{P1} et \ref{P0} est $||x||_0 < \sqrt{N}$.
	Or, les hypothèses du théorème ne sont plus vérifiées dès que $||x||_0 = \sqrt{N}$ (car au moins, soit $x_T$, soit $x_W$ est supporté sur au moins $\frac{\sqrt{N}}{2}$ coefficients). 
\end{remarque}
\begin{proof}
	La preuve de ce théorème se fait en plusieurs parties.
	Tout d'abord, remarquons que si $x$ vérifie \ref{eq:cond1}, alors $x$ vérifie \ref{eq:Incert1} et donc d'après le théorème \ref{th:Incert2} $x$ est donc l'unique condition de \ref{P0}.
	Il nous faut donc vérifier que cette solution est bien la solution de (P1).
	On montre ensuite un lemme qui permet de donner une condition suffisante pour qu'une paire de bases vérifie que la solution obtenue est bien celle de \ref{P1}.
	On vérifiera ensuite que dans la paire de bases Fourier-Dirac, les conditions du lemme sont vérifiées et cela permettra de conclure la preuve du théorème.
	Avant d'énoncer le lemme, définissons une quantité $\mu$ qui mesure dans une paire de bases $\Phi=(T,W)$ à quel point un élément dans le noyau de $F_\Phi$ peut être supporté à la fois sur $T$ et sur $W$.
	\begin{definition}
		Soit $\Phi = (T,W)$ une paire de bases, on note $\mathcal{N} = \{\delta = (\delta_T, \delta_W): F_\Phi \delta = 0\}$, soit $\Gamma_T$ (resp. $\Gamma_W$) un ensemble d'indices de $T$ (resp. $W$), alors on pose
		\begin{equation}
			\mu(\Gamma_T, \Gamma_W) = \sup_{\delta \in \mathcal{N}} \frac{\sum_{t \in \Gamma_T} |\delta_{T,t}| + \sum_{\omega \in \Gamma_W} |\delta_{W,\omega}|  }{||\delta_T||_1 + ||\delta_W||_1 }
		\end{equation}
	\end{definition}
	\begin{lemme}\label{th:muP1}
		Soit un signal $f \in \mathbb{R}^N$ et $\Phi=(T,W)$ une paire de bases de $\mathbb{R}^N$, alors n'importe quel $x = (x_T, x_W)$, où $\Gamma_T$ est le support de $x_T$ et $\Gamma_W$ est le support de $x_W$, 	vérifiant $f = F_T x_T + F_W x_W$ et
		\begin{equation}\label{eq:condmu}
			\mu(\Gamma_T, \Gamma_W) < \frac{1}{2}
		\end{equation}
		est l'unique solution de \ref{P1}.
	\end{lemme}
	Pour prouver le théorème on vérifiera donc dans la base de Fourier-Dirac que pour n'importe quelle paires d'indices vérifiant les conditions du théorème alors l'inégalité \ref{eq:condmu} sera vérifiée, et ainsi la solution de \ref{P0} sera bien la même que celle de \ref{P1} ce qui permettra de conclure la preuve du théorème.
	Enonçons donc cela sous la forme d'un autre lemme
	\begin{lemme}\label{th:muFD}\footnote{C'est ce lemme dont il est fait mention dans la remarque précédant la preuve et qui permet la généralisation du théorème de Donoho et Huo.}
		Soit $\Phi=(T,W)$ la paire de bases Fourier-Dirac et soient $\Gamma_T$  et $\Gamma_W$ des sous ensembles d'indices de $T$ et respectivement de $W$ vérifiant
		\begin{equation}
			|\Gamma_T| < \frac{\sqrt{N}}{2} \quad \text{et} \quad |\Gamma_W| < \frac{\sqrt{N}}{2},
		\end{equation}
		alors on a,
		\begin{equation}
			\mu(\Gamma_T, \Gamma_W) < \frac{1}{2}.
		\end{equation}
	\end{lemme}
	Ainsi, une fois les lemmes démontrés, le théorème le sera aussi.
	\end{proof}
	Commençons par la preuve du lemme \ref{th:muP1}.
	\begin{proof}
		Supposons que $x$ vérifie les conditions du lemme, c'est à dire, $x$ est effectivement une solution de l'équation $f=F_\Phi x$ et la condition \ref{eq:condmu} est vérifiée sur $\Phi$, alors on doit donc montrer que $x$ est l'unique solution de (P1), on doit donc montrer que pour tout $x_1$ différent de $x$ qui vérifie $f = F_\Phi x_1$ alors $||x_1||_1 > ||x||_1$. 
		Donc de façon équivalente, pour tout $\delta \in \mathcal{N} = \{\delta : F_\Phi \delta = 0\}$ non nul, on doit vérifier que
		\begin{equation}\label{eq:ineqdelta3}
			||x + \delta||_1 - ||x|| > 0.
		\end{equation}
		Notons $\Gamma = \{\gamma : c_\gamma \neq 0\} = \Gamma_T \cup \Gamma_W \subset [0, 2N-1]$ l'ensemble des indices non nuls de $x = (c_\gamma)_\gamma$, 
		on peut donc décomposer la somme
		\begin{equation}
			||x + \delta||_1 - ||x||_1 = \sum_{\gamma \in \Gamma^c} |\delta_\gamma| + \sum_{\gamma \in \Gamma} |c_\gamma + \delta_\gamma| - |c_\gamma|.
		\end{equation}
		Par l'inégalité triangulaire on a $|c_\gamma| \leq |c_\gamma + \delta_\gamma| + |\delta_\gamma|$ quel que soit $\gamma$.
		On a ainsi 
		\begin{equation}
			|c_\gamma + \delta_\gamma| - |c_\gamma| \geq -|\delta_\gamma|
		\end{equation}
		et en insérant cette inégalité dans la somme on obtient
		\begin{equation}
			||x + \delta||_1 - ||x||_1 \geq \sum_{\gamma \in \Gamma^c} |\delta_\gamma| - \sum_{\gamma \in \Gamma} |\delta_\gamma|,
		\end{equation}
		ainsi une condition suffisante pour obtenir l'unicité est que pour $\delta \in \mathcal{N}$ non nul on ait
		\begin{equation}\label{eq:ineqdelta}
			\sum_{\gamma \in \Gamma} |\delta_\gamma| < \sum_{\gamma \in \Gamma^c} |\delta_\gamma|. 
		\end{equation}
		Avec des mots cela revient à dire que si $\delta$ est dans $\mathcal{N}$ et non nul, alors $\delta$ a plus de poids hors du support de $x$ que sur le support de $x$.
		En ajoutant le terme de gauche de l'inégalité précédente des deux côtés on obtient
		\begin{equation}
			\sum_{\gamma \in \Gamma} |\delta_\gamma| < \frac{1}{2} \left(\sum_{t \in T} |\delta_{T, t}| + \sum_{\omega \in W} |\delta_{W, \omega}|\right) = \frac{||\delta_T||_1 + ||\delta_W||_1}{2}.
		\end{equation}
		Donc l'inégalité précédente est aussi une condition suffisante pour que \ref{eq:ineqdelta3} soit vérifiée et on peut réécrire cette inégalité sous la forme
		\begin{equation}\label{eq:ineqdelta4}
			\frac{\sum_{t \in \Gamma_T} |\delta_{T,t}| + \sum_{\omega \in \Gamma_W} |\delta_{W,\omega}|  }{||\delta_T||_1 + ||\delta_W||_1 } < \frac{1}{2}.
		\end{equation}
		On veut que l'inégalité soit vérifiée pour n'importe quel delta, donc en vérifiant la condition sur le suprémum des $\delta$ dans le noyau de $F_\Phi$ le lemme sera vrai.
		C'est exactement la condition \ref{eq:ineqmu} du lemme
		\begin{equation}
			\mu(\Gamma_T, \Gamma_W) := \sup_{\delta \in \mathcal{N}} \frac{\sum_{t \in \Gamma_T} |\delta_{T,t}| + \sum_{\omega \in \Gamma_W} |\delta_{W,\omega}|  }{||\delta_T||_1 + ||\delta_W||_1 } < \frac{1}{2}.
		\end{equation}
		Le lemme \ref{th:muP1} est donc bien démontré.
		
		
		On peut au passage remarquer qu'on peut utiliser la structure du noyau de $F_\Phi$ de la façon suivante afin d'obtenir une écriture équivalente de \ref{eq:ineqmu} mais qui utilise le fait qu'un élément du noyau de $F_\Phi$ est entièrement déterminé par ses coefficients dans l'une des deux bases.
		On avait vu avec \ref{eq:structN} que les éléments $\delta$ de $\mathcal{N}$ sont de la forme $(\delta_T, -F_W^t F_T \delta_T) =: (\delta_T, -\widehat{\delta_T})$, donc \ref{eq:ineqdelta4} devient
		\begin{equation}
			\frac{\sum_{t\in \Gamma_T} |\delta_{T, t}| + \sum_{\omega \in \Gamma_W} |\widehat{\delta_T}_\omega|}{||\delta_T||_1 + ||\widehat{\delta_T}||_1} < \frac{1}{2}.
		\end{equation}
	\end{proof}	
	On peut maintenant passer à la preuve du lemme \ref{th:muFD}
	\begin{proof}
		\begin{equation}\label{eq:ineqmu}
			\mu(\Gamma_T, \Gamma_W) \leq \frac{\sum_{t \in \Gamma_T} |\delta_{T, t}| + \sum_{\omega \in \Gamma_W} |\widehat{\delta_{T}}_\omega|}{||\delta_T||_1 + ||\delta_W||_1}. 
		\end{equation}
		Maintenant majorons le numérateur avec 
		\begin{equation}\label{eq:ineqnum}
			\sum_{\omega \in \Gamma_W} |\widehat{\delta_T}_\omega| = ||R_{\Gamma_W} F_W^t F_T \delta_T||_1 \leq ||R_{\Gamma_W} F_W^t F_T ||_1 ||\delta_T||_1 
		\end{equation}
		où $||A||_1 = \sup_i ||c_i||_1$ avec $c_i$ les colonnes de la matrice, et $R_{\Gamma_W}$ est la matrice de projection dans l'espace engendré par les vecteurs indexés par $\Gamma_W$.
		Donc $R_{\Gamma_W} F_W^t F_T$ est une matrice à $|\Gamma_W|$ lignes et $N$ colonnes, la norme $\ell_1$ de chaque colonne est égale à $\frac{|\Gamma_W|}{\sqrt{N}}$, ainsi, on a\footnote{C'est ici que le choix de la paire de bases a une importance, la matrice $F_W^t F_T$ contient tous les produits scalaires des vecteurs de $W$ et de $T$, dans le dictionnaire de Fourier-Dirac, chacun des coefficients vaut $1/\sqrt{N}$} :
		\begin{equation}\label{eq:ineqdelta1}
			||R_{\Gamma_W} F_W^t F_T||_1 = \frac{|\Gamma_W|}{\sqrt{N}}.
		\end{equation}	
			Maintenant appliquons la même chose à $\delta_T = -R_{\Gamma_T}F_T^tF_W \delta_W$:
			\begin{equation}
				\sum_{t \in \Gamma_T} |\delta_{T,t}| = ||R_{\Gamma_T} F_T^t F_W \delta_W||_1 \leq ||R_{\Gamma_T} F_T^t F_W ||_1 ||\delta_T||_1 
			\end{equation}
		ainsi que
		\begin{equation}
			||R_{\Gamma_T} F_T^t F_W||_1 = \frac{|\Gamma_T|}{\sqrt{N}}.
		\end{equation}
		On peut maintenant rassembler les résultats:
		\begin{equation}
			\sum_{t \in \Gamma_T} |\delta_{T, t}| + \sum_{\omega \in \Gamma_W} |\widehat{\delta_T}_\omega| 
			\leq ||\delta_{W}||_1 \frac{|\Gamma_T|}{\sqrt{N}} + ||\delta_{T}||_1 \frac{|\Gamma_W|}{\sqrt{N}}. 
		\end{equation}
			On utilise maintenant les hypothèses $|\Gamma_T| < \sqrt{N}/2$ et $|\Gamma_W| < \sqrt{N}/2$, on obtient ainsi :
		\begin{equation}
			\sum_{t \in \Gamma_T} |\delta_{T, t}| + \sum_{\omega \in \Gamma_W} |\widehat{\delta_T}_\omega| 
			< \frac{||\delta_{T}||_1 + ||\delta_{W}||_1 }{2}.
		\end{equation}
			Il nous reste maintenant à appliquer la majoration que l'on vient de trouver à \ref{eq:ineqmu} et on obtient
		\begin{equation}
			\mu(\Gamma_T, \Gamma_W) < \frac{1}{2} \frac{||\delta_T||_1 + ||\delta_W||_1}{||\delta_T||_1 + ||\delta_W||_1} = \frac{1}{2} 
		\end{equation}
			Ce qui conclut la preuve du lemme \ref{th:muFD} et donc du théorème \ref{th:DiracFourier}.
	\end{proof}


\subsection{Définition de (P1)}
\subsection{Propriétés du minimiseur}

\section{Lien géométrique entre (P0) et (P1)}
\begin{figure}[h]
	\floatbox[{\capbeside\thisfloatsetup{capbesideposition={left,top},capbesidewidth=4cm}}]{figure}[\FBwidth]
	{\caption{Minimisation de $y=Fx$ pour la norme $\ell^1$. La solution de \ref{P1} est généralement celle de \ref{P0} sauf si les solutions sont parallèles à l'une des faces de la boule de $\ell^1$.}}
	{\includegraphics{Figs/l1vs1}}
	
	\floatbox[{\capbeside\thisfloatsetup{capbesideposition={right,bottom},capbesidewidth=4cm}}]{figure}[\FBwidth]
	{\caption{Minimisation de $y=Fx$ pour la norme $\ell^2$. La solution de \ref{P2} n'est généralement pas celle de \ref{P0} sauf si les solutions sont parallèles à l'un des axes.}}
	{\includegraphics{Figs/l1vs2}}
	
	\end{figure}

\subsection{Boules unité en grande dimension}
\subsection{Unicité de la solution de (P0) et (P1)}




\chapter{Compressed sensing et approche aléatoire}
\section{Introduction au Compressed Sensing}
Le chapitre précédent correspond à des résultats publiés entre 1998 et 2003, grâce à eux une classe de problème en apparence impossibles à résoudre (le problème \ref{P0}) devenaient finalement accessibles sous des hypothèses de parcimonie. 
Cependant, les résultats prouvés donnent des informations quantitatives sur les plus petits signaux qui feront que la résolution de \ref{P0} en résolvant \ref{P1} ne fonctionne pas.
Mais en fait, il y a très peu de tels contre-exemples et dans la pratique il était déjà observé que la résolution de \ref{P0} par \ref{P1} fonctionnait pour des signaux moins parcimonieux que ceux étudiés.
C'est ainsi que plutôt que de chercher des résultats toujours vrais comme dans le chapitre précédent la recherche de nouveaux théorèmes s'est tournée vers des questions, demandant une approche probabiliste, du type : Quel $s$ peut on choisir pour reconstruire \emph{presque} tous les signaux $s$-parcimonieux en résolvant \ref{P1} ?
\newline 
La deuxième question qui s'est posée, plus subtile, est la suivante, supposons que l'on sache que le signal est parcimonieux dans une base quelconque : Quel est le nombre minimal $m$ de mesures que l'on peut faire pour être sûr\footnote{On aurait bien sûr pu mettre \emph{presque sûr} au lieu de \emph{sûr}} de reconstruire tous les signaux $s$-parcimonieux de cette base quelconque ?
\newline
A cette deuxième question, la réponse à apporter est plus claire qu'à la première question.
On cherche à reconstruire un signal à $N$ coefficients qui n'a que $s$ coefficients non nuls, cependant on veut pouvoir le reconstruire avec $m<N$ mesure, ce signal est arbitraire donc on ne sait pas où sont les $s$-coefficients non nuls, il faut donc que la mesure se fasse sur un grand nombre de coefficients à la fois.
Donc, dans la base dans laquelle on mesure le signal, il ne doit pas être parcimonieux.
Il nous faut donc une base qui vérifie un principe d'incertitude avec la base dans laquelle le signal est parcimonieux.
C'est le contenu de la condition \textbf{Uniform Uncertainty Principle} (\textbf{UUP}).
\newline
A partir de là il semble qu'il y ait une possibiité pour reconstruire le signal car n'importe quel coefficient peut être mesuré avec un nombre assez faible de mesures. 
Cependant une difficulté apparait directement, chaque mesure va mesurer plusieurs coefficients à la fois et il  nous faut donc suffisament de mesures pour être certain de distinguer chaque coefficient non nul.
Il nous faudra donc des garanties sur la famille utilisée pour la mesure pour qu'elle puisse nous permettre de distinguer les coefficients non nuls des coefficients nuls des signaux $s$-parcimonieux.
C'est le contenu de la condition \textbf{Exact reconstruction principle} (\textbf{ERP}).
\newline
La deuxième difficulté est que l'on ne connait pas à l'avance la base dans laquelle le signal est parcimonieux, on sait aussi que l'on fera un nombre limité $m$ de mesures, donc le principe d'incertitude devra être vérifié entre la base des mesures et n'importe quelle restriction à $m$ coordonées de la base inconnue.
Cela revient donc à la projection dans un espace à $m$ dimensions arbitraires.
C'est ainsi qu'ont été publiés par Emmanuel Candes, Justin Romberg et Terence Tao \cite{CR}, \cite{CT}, \cite{CRT}, et indépendamment par David Donoho \cite{DonohoCS}, les articles fondateurs du Compressed Sensing, dont les théorèmes montrent qu'un tel raisonnement peut-être démontré. 
\newline
On suivra dans un premier temps l'article de Emmanuel Candes et Terence Tao pour prouver un théorème de compressed sensing, ensuite, on verra sans preuve des théorèmes de Emmanuel Candes et Justin Romberg dont l'énoncé sera plus simple à comprendre et qui mettra en valeur l'incohérence, quantité clé du chapitre précédent.
Dans ce chapitre, on verra \textbf{UUP} et \textbf{ERP} comme des axiomes, on ne démontrera pas que de telles familles existent, de la même façon que dans le chapitre 2 nous n'avons pas démontré que les ondelettes de Daubechies existent.
On verra cependant des exemples de familles qui vérifient ces conditions et des preuves de ces résultats, avec des bornes plus ou moins optimales peuvent être trouvées dns les articles précedemment cités ou bien dans \cite{foucartbook}. 
\section{Axiomatisation, \textbf{UUP} et \textbf{RIP}}
\subsection{Notations}
Tout d'abord réintroduisons certaines choses déjà vues dans ce mémoire et les notations qui seront utilisées.
Dans ce chapitre on considère $\mathcal{F} \subset \mathbb{R}^N$ un classe de signaux.
On cherche à pouvoir reconstruire chaque élément $f\in \mathcal{F}$ avec une précision $\varepsilon$ en utilisant une famille de vecteurs $\Psi=(\psi_k)_{k \in \Omega}$.
\newline
C'est à dire, on considère une application d'analyse,
\begin{align}
	A : 	&\mathcal{F} \longrightarrow \mathbb{R}^{|\Omega|}\\
	(f_k)_{k = 0, \cdots, N} = &f \longmapsto (\langle f, \psi_k \rangle )_{k\in \Omega}
\end{align}
qui a chaque signal associe la projection sur chaque élément de $\Psi$.
Pour l'instant cela est similaire à la situation dans l'étude des frames, la différence essentielle concerne $\Omega$, dans la situation des frames $\Omega$ est fixé et généralement $|\Omega| \geq N$.
Comme discuté, on s'intéresse ici à la projection dans un sous espace arbitraire, c'est ainsi le rôle que va jouer $\Omega$ en étant une variable aléatoire, et comme on souhaite faire un nombre minimal de mesures, on va chercher à avoir que le nombre de mesures moyen $K = \mathbb{E}(|\Omega|)$ sera inférieur à $N$.
La difficulté vient alors dans la construction de l'application de synthèse associée, pour l'instant définissons la pour fixer les notations :
\begin{align}
	S: \mathbb{R}^{|\Omega|} &\longrightarrow \mathbb{R}^N \\
	(y_k)_\Omega &\longmapsto (y_k)^\# = (f_k ^\#)_{k = 0, \cdots, N} =: f^\#
\end{align}
et on cherche à obtenir une $\varepsilon$-reconstruction :
\begin{equation}
	|| f- f^\#||_2 \leq \varepsilon \quad, \forall f \in \mathcal{F}. 
\end{equation}
Le problème est donc de choisir une famille $(\psi_k)_{k \in \Omega}$ pour qu'il soit possible d'obtenir la dernière inégalité.
Il est clair que le problème est mal posé si on prend $\mathcal{F} = \mathbb{R}^N$, on verra plus bas sur quelles familles de signaux on pourra démontrer des résultats.
\newline
Notons $F_\Omega$ la matrice, aléatoire, avec $|\Omega|$ lignes et $N$ colonnes qui représente $A$.
Pour aider à se fixer les idées sur le type de matrice que l'on considérera, $F_\Omega$ peut être la restriction de $|\Omega|$ lignes d'une matrice de Fourier, ou bien une matrice dont les coefficientss suivent une loi normale de moyenne nulle et de variance égale à 1.
\newline
Notons maintenant $R_\Omega$ la matrice de restriction aux indices de $\Omega$.
\begin{align}
	R_\Omega : \ell^2([0, N]) &\longrightarrow \ell^2(\Omega) \\
		(g_k)_{0\leq k\leq N} &\longmapsto (g_k)_{k\in \Omega}
\end{align}
et l'inclusion prolongée par des zéros
\begin{align}
	R_T^* : \ell^2(T) &\longrightarrow \ell^2([0,N])\\
		(g_k)_{k \in T} &\longmapsto (g_k)_{k\in T} \oplus (0)_{k\in T^c}.
\end{align}
On considèrera aussi par la suite la matrice aléatoire $F_{\Omega T}$ en conservant que les $|T|$ colonnes indexées par $T$ de la matrice $F_{\Omega}$, c'est à dire :
\begin{align}
	F_{\Omega T} = F_{\Omega} R_T^* : \ell^2(T) &\longrightarrow \ell^2(\Omega)\\
		(g_k)_T &\longmapsto F_{\Omega}( (g_k)_T \oplus (0)_T^c ).
\end{align}
On remarque aussi que $F_{\Omega T}^* F_{\Omega T} : \ell^2(T) \rightarrow \ell^2(T)$ est symétrique et que l'on peut la diagonaliser sous la forme $U \Lambda U^*$ où $\Lambda = (\lambda_1 \geq \cdots \geq \lambda_{|T|})$ sont les valeurs propres de $F_{\Omega T}^* F_{\Omega T}$ qui sont aussi appelées valeurs singulières de $F_{\Omega T}$.
\newline
Dans ce chapitre on considérera trois modèles de matrices aléatoires, on a déjà vu le modèle aléatoire de Fourier en échantillonant les lignes indéxées par $\Omega$ de la matrice de Fourier.
On verra aussi la matrice gaussienne dont les coefficients suivent une loi normale centrée réduite normalisée et le modèle de Bernoulli où les coefficients sont égaux à $+1$ ou $-1$.
\newline
Deux modèles d'échantillonage sont utilisés dans la suite, si on fixe $|\Omega|$ et que $\Omega$ peut être n'importe quel ensemble de taille $\Omega$ on peut parler de modèle uniforme.
Si $\Omega$ est obtenu en faisant un échantillonage qui suit une loi de Bernoulli, on peut parler de modèle de Bernoulli.
\subsection{Définition de \textbf{UUP}} 

On peut alors définir le \emph{principe uniforme d'incertitude} (\textbf{Uniform Uncertainty Principle}), 
\begin{definition}
	On dit que $F_\Omega$ vérifie $\lambda$-\textbf{UUP} si il existe $\rho$ tel que avec probabilité $1 - \mathcal{O}(N^{-\rho / \alpha})$ on ait:
	\newline
	$\forall f \subset \mathbb{R}^N$ signal tel que 
	\begin{equation}\label{eq:UUP1}
		|supp(f)| \leq \alpha K /\lambda
	\end{equation}
	on ait l'inégalité
	\begin{equation}\label{eq:UUP2}
		\frac{1}{2}\frac{K}{N} ||f||_2^2\leq ||F_\Omega f||_2^2 \leq \frac{3}{2} \frac{K}{N} ||f||_2^2.
	\end{equation}
\end{definition}
Quelques mots s'imposent sur cette définition. 
Tout d'abord, comme mentionné en début de chapitre, on ne cherche pas à avoir un résultat toujours vrai, on veut seulement qu'il soit presque toujours vrai, d'où le fait que le résultat soit vrai avec une probabilité $1-\mathcal{O}(N^{-\rho/\alpha})$.
Remarquons aussi que le résultat est exprimé en terme de $\mathcal{O}$ et on en déduit donc que le résultat peut devenir vrai avec probabilité égale à 1 si on peut choisir $N$ arbitrairement grand.
\newline
Ensuite, on voit que les signaux sur lesquels le résultat est vrai sont ceux dont le support est inférieur $\alpha K /\lambda$, pour couvrir un maximum de signaux, on cherchera donc à avoir $F_\Omega$ qui vérifie $\lambda$-\textbf{UUP} avec $\lambda$ aussi petit que possible.
\newline
Concernant \ref{eq:UUP2} avec l'étude des frames et de leurs liens avec les bases orthonormales, il devrait être clair que \ref{eq:UUP2} est entre une condition de frame et de base orthonormale.

On aurait aussi pu définir le principe uniforme d'incertitude à l'aide des valeurs propres :
\begin{proposition}\label{th:lambdauup}
	$F_\Omega$ vérifie $\lambda$-\textbf{UUP} si et seulement si 
	\newline
	avec probabilité au moins $1-\mathcal{O}(N^{-\rho / \alpha})$ pour un certain $\rho>0$ on a
	$\forall T \subset [0, N]$ qui vérifie $|T| \leq \alpha \frac{K}{\lambda}$ alors les valeurs propres de $F_{\Omega T}$ vérifient
	\begin{equation*}
		\frac{1}{2} \frac{K}{N} \leq \lambda_{min}(\Lambda) \leq \lambda_{max}(\Lambda) \leq \frac{3}{2}\frac{K}{N}.
	\end{equation*}
	où $\lambda_{max}$ (resp. $\lambda_{min}$) est la valeur propre maximale (resp. minimale) de $F_{\Omega T} F_{\Omega T}^*$. 
\end{proposition}
\begin{preuve}
	Pour montrer que la définition implique la proposition, on prend un vecteur propre $f$ de $F_{\Omega T}F_{\Omega T}^*$ de valeur propre maximale (resp. minimale), d'où:
	\begin{equation}
		||F_{\Omega T} f||_2^2 = f^t F_{\Omega T}^*F_{\Omega T} f = \lambda_{max} ||f||_2^2 
	\end{equation}
	et le dernier terme est plus petit (resp. plus grand pour $\lambda_{min}$) que $\frac{3K}{2N}||f||_2^2$  (resp. $\frac{K}{2N}||f||_2^2$)par hypothèse.
	\newline
	Dans l'autre sens, on prend un vecteur $f$ tel que $T$ est le support de $f$, alors $F_{\Omega T} f = F_\Omega f$ et donc:
	\begin{equation}
		\lambda_{min} ||f||_2^2 \leq ||F_\Omega f||_2^2 \leq \lambda_{max} ||f||_2^2
	\end{equation}
	et il suffit de remplacer $\lambda_{min}$ et $\lambda_{max}$ par leurs valeurs dans la proposition.
\end{preuve}
\begin{remarque}
	Pour expliciter le fait que cela définit bien un principe d'incertitude, considérons $F_\Omega$ comme étant la transformée de fourier discrète partielle, et un signal concentré en temps ($|supp(f)| \leq \alpha \frac{K}{\lambda}$), alors on a 
	\begin{equation}
		||F_\Omega f||_{\ell^2} = ||\hat{f}||_{\ell^2(\Omega)} \leq \sqrt{\frac{3 K}{2N}}||f||_{\ell^2}
	\end{equation}
	en appliquant le principe d'incertitude.
	On déduit donc que 
	\begin{equation}
		\frac{ ||\hat{f}||_{\ell^2(\Omega)}}{||\hat{f}||_{\ell^2}} \longrightarrow 0
	\end{equation}
	si $K=o(N)$, c'est à dire que si $f$ est à support compact, il est nécessaire d'avoir un nombre de mesures $K$ qui est au moins de l'ordre de $f$.
	Donc $f$ ne peut pas être localisé à la fois en temps et en fréquence, ce qui justifie l'appélation "principe d'incertitude". 
\end{remarque}
\begin{remarque}
Justifions maintenant le fait que c'est un principe uniforme. 
	Une version non uniforme (et donc plus faible) serait que pour chaque $f$ vérifiant \ref{eq:UUP1}, alors avec probabilité au moins $1 -\mathcal{O}(N^{-\rho / \alpha})$  \ref{eq:UUP2} est vérifié. Mais il y a beaucoup de choix possibles de $f$ vérifiant \ref{eq:UUP1}, et parmi ceux-ci il peut y avoir un grand nombre de $f$ ayant la propriété rare de ne pas vérifier $\ref{eq:UUP2}$, et alors l'union de ces événements n'a pas nécessairement une faible probabilité de se produire.
	\newline
	Ainsi, le principe est uniforme car la propriété \textbf{UUP} est telle que l'on a une probabilité au moins $1- \mathcal{O}(N^{-\rho / \alpha})$ que \ref{eq:UUP2} soit vrai pour tous les $f$ possibles vérifiant \ref{eq:UUP1}. Ce qui justifie l'appélation uniforme.
\end{remarque}
\begin{remarque}
	Remarquons que l'on peut réécrire \ref{eq:UUP2} peut se réécrire
	\begin{equation*}
		(1-\delta_K)||f||_2^2 \leq ||F_\Omega f||_2^2 \leq ||f||_2^2(1 + \delta_K)
	\end{equation*}
	avec $\delta = 1 - \frac{K}{2N}$ ce qui rappelle la définition d'un frame avec des bornes $m = M = \frac{1}{2}$ dans le meilleur des cas.
	Cela justifie que certaines fois le principe uniforme d'incertitude est aussi appelé propriété d'isométrie restreinte (\textbf{RIP}) (Restricted Isometry Property).
\end{remarque}
\begin{proposition}\footnote{Pour certains résultats concernant ERP et UUP : \url{https://www.math.ucla.edu/~tao/preprints/sparse.html}}
	\newline 
	\begin{itemize}
		\item Les ensembles Gaussiens et binaires vérifient $\log(N)-\textbf{UUP}$
		\item L'ensemble de Fourier vérifie $\log(N)-\textbf{UUP}$.
	\end{itemize}
\end{proposition}
On peut trouver des démonstration dans \cite{CT} et \cite{foucartbook}.
\subsection{Définition de \textbf{ERP}}
Un autre principe que l'on va utiliser qui nous permettra de nous assurer que l'approximation $f^\#$ obtenue est proche de $f$ pour la norme $\ell^1$ est le principe de reconstruction exacte (\textbf{ERP} - Exact Reconstruction Principle).
\begin{definition}
	$F_\Omega$ vérifie $\lambda$-\textbf{ERP} si
	\begin{itemize}
		\item $\forall T \subset [0, N]$ vérifiant $|T| \leq \alpha \frac{K}{\lambda}$
		\item $\forall \sigma \in \{\pm 1\}^T$
	\end{itemize}
	il existe avec probabilité au moins $1-\mathcal{O}(N^{-\rho / \alpha})$ pour un certain $\rho>0$, un vecteur $P\in \mathbb{R}^N$ tel que
	\begin{enumerate}
		\item $P(t) = \sigma(t), \forall t \in T$
		\item $P$ est une combinaison linéaire des lignes de $F_\Omega$ \footnote{ C'est équivalent à $P$ appartient au \textit{rowspace} de $F_\Omega$, ce qui est équivalent à : $\exists Q$ tel que $P = F_\Omega ^* Q$ donc $Q$ avec $|\Omega|$ coordonnées.}
		\item $P(t) < \frac{1}{2}, \forall t \in T^c$\footnote{Le $\frac{1}{2}$ n'a pas vraiment d'importance, n'importe quelle constante $0 < \beta < 1$ permet d'obtenir les mêmes résultats} 
	\end{enumerate}
\end{definition}
Comme discuté dans l'introduction de ce chapitre, cela revient à pouvoir reconstruire (et surtout distinguer) n'importe quelle suite de signes supportée sur $T$
\begin{proposition} 
	\begin{itemize}
		\item Les ensembles Gaussiens et binaires vérifient $\log N$-\textbf{ERP}
		\item L'ensemble de Fourier vérifie $\log N$-\textbf{ERP}.
	\end{itemize}
\end{proposition}

\section{Théorème de Candes-Tao}
Avec les notations et les définitions ci-dessus on peut maintenant énoncer puis prouver le théorème de Emmanuel Candes et Terence Tao.
\begin{theoreme}
	Soit $F_\Omega$ qui vérifie $\lambda_1$-\textbf{ERP} et $\lambda_2$-\textbf{UUP}.
	On pose $\lambda = \max(\lambda_1, \lambda_2)$, soit $K\geq \lambda$.
	\newline
	Soit $f$ un signal dans $\mathbb{R}^N$ tel que ses coefficients dans une base de référence décroissent comme\footnote{les coefficient $(|\theta_{(n)}|)$sont triés par ordre décroissant} :
	\begin{equation}\label{eq:ineqtheta}
		|\theta_{(n)}| \leq C n^{-\frac{1}{p}}
	\end{equation}
	pour un certain $C >0$ et $0 < p \leq 1$. 
	\newline
	On pose $r = \frac{1}{p} - \frac{1}{2}$, alors n'importe quel minimiseur de (P1) vérifie :
	\begin{equation}
		||f - f^\#||_2 \leq C_r (\frac{K}{\lambda})^{-r}
	\end{equation}
	avec probabilité au moins $1 - \mathcal{O}(N^{-\frac{\rho}{\alpha}})$, pour certains $\rho$ et $\alpha$.
\end{theoreme}
Tout d'abord on peut préciser la famille de signaux que l'on considère, ce sont ceux dont les coefficients décroissent comme une loi en puissance dans une certaine base. 
C'est une telle décroissance que l'on a par exemple identifié pour les coefficients d'ondelettes d'une fonction lipschitzienne par rapport à l'échelle dans le théorème de Stéphane Jaffard \ref{th:Jaffard}.
On voit donc que le théorème peut s'appliquer de façon générale.
De plus, contrairement aux résultats du chapitre 3, on ne demande pas à ce que le signal soit exactement parcimonieux pour avoir une bonne reconstruction $\ell^2$.
L'avantage de ce théorème est donc qu'en résolvant \ref{P1}, même si une solution vraiment parcimonieuse n'existe pas, on a quand même une reconstruction du signal pour la norme euclidienne.
\newline
Passons à la preuve du théorème, on discutera ensuite de certaines conséquences ainsi que de résultats analogues.
\begin{proof}
	Soit $F_\Omega$ comme dans le théorème.
	Soit $f \in \mathbb{R}^N$ dont les coefficients vérifient \ref{eq:ineqtheta}, on note alors $f^\#$ une solution $y=F_\Omega f$.
	L'objectif est de déterminer
	\begin{equation}
		||f - f^\#||_2.
	\end{equation}
	On note $T$ l'ensemble des $T$ plus grandes valeurs de $|f| + |f^\#|$.
	Comme vu dans les preuves du chapitre 3, on a:
	\begin{equation}\label{eq:kerFO}
		F_\Omega (f - f^\#) = 0.
	\end{equation}
	Cependant, les hypothèses que l'on peut utiliser (\textbf{ERP} et \textbf{UUP}) ne peuvent être utilisées que dans sur des ensemble de taille $T$ où $T\leq \alpha K /\lambda$.
	Donc, plutôt que de considérer $f-f^\#$, on va considérer $h = (f-f^\#) 1_T$, où on note $1_T$ la restriction aux indices de $T$.
	Il est alors possible de démontrer en appliquant \textbf{UUP} (voir le lemme \ref{th:tao3}) que l'on peut trouver $g \in \mathbb{R}^N$ qui s'écrit $g = F_\Omega^* V$ pour un certain $V \in \mathbb{R}^\Omega$ et qui vérifie à la fois de prendre les mêmes valeurs que $h$ sur $T$ et
	\begin{equation}\label{eq:ineqgf}
		\sum_{t \in E} |g(t)|^2 \leq C \sum_{t\in T} |f(t) - f^\#(t)|^2
	\end{equation}
	pour n'importe que ensemble $E$ disjoint de $T$ et de taille $|E| = \mathcal{O}(K/\lambda)$.
	Avec des mots, l'image de $g$ par $F_{\Omega}$ vaut 0 sur $T$, et hors de $T$ les valeurs de $g$ sont majorées par la distance euclidienne de $f$ et $f^\#$.
	On peut alors utiliser le fait que $g$ s'écrive $g=F_\Omega^* V$:
	\begin{equation}
		\langle f-f^\#, g \rangle = \langle f-f^\#, F_\Omega^* V\rangle = \langle F_\Omega(f-f^\#), V \langle = 0
	\end{equation}
	d'après \ref{eq:kerFO}.
	Ainsi, on a 
	\begin{equation}
		\sum_{t=0,\cdots, N} (f - f^\#)(t) g(t) = 0
	\end{equation}
	que l'on peut réécrire
	\begin{equation}\label{eq:ineqfg}
		\sum_{t\in T} (f - f^\#)(t) g(t) = ||f-f^\#||_{\ell^2(T)}^2 = - \sum_{t\in T^c} (f - f^\#)(t) g(t).
	\end{equation}
	Par ailleurs, on a toujours:
	\begin{equation}
		||f-f^\#||_{\ell^1(T^c)} \leq ||f||_{\ell^1(T^c)} + ||f||_{\ell^1(T^c)}
	\end{equation}
	D'après un lemme qui utilise \textbf{ERP} que l'on peut trouver en annexe \ref{th:tao1} 
	\begin{equation}\label{eq:ineqtc}
		||f + f^\#||_{\ell^1(T^c)} \leq C |T|^{1-\frac{1}{p}}.
	\end{equation}
	et d'après un autre lemme \ref{th:tao2} démontré en annexe on a 
	\begin{equation}\label{eq:ineqinf}
		||f-f^\#||_{\ell^\infty (T^c)} \leq C |T|^{-1/p}
	\end{equation}
	Avec ces deux inégalités on peut appliquer l'inégalité de Hölder pour obtenir :
	\begin{equation}
		||f-f^\#||_{\ell^2(T^c)} = \sqrt{|||f-f^\#|^2||_{\ell^1(T^c)}} \leq \sqrt{||f + f^\#||_{\ell^1(T^c)} 	||f-f^\#||_{\ell^\infty (T^c)}} \leq C |T|^{\frac{1}{2} - \frac{1}{2p}}.
	\end{equation}
	Pour prouver le théorème, il reste donc à montrer qu'une telle majoration est encore vraie sur $T$.
	Pour obtenir cela on va réordonner les indices, tout d'abord on énumère $T^c=(n_1, \cdots, n_{N-|T|})$ tels que les coefficients correspondants dans $|f-f^\#|$ soient classés par ordre décroissant.
	Maintenant on regroupe ces coefficients en blocs de taille $|T|$, à part peut-être pour le dernier qui peut être plus petit.
	On les notes $B_J =\{n_j, J|T| < j \leq (J+1)|T|\}$ ces blocs pour $J=0,\cdots, \lfloor N/|T| \rfloor$.
	On a alors, d'après Cauchy-Schwarz, à $J$ fixé:
	\begin{equation}
		\sum_{j\in B_J} (f-f^\#)(n_j) g(n_j) \leq ||f-f^\#||_{\ell^2(B_J)} ||g||_{\ell^2(B_J)}.
	\end{equation}
	Or d'après \ref{eq:ineqgf}, $||g||_{\ell^2(B_J)} \leq C||f-f^\#||_{\ell^2(T)}$, donc l'inégalité précédente devient:
	\begin{equation}
		\sum_{j\in B_J} (f-f^\#)(n_j) g(n_j) \leq C ||f-f^\#||_{\ell^2(T)} ||f-f^\#||_{\ell^2(B_J)} \leq C ||f-f^\#||_{\ell^2(T)} I_J
	\end{equation}
	en notant
	\begin{equation}
		I_J := ||f-f^\#||_{\ell^2(B_J)} = \sqrt{\sum_{j=J|T| + 1}^{(J+1)|T|} |(f-f^\#)(n_j)|^2}.
	\end{equation}
	Comme les coefficients sont par ordre décroissants, on va pouvoir obtenir des inégalités successives entre les $I_J$.
	Tout d'abord, pour $J=0$, on  a clairement :
	\begin{equation}
		I_0 \leq \sqrt{|T|}||f-f^\#||_{\ell^\infty(B_J)} \leq \sqrt{|T|} |(f-f^\#)(n_0)|.
	\end{equation}
	D'après \ref{eq:ineqinf} on a donc:
	\begin{equation}
		I_0 \leq C|T|^{\frac{1}{2} - \frac{1}{p}} = C|T|^{-r}.
	\end{equation}
	Pour $J\geq 1$ on a ainsi:
	\begin{equation}
		I_J =||f-f^\#||_{\ell^{2}(B_{J})} \leq |T|^\frac{1}{2}|f-f^\#|(n_{J|T|+1}) \leq |T|^\frac{1}{2} |T|^{-1} ||f-f^\#||_{\ell^1(B_{J-1})}.
	\end{equation}
	Afin de conclure il nous faut donc évaluer la somme sur $J$ des $I_J$,
	\begin{equation}\label{eq:ineqij}
		\sum_{J\geq 0} I_J \leq I_0 + \sum_{J\geq 1} I_J  \leq C|T|^{-r} +\frac{1}{|T|^{\frac{1}{2}}} \sum_{J\geq 0} I_J.
	\end{equation}
	On déduit de la précédente inégalité:
	\begin{equation}
		\sum_{J\geq 0} I_J \leq C|T|^{-r} + |T|^{-\frac{1}{2}} ||f-f^\#||_{\ell^1(T^c)}
	\end{equation}
	et d'après \ref{eq:ineqtc} on a ainsi:
	\begin{equation}
		\sum_{J\geq 0} I_J \leq C|T|^{-r} + C|T|^{\frac{1}{2} - \frac{1}{p}} = 2C|T|^{-r}.
	\end{equation}
	Les $B_J$ formant une partition de $\{0, \cdots, N\}\backslash T$ on a, d'après \ref{eq:ineqbj}, puis \ref{eq:ineqij}:
	\begin{align}
		\sum_{t \in T^c} (f-f^\#)(t)g(t)&\leq \sum_J C ||f-f^\#||_{\ell^2(T)}||f-f^\#||_{\ell^2(B_J)} \\
						&\leq C||f-f^\#||_{\ell^2(T)}\sum_J I_J \leq 2C||f-f^\#||_{\ell^2(T)}|T|^{-r}
	\end{align}
	Et donc finalement avec \ref{eq:ineqfg} on a:
	\begin{equation}
		||f-f^\#||_{\ell^2(T)}^2\leq 2C ||f-f^\#||_{\ell^2(T)}|T|^{-r}
	\end{equation}
	que l'on peut réécrire :
	\begin{equation}
		||f-f^\#||_{\ell^2(T)}\leq 2C|T|^{-r}.
	\end{equation}
	On a donc bien démontré le théorème.	
\end{proof}
Dans la preuve, deux résultats intermédiaires ont été admis. Le premier utilise \emph{UUP} et est une conséquence d'un théorème d'extension, avec des mots, on a construit un vecteur dans l'espace engendré par $F_\Omega^*$ qui coincide avec un autre vecteur arbitraire sur un support fini et qui vérifie une forme de stabilité $\ell^2$. On pourra trouver une démonstration de ce résultat en annexe \ref{th:tao3}.
\newline
Le second résultat utilise \emph{ERP} et permet de démontrer un autre résultat sur la concentration en dehors d'un support fixé, mais pour la norme $\ell^1$. On pourra en trouver un résultat formel en annexe également \ref{th:tao1}.
\newline
Cependant il aurait été possible d'utiliser d'autres conditions, la condition \textbf{UUP} n'étant pas très difficile à vérifier contrairement à la condition \textbf{ERP} qui doit être vérifiée pour n'importe quelle suite de signes. 
Il est ainsi possible d'affaiblir cette condition en \textbf{Weak Exact Reconstruction Principle} (\textbf{WERP}) dans laquelle plutôt que d'avoir \textbf{ERP} pour n'importe quelle suite de signe, on souhaite pour presque n'importe quelle suite de signes \textbf{ERP}. 
En fait, Terence Tao et Emmanuel Candes prouvent dans \cite{CT} que \textbf{WERP} et \textbf{UUP} impliquent \textbf{ERP} pour presque n'importe quel signal.
\section{Théorème de Donoho}
Etudions maintenant un autre théorème fondamental du Compressed Sensing, et en fait très similaire à celui de Emmanuel Candes et Terence Tao, avec le théorème de David Donoho.
Ce théorème est présenté ici car il est plus facile à exprimer que le théorème précédent, il n'y a pas besoin d'autant de notations et définitions, aussi, on va voir que ce résultat est assez proche de l'esprit des résultats du chapitre 3.
Cependant on ne fera pas la preuve de ce résultat pour plusieurs raisons, premièrement, après le théorème précédent et le chapitre 3 il devrait être clair qu'un tel résultat soit démontrables, deuxièmement, les techniques de preuves utilisées par David Donoho sont assez élaborées et reposent sur l'usage d'inégalités sur des lois de probabilités, finalement, il aurait été difficile d'introduire de façon raisonnable ces résultats sans ajouter trop de pages supplémentaires à ce mémoire (qui est peut-être déjà trop long).
\newline
Cela étant dit, le résultat de David Donoho n'a besoin que de d'un rappel de définitions utilisée dans les chapitres précédents.
Soit $(\Phi, \Psi)$ une paires de matrices et $(\psi_1, \cdots, \psi_N)$ les colonnes de $\Psi$ et $(\varphi_1, \cdots, \varphi_N)$ les colonnes de $\Phi$.
On note la cohérence entre $\Phi$ et $\Psi$:
\begin{equation}
	M_{\Phi, \Psi} = \sup_{i,j}|\langle \psi_i, \varphi_j \rangle|
\end{equation}
On notera en particulier $M_\Phi = \sup_{i,j} |\varphi_{i,j}|$.

On peut alors énoncer le théorème de David Donoho:
\begin{theoreme}
	Soit $\Phi$ une base orthonormale de $\mathbb{R}^N$. 
	Soit $T \subset \{0, \cdots, N\}$ un sous-ensemble fixé et soit $z \in \{\pm 1\}^T$ une suite de signes tirée uniformément au hasard ($\mathbb{P}(z(t) = 1) = \mathbb{P}(z(t) = -1) = \frac{1}{2}$ pour tout $t\in T$).
	Si le nombre de mesures $m$ vérifie:
	\begin{equation}
		m\geq C_0 |T| N M_{\Phi}^2 \log(\frac{N}{\delta})
	\end{equation}
	et
	\begin{equation}
		m\geq C_1 \log^2(\frac{N}{\delta})
	\end{equation}
	pour des constantes fixées $C_0$ et $C_1$.
	Alors, avec probabilité au moins $1-\delta$, on peut reconstruire n'importe quel signal $x_0$ supporté sur $T$ et ayant la même suite de signes que $z$ à partir de $m$ mesures
	\begin{equation}
		y = F_{\Phi \Omega} x_0
	\end{equation}
	en résolvant \ref{P1}.
\end{theoreme}
On ne démontre pas ce théorème mais discutons de son lien avec le théorème de Emmanuel Candes et Terence Tao, dans celui ci on choisit d'abord une suite de signes, qui n'est pas arbitraire et qui a été obtenue par un processus aléatoire et on peut ensuite conclure sur la reconstruction, on n'est donc pas dans le cas de l'\textbf{Exact Reconstruction Principle} mais plutôt du \textbf{Weak Exact Reconstruction Principle}.
\newline
Ensuite, ici, le lien entre la cohérence de la matrice utilisée pour mesurer et le nombre de mesures à réaliser est explicite, en fait ce nombre de mesures est choisi car il permet de vérifier (théorème 1.2 \cite{CR}), que les valeurs singulières de $F_{\Phi, \Omega T}$ sont toutes proches de $\frac{m}{N}$, c'est bien le contenu de l'\textbf{Uniform Uncertainty Principle}.
\newline
Aussi, il est possible de relier les $\lambda_1$ et $\lambda_2$ du théorème de Emmanuel Candes et Terence Tao aux deux inégalités du théorème de Donoho. 
On remarquera juste que dans le cas où $F_\Phi$ est la matrice de Fourier, alors $M_\Phi= \frac{1}{\sqrt{N}}$, on utilise donc la seconde inégalité si $|T|$ est petit.
De même pour une matrice dont les coefficients suivent une loi normale centrée de variance égale à 1, alors\footnote{Pour montrer cela il faut montrer que l'espérance de la valeur maximale de $n$ tirages de loi normale centrée réduite est majoré par $C\sqrt{\log(n)}$, la démonstration étant hors du sujet du mémoire n'est pas faite ici, on pourr consulter \cite{foucartbook} pour de nombreux résultats et démonstrations sur les matrices aléatoires dans le cadre du compressed sensing.} $M_\Phi \leq C\frac{1}{N}\sqrt{\log(N)}$, on est donc également dans le cas de la seconde inégalité du théorème, si $|T|$ est petit.
Cependant, on peut revenir au cas de la première inégalité en supposant $C_0> C_1\log(\frac{N}{\delta})$. 
\newline
On a donc le corollaire suivant :
\begin{theoreme}
	Soit $F_{\Omega}$ la restriction de la matrice de Fourier ou d'une matrice gaussienne à un sous ensemble $\Omega \subset \{0, \cdots, N-1\}$ qui vérifie:
	\begin{equation}
		|\Omega| \geq C S \log(N)
	\end{equation}
	pour une certaine constante $C$.
	Soit $f$ un signal $s$-parcimonieux, donc tel qu'il existe $x_0$ avec $s$-composantes vérifiant $f=F_{\Omega}x_0$ alors avec probabilité tendant vers 1 la solution $x$ de \ref{P1} est $x=x_0$.
\end{theoreme}
On peut aussi déduire du théorème de Donoho la version asymptotique des théorèmes \ref{th:DiracFourier}, \ref{th:recovgen} et \ref{th:eladbruc} en considérant que l'on fait toutes les mesures $(|\Omega| =N)$:
\begin{theoreme}\label{th:recovgen}
		Soit $N$ un entier positif, $(\Phi, \Psi)$ une paire de bases orthonormales, $M$ la quantité définie 
		\begin{equation}
			M = \sup_{i,j} |\langle \varphi_i, \psi_j \rangle|
		\end{equation}
		et un signal $f\in \mathbb{R}^N$, alors avec probabilité tendant vers 1, n'importe quel $x = (x_\Phi, x_\Psi)$ vérifiant $f = F_\Phi x_\Phi + F_\Psi x_\Psi$ et
	\begin{equation}\label{eq:cond1}
		||x_\Phi||_0 +  ||x_\Psi||_0 < C \frac{N}{M^2\log(N)}
	\end{equation}
	est l'unique solution de \ref{P1}, et c'est la solution de \ref{P0}.
\end{theoreme}






%\chapter{Generalised models}
%\input{Chapters/ConfigurationModel.tex}

\appendix

\chapter{Annexe}
\section{Valeurs propres de $F_\Omega F_\Omega ^*$ et Analyse en composante principale}

\section{Outils probabilistes de la preuve du théorème}


%\nocite{*}
\printbibliography

\end{document}
