\section{Traitement du signal, analyse et synthèse}
La reconstruction du signal est un problème que l'on considère dans le cadre du traitement du signal, c'est à dire que l'on considère qu'à un signal, on peut appliquer une transformation, et de cette transformation on obtient un nouveau signal qui aura certaines caractéristiques permettant de mieux comprendre ce signal.
D'un point de vue plus formel, on considère une famille de signaux $\mathcal{F}$, chaque élément de cette famille étant une application $f : X \longrightarrow Y$, et on considère un opérateur $F : \mathcal{F} \longrightarrow \mathcal{G}$, où $\mathcal{G}$ est une autre famille de signaux.
\newline
Par exemple une famille de signaux que l'on considérera est $\mathbb{R}^N$ et on verra donc $f\in \mathbb{R}^N$ comme une application $\llbracket 0,N-1 \rrbracket \rightarrow \mathbb{R}$.
Une autre famille que l'on étudiera est $L^2(\mathbb{R})$ et de façon plus générale les espaces de Hilbert. 
\newline
L'objectif de ce mémoire est de présenter des techniques permettant la reconstruction de signaux, donc on cherchera à obtenir des opérateurs $F	:\mathcal{F} \rightarrow \mathcal{F}$ qui vérifieront $f=F(f)$. 
La première étape en traitement du signal est la mesure, à partir d'un signal, on lui associe des coefficients, ainsi on a une application
\begin{align}
	A : 	\mathcal{F} &\longrightarrow \mathcal{G} \\
		f &\longmapsto A(f) = (a_i(f))_{i\in I}
\end{align}
que l'on appelle \emph{opérateur d'analyse}.
\newline
Des opérateurs d'analyse que l'on rencontrera seront par exemples la transformée de Fourier, qui a une application associe ses coefficients de Fourier, ou bien la transformée en ondelette, qui a une application associe ses coefficients d'ondelette.
Un autre exemple d'opérateur que l'on étudiera est celui qui à un élément de $\mathbb{R}^N$ lui associe ses coefficients dans deux bases, donc un élément de $\mathbb{R}^{2N}$.
Pour donner un dernier exemple d'opérateur que l'on considérera consistera à associer à un élément de $\mathbb{R}^N$ les valeurs de ses projection sur $M$ vecteurs aléatoires de $\mathbb{R}^N$, donc un élément de $\mathbb{R}^M$.
\newline 
TODO : ajouter des figures avec un signal et sa transformée de Fourier.

\begin{figure}[h]%%Wikicommons
\centering
\begin{subfigure}{.5\textwidth}
  \centering
  \includegraphics[width=.9\linewidth]{Figs/castle.png}
  \label{fig:sub1}
\end{subfigure}%
\begin{subfigure}{.5\textwidth}
  \centering
  \includegraphics[width=.9\linewidth]{Figs/castle_wavelet.png}
  \label{fig:sub2}
\end{subfigure}
	\caption{Une photographie et sa tranformée en ondelettes.\footnote{Images de Wikicommons}}
\label{fig:test}
\end{figure}
Comme nous souhaitons reconstruire des signaux, notre objectif est de pouvoir utiliser ces coefficients afin de revenir dans l'espace initial.
On définit donc l'application
\begin{align}
	S : \mathcal{G} &\longrightarrow \mathcal{F}\\
		A(f) &\longmapsto S(A(f))
\end{align}
que l'on appelle \emph{opérateur de synthèse}.
\newline
On aura donc une formule de reconstruction si $S \circ A = Id_{\mathcal{F}}$.
Par exemple, avec $\mathcal{F}$ bien choisi, des opérateurs de synthèse que l'on rencontrera seront par exemple, la transformée de Fourier inverse, ou bien la transformée en ondelette inverse, qui a des coefficients de Fourier, respectivement d'ondelettes, associe le signal initial.
La première partie du chapitre 2 introduira la notion de frame qui nous permettra de caractériser des opérateurs sur lesquels on aura la garantie d'avoir une formule de reconstruction.

\begin{figure}[h]%%Courtesy of Allen D. Elster, MRIquestions.com
\centering
\begin{subfigure}{.5\textwidth}
  \centering
  \includegraphics[width=.9\linewidth]{Figs/kspacefourier.jpg}
  \label{fig:sub1}
\end{subfigure}%
\begin{subfigure}{.5\textwidth}
  \centering
  \includegraphics[width=.9\linewidth]{Figs/kspacebrain.jpg}
  \label{fig:sub2}
\end{subfigure}
	\caption{Mesure par résonnance magnétique d'un cerveau (le signal), qui correspond à des coefficients de Fourier (l'analyse), et reconstruction du signal (la synthèse).\footnote{Images du site \url{www.MRIquestions.com} de Allen D. Elster.} }
\label{fig:test}
\end{figure}

\section{Reconstructions atomiques et parcimonie}
Dans ce mémoire on va, suivant l'approche de Stéphane Mallat\footnote{L'approche parcimonieuse est au coeur du programme de Stéphane Mallat, son livre qui fait référence en traitement du signal notamment concernant les ondelettes \emph{A wavelet tour of signal processing} s'est vu dans sa dernière édition ajouter le sous-titre \emph{The sparse way} et le cours présenté durant sa chaire au Collège de France était centré sur \emph{Le triangle "régularité, approximation, parcimonie"}.},
approcher la problématique de la recontruction et du traitement du signal en donnant une place centrale à la notion de parcimonie.
De façon large, l'approche parcimonieuse consistera soit à minimiser le nombre de coefficients obtenus par l'analyse ou utilisés par la synthèse, soit à exploiter, à l'aide d'un principe d'incertitude, l'hypothèse que le signal est représentable avec peu de coefficients.
Décrivons donc ces trois stratégies que l'on développera dans ce mémoire.
\newline

La première approche consiste à minimiser le nombre de coefficients que l'on obtient lors de l'analyse, ainsi, cela revient à choisir un opérateur d'analyse qui est adapté à $\mathcal{F}$.
On verra ainsi, que si on étudie des fonctions avec certaines propriétés de régularité, il est possible de choisir une transformation en ondelettes pour laquelle les coefficients seront rapidement faibles. 
Donc, pour un signal donné, il peut être intéressant de regarder, pour plusieurs transformées en ondelettes, laquelle est celle qui a le minimum de coefficients non nuls (c'est le principe de l'algorithme Best Ortho Basis de Coifman et Meyer \cite{coifortho}).
Par exemple, pour reconstruire une image en deux dimensions, il est possible de reconstruire cette image en choisissant parmi des ondelettes de Daubechies, des coiflets, des curvelets, des brushlets, ...
chacune de ces ondelettes étant plus ou moins adaptée pour représenter des images avec différentes propriétés.
On verra ainsi dans la seconde partie du chapitre 2 que les ondelettes ayant des moments nuls et à support compact (les ondelettes de Daubechies et les coiflets) permettent une reconstruction avec une décroissance rapide de leurs coefficients pour les fonctions Lipschitziennes.

\begin{figure}[h]%%Wikicommons
\centering
\begin{subfigure}{.5\textwidth}
  \centering
  \includegraphics[width=.9\linewidth]{Figs/daubechies.png}
  \label{fig:sub1}
\end{subfigure}%
\begin{subfigure}{.5\textwidth}
  \centering
  \includegraphics[width=.9\linewidth]{Figs/coiflet.png}
  \label{fig:sub2}
\end{subfigure}
	\caption{L'ondelette de Daubechies à 2 moments nuls et la coiflet à 2 moments nuls.\footnote{Images de Wikicommons}}
\label{fig:test}
\end{figure}

Les deux autres approches reposent sur le fait que l'on dispose de différentes façons de représenter un signal.
Par exemple dans le cadre de l'algèbre linéaire et d'un vecteur $f \in \mathbb{R}^N$ ou dans un espace de Hilbert, chaque choix de base revient à choisir un couple d'opérateurs analyse-synthèse $(A, S)$ et donc à une formule de reconstruction.
De la même façon, chaque transformée en ondelettes donne un couple d'opérateurs analyse-synthèse $(A, S)$, on a ainsi le double bénéfice des ondelettes qui, en plus d'avoir de bonnes propriétés, existent en grand nombre.
\newline
Avec cette multitude de représentations disponibles d'un même objet, la notion de \emph{dictionnaire} a été développée.
Introduisons cette notion par l'algèbre linéaire, un dictionnaire est alors simplement une famille de vecteurs $\{f_i\}_I$. Disons que ce dictionnaire engendre $\mathcal{F}$, alors pour chaque signal $f\in\mathcal{F}$, il existe au moins une suite de coefficients $(c_i)_J$,$J\subset I$, telle que
\begin{equation}
	f = \sum_J c_i f_i.
\end{equation}
Ainsi, on peut identifier la suite de coefficients $(c_i)_J$ à $f$, et chacune de ces suites de coefficients donne lieu à une identification avec $f$.
Une telle suite de coefficients peut être appellée une décomposition atomique de $f$, chaque vecteur $f_i$ étant alors appelé un atome.
Avec une approche parcimonieuse, on souhaite alors pour un signal donné, trouver la décomposition atomique qui utilise un minimum d'atomes. 
Cela correspond donc à utiliser un minimum de coefficients pour la synthèse de $f$, on appelera ce problème \ref{P0} et on montrera dans le début du chapitre 3 des conditions pour avoir l'unicité d'une telle décomposition.
Obtenir une telle unicité signifie qu'il n'y a pas deux décompositions distinctes qui soient concentrées sur peu de coefficients, on dira alors que le dictionnaire vérifie un principe d'incertitude.
\newline
Cependant ce problème est de nature combinatoire et très difficile à résoudre de façon générale (c'est un problème NP-difficile \cite{NPHard})
On verra dans la suite du chapitre 3 que sous certaines conditions plutôt que de résoudre directement le problème \ref{P0}, on pourra passer par un problème beaucoup plus simple \ref{P1} et trouver la décomposition atomique.
Les conditions à vérifier étant que le dictionnaire vérifie un principe d'incertitude et qu'une décomposition suffisamment parcimonieuse existe.
\newline 
La dernière approche et la plus récente est celle du compressed sensing \cite{DonohoCS}, \cite{CandesTaoUniv}, \cite{CandesSparsityIncoherence}. 
Dans ce cadre, on souhaite reconstruire un signal $x_0 \in \mathbb{R}^N$ avec $k << N$ composantes non nulles, dans une certaine base, on se demande alors si il est possible de reconstruire ce signal en faisant $m<N$, ou même $m\sim k$, mesures de $x_0$.
On verra qu'il est possible de répondre à cette question par l'affirmative.
Une partie des arguments provient de l'approche précédente, la mesure devra vérifier un principe d'incertitude avec la base dans laquelle le signal est parcimonieux, une fois que l'on a un tel opérateur d'analyse, en résolvant le problème de décomposition atomique on devrait pouvoir retrouver $x_0$.
Cependant, on ne connait pas à l'avance dans quelle base le signal est parcimonieux, l'approche prend alors un tournant probabiliste et on va chercher à faire une mesure qui va presque sûrement vérifier un principe d'incertitude avec la base inconnue.
\newline
Donnons un exemple du type de résultat auquel on peut arriver à partir du compressed sensing, supposons que l'on souhaite reconstruire une image $x_0$ en niveux de gris de taille $1000\times1000$, donc un vecteur dans $\mathbb{R}^d$ avec $d=10^6$. 
Supposons que ce vecteur a $20 000$ coefficients d'ondelette non nuls, alors avec $10^5$ mesures, en cherchant une décomposition atomique qui coincide avec ces mesures, on devrait pouvoir reconstruire exactement $x_0$.
Le mécanisme (d'après \cite{DonohoForMost}) qui fait que l'on peut reconstruire ce signal est similaire au fait que si l'on tire $10 000$ points au hasard dans $\mathbb{R}^{10^6}$ (l'espace de l'image) et que l'on forme un cône partant de l'origine avec ces points, alors faire coincider le signal avec les $10^5$ mesures revient à choisir comment passer d'un plan à $10^5$ dimensions à ce cône.
Le compressed sensing est alors rendu possible par le fait que la plupart des plans à $900 000$ dimensions ne coupent pas le cône, donc le signal reconstruit avec $10^5$ mesures devrait effectivement être le signal original.

\section{Exemples d'applications}
Maintenant que l'on a vu l'approche que nous allons utiliser pour la reconstruction de signaux, voyons quelques exemples de problèmes de traitement du signal sur lesquels les techniques développées dans ce mémoire donnent des méthodes de résolution.
Afin de formaliser ces problèmes nous aurons besoin de $||\cdot||_1$ et $||\cdot||_2$ les normes sur $\ell^1$ et $\ell^2$, ainsi que la "norme" 0 $||\cdot||_0$ définie pour $x:=(c_I)_{i\in I}$ par  $||x||_0 = |\{i: c_i \neq 0\}|$. 
Cette "norme" permet de mesurer la parcimonie de l'écriture de $x$, cependant, bien qu'on l'appelle ici norme (en omettant les guillemets), ce n'en est pas une car elle n'est pas homogène.
L'un des objectifs de ce mémoire est de comprendre comment manipuler et obtenir des résultats avec cette norme, de nombreux résultats seront ainsi obtenus en la reliant à la norme $||\cdot||_1$.
\begin{exemple}[Compression]
	Soit $y \in \mathbb{R}^N$ un signal, est-ce possible de représenter $y$ avec (nettement) moins de $N$ coefficients ?
	\newline
	Normalement on doit décrire $y$ avec ses $N$ coefficients, avec ce qui est développé dans ce mémoire, plusieurs méthodes sont proposées pour décrire $y$ avec $k<N$ coefficients.
	On répondra à cette question sous des conditions différentes avec la théorie des frames et avec la décomposition atomique.
\end{exemple}	
\begin{exemple}[Débruitage]
	Soit $y \in \mathbb{R}^N$ un signal et soit $\epsilon \in \mathbb{R}^N$ un vecteur avec des petits coefficients.
	Est-ce possible de retrouver $y$ à partir de $y+\epsilon$ ?
	\newline
	On répondra à cette question avec la théorie des frames serrés et avec la décomposition atomique.
\end{exemple}
\begin{exemple}[Problèmes inverses]
	Soit $y$ un signal dont on fait une mesure indirecte, $\tilde{y} = Hy$ avec $H$ un opérateur linéaire. 
	Est-ce possible de trouver une solution $x$ parcimonieuse à $\tilde{y} = HAx$ ?
	\newline
	Dans ce cas là $Ax$ donne une décomposition atomique de $y$ avec des colonnes de $A$ comme atomes. Sous certaines conditions, la partie sur la décomposition atomique peut permettre de résoudre ce type de problèmes.
\end{exemple}
\begin{exemple}[Tomographie et mesure de Radon]
	Dans ce contexte on cherche à mesurer un objet que l'on ne peut pas observer directement (par exemple la composition de la croûte terrestre en géologie sysmique ou bien un organe en imagerie médicale par rayons X), la technique consiste alors à faire passer au travers de ce système un signal que l'on contrôle (par exemple une onde de choc ou bien un rayon X) et à en mesurer la réponse.

	Soit $x=(x_i)_I$ un signal connu que l'on fait passer à travers un système $A = (a_i)_I$, on obtient donc $y_x = \sum_I a_i x_i$.
	On se demande alors si il est possible de trouver une décomposition atomique de $A$.
	Des techniques basées sur la recherche d'une telle décomposition ont été développées par exemple pour la géologie avec des ondelettes \cite{TomoInv}, pour l'imagerie par rayons X \cite{CSTom} \cite{XraySparse} et plus généralement pour des problèmes d'inversion de la transformée de Radon \cite{RobustSparseRadon} \cite{ViewsSparseRadon}.
	Dans ces articles des techniques relevant de la décomposition atomique (en minimisant la norme $||\cdot||_1$) ou du compressed sensing sont utilisées.
\end{exemple}
\begin{exemple}[Imagerie par résonnance magnétique (IRM)]
	Terminons cette série d'exemples avec l'apport du compressed sensing aux techniques d'IRM, on se base sur \cite{LustigCS} pour la discussion. 
	
	En IRM l'objectif est de reconstruire des images de l'intérieur d'un corps humain (par exemple), ainsi la mesure se doit d'être indirecte et sans danger.
	Cette mesure est faite en induisant un champ magnétique pour aligner la polarisation des protons contenus dans le corps, ensuite des trains d'ondes radio-fréquence sont envoyés qui vont provoquer le désalignement de ces protons.
	Peu après ces photons se réorientent avec le champ magnétique et c'est ce changement qui est mesurable.
	Comme décrit dans l'article précédemment cité, ce qui est mesuré au cours du temps par les bobines correspond à la transformée de Fourier d'une section du corps à différentes fréquences spatiales.
	\newline
	Ce que l'on retient de cela du point de vue du traitement du signal c'est que l'IRM permet d'obtenir la transformation de Fourier d'un objet et du point de vue de la reconstruction on cherche à partir de cette transformation à retrouver l'objet mesuré.
	Cependant, il est important de noter que la mesure est limitée en vitesse par des contraintes physiques et physiologiques inévitables (voir \cite{LustigCS}) et donc il n'est pas possible de connaitre l'ensemble des coefficients de Fourier.
	\newline
	Formalisons donc ce problème, notons $F$ la transformée de Fourier, et $F_I$ la restriction de $F$ aux $I$ coefficients mesurés mesurés.
	Notons $m$ la section de l'objet que l'on cherche à trouver, et on a donc $y=F_I m$ le signal obtenu.
	Cependant l'équation générale $y=F_I x$ n'admet pas de solution unique, il nous faut donc rajouter une contrainte.
	Grâce au chapitre 2 on saura que l'on peut généralement représenter une image avec peu de coefficients dans une base d'ondelette, notons cette décomposition $W x$.
	Grâce au compressed sensing on verra que le problème de minimiser $||Wx||_1$ parmi les $x$ qui vérifient $y = F_I x$ admet une unique solution, que l'on peut calculer et c'est $x=m$. 
	Ce type de résultat correspond à une extension des résultats du chapitre 3.
	\newline
	Ainsi, avant le compressed sensing les images reconstruites demandaient une longue durée de scan (6mn avec certaines périodes d'apnée) et les résultat n'étaient pas de très bonne qualité.
	Après le compressed sensing, la reconstruction ne nécessitant pas de récupérer tous les coefficients, les scans sont devenus bien plus rapides et de meilleure qualité (25s avec respiration libre).	
\end{exemple}
Avant de poursuivre et de s'éloigner des applications pour passer à la théorie, notons quelques unes de leurs caractéristiques auxquelles on veillera dans les résultats obtenus.
\newline
Tout d'abord, les problèmes énoncés sont résolubles avec les méthodes de ce mémoire sous certaines conditions, ainsi, sur certains problèmes appliqués, par exemple ceux dont la résolution implique l'utilisation d'une décomposition atomique, l'existence d'une telle décomposition n'est pas automatique.
En effet, il est nécessaire d'avoir un dictionnaire avec des atomes permettant une décomposition atomique, et en pratique cela peut être très difficile de trouver un tel dictionnaire pour certains problèmes.
\newline
Ensuite, certains de ces problèmes sont issus de la physique, il y a donc de fortes chances pour que les mesures soient perturbées et avec une certaine imprécision. 
Il sera donc intéressant de vérifier que les techniques de reconstruction que l'on utilise sont stables lorsque la solution est légèrement modifiée.
De la même façon, en pratique, dans les égalités du type $y=Ax$ des exemples ci-dessus, on cherchera plutôt à résoudre $||y-Ax||_2 < \varepsilon$.
\newline
Finalement, voici un dernier point auquel il faut veiller : la première étape étant la mesure ; il sera intéressant de pouvoir donner pour certaines classes de signaux le nombre minimal de mesures nécessaires afin de reconstruire le signal mesuré.
Le dernier exemple nous indique aussi qu'il peut être intéressant de pouvoir dire quand est-ce qu'on a suffisament de mesures pour pouvoir reconstruire le signal.

