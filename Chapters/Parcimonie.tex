\section{Signaux ayant une représentation parcimonieuse}
\subsection{Définition d'une représentation parcimonieuse}
\subsection{Importance de la base}

\section{Décroissance des coefficients et régularité}
\subsection{Approximation linéaire et régularité}
On s'intéresse dans cette partie au lien entre une fonction $f\in L^2(]0, 1[)$ et son approximation dans une base. On verra un résultat reliant la décroissance des coefficients de la fonction dans une base fixée et la vitesse de convergence de la reconstruction. 
On verra ensuite à l'aide de ce résultat, que pour la base de Fourier (et resp. certaines bases d'ondelettes), on obtient des formules de reconstruction pour les fonctions dérivables (et resp. pour les fonctions Lipschitziennes) avec une erreur de reconstruction qui décroit rapidement.
\newline 
On considère ainsi un espace d'approximation de fonctions $U_N \subset L^2([0, 1])$.
Par construction, la meilleure approximation linéaire de $f$ dans $U_N$, est la projection orthogonale $f_N$ de $f$ dans $U_N$, qui peut être obtenue à l'aide de la base biorthogonale de synthèse associée $(\tilde{\phi}_k)_{k=1, \cdots, N}$ et la formule de reconstruction :
\begin{equation}
	f_N = \sum_{k=0}^{N-1}\langle f, \phi_k \rangle \tilde{\phi}_k.
\end{equation}
Afin de mesurer l'erreur d'approximation par rapport à $f$, on considère une base $\mathcal{B} = \{g_k\}_{k\in \mathbb{N}}$ de l'espace $L^2([0, 1])$ entier à laquelle on ajoute la condition de contenir une famille $(g_k)_{k\in I}$, avec $\#I = N$ qui forme une base de l'espace d'approximation $U_N$.
On peut ainsi écrire, en réordonnant la famille $(g_k)$,  $f_N \in U_N$ dans cette base :
\begin{equation}
	f_N = \sum_{k=0}^{N-1} \langle f, g_k \rangle g_k 
\end{equation}
et les $\{g_k\}_{\mathbb{N}}$ formant une base de $L^2([0, 1])$, on peut écrire $f$ dans cette base:
\begin{equation}
	f = \sum_{k=0}^{+\infty} \langle f, g_k \rangle g_k.
\end{equation}
On obtient donc que la partie orthogonale à la famille $\{\phi_k\}_{k=0, \cdots, N-1}$, est celle analysée par $\{g_k\}_{k\geq N}$.
C'est à dire, 
\begin{equation}
	f - f_N = \sum_{k=N}^{+\infty} \langle f, g_k \rangle g_k 
\end{equation}
et la mesure de l'erreur d'approximation avec $N$ coefficients est donc 
\begin{equation}
	\varepsilon_l(N, f) = ||f-f_N||^2 = \sum_{k=N}^{+\infty} |\langle f, g_k\rangle|^2.
\end{equation}
Comme on a supposé que $f\in L^2([0, 1])$ et que la famille $(g_k)$ est génératrice, on a que l'erreur d'approximation tend vers $0$ lorsque $N$ augmente.
On va maintenant s'intéresser au théorème suivant de Stéphane Mallat qui relie la décroissance des coefficients de $f$ dans la base de $L^2([0, 1])$ à la vitesse de décroissance de l'erreur d'approximation de la fonction.
\begin{theoreme}
	Soit $r > 1/2$, il existe des constantes $A, B > 0$ telles que si 
	\begin{equation}
		\sum_{k=0}^{+\infty} |k|^{2r}|\langle f, g_k \rangle |^2 < \infty,
	\end{equation}
	alors on a 
	\begin{equation}\label{eq:regframe}
		A \sum_{k=0}^{+\infty} k^{2r}|\langle f, g_k \rangle |^2 \leq 
		\sum_{N=0}^{+\infty} N^{2r-1} \varepsilon_l(N, f) \leq
		B \sum_{k=0}^{+\infty} k^{2r} |\langle f, g_k\rangle |^2
	\end{equation}
	et ainsi on a $\varepsilon_l(N, f) = o(N^{-2r})$.
\end{theoreme}
\begin{proof}
	On développe le terme au centre de l'égalité dela façon suivante 
	\begin{equation*}
		\sum_{N=0}^{\infty} N^{2r-1} \varepsilon_l(N, f) =
		\sum_{N=0}^{\infty} \sum_{k=N}^{\infty} N^{2r-1} |\langle f, g_k \rangle |^2 =
		\sum_{k=0}^{+\infty} |\langle f, g_k \rangle |^2 \sum_{N=0}^{k} N^{2r-1}.
	\end{equation*}
	Puis on majore des deux côtés avec 
	\begin{equation}
		\int_{0}^M x^{2r-1} dx \leq \sum_{N=0}^m N^{2r-1} \leq \int_{1}^{m+1} x^{2r-1}dx.
	\end{equation}
	En calculant les deux intégrales on déduit 
	\begin{equation}
		A m^{2r} \leq \sum_{N=0}^{m} N^{2r-1} \leq B m^{2r}
	\end{equation}
	où $A$ et $B$ dépendent seulement de $r$, ce qui nous donne \ref{eq:regframe}.
	Montrons maintenant $\varepsilon_l(N, f) = o(N^{-2r})$, remarquons tout d'abord que $\varepsilon_l(N, f)$ est décroissant par rapport à la première variable, on déduit de cela
	\begin{equation*}
		\varepsilon_l(N, f)\sum_{m=N/2}^{N-1} m^{2r-1} \leq \sum_{m=N/2}^{+\infty} m^{2r-1} \varepsilon_l(m, f)
	\end{equation*}
	on a donc avec le calcul précédent que le terme de droite converge quel que soit le choix de $N$, et ainsi
	\begin{equation*}
		\lim_{N \to +\infty} \sum_{m=N/2}^{+\infty} m^{2r-1}\varepsilon_l(m, f) = 0
	\end{equation*}
	donc tous les termes de l'inégalité précédente tendent vers 0. 
	De plus, $\sum_{m = N/2}^{N-1} m^{2r-1} \geq CN^{2r}$ et donc
	\begin{equation*}
		\lim_{N\to \infty} \varepsilon_l(N, f)N^{2r} = 0.
	\end{equation*}
\end{proof}
On a ainsi démontré que si $f$ appartient à l'espace
\begin{equation}
	W_{\mathcal{B}, r} =\{f : \sum_{m=0}^{+\infty} m^{2r} |\langle f, g_m \rangle |^2 < \infty\}
\end{equation}
alors l'approximation linéaire dans la base $\mathcal{B}$ décroit au moins comme $N^{-2r}$.
On montrera dans la prochaine sections que si $\mathcal{B}$ est une base de Fourier alors $W_{\mathcal{B}, r}$ contient les fonctions $r$-différentiables. 
On montrera ensuite que si $\mathcal{B}$ est une base d'ondelette avec une certaine propriété alors l'espace $W_{\mathcal{B}, r}$ contient les fonctions $\alpha$-Lipschitziennes pour $1 < \alpha < r$.
Des énoncés réciproques existent aussi et des démonstrations de ceux-ci peuvent être trouvés dans ( TODO :Ajouter Ref )
\subsection{Décroissance des coefficients de Fourier}
On considère ici $\mathcal{B} = (e_{n})_{n \in \mathbb{Z}}$ la base de Fourier ( voir 1.3.3) de $L^2(\mathbb{R})$.
On peut ainsi définir l'espace $U_N = \{f \in L^2(\mathbb{R}) : |k| > N \implies \hat{f}(k) = 0\}$ sur lequel $\mathcal{B}_N = (e_{n})_{|n|\leq N}$ est une base. 
Avec des mots, $U_N$ est l'espace des fonctions qui ne sont portées par aucune exponentielle complexe de fréquence supérieure ou égale à $N$.
Le théorème de Shannon nous indique que $2N$ fréquences permettent de séparer n'importe laquelle de ces fonctions, ainsi on  a la formule de reconstruction.
On dispose ainsi d'une formule de projection (et de reconstruction), soit $f \in L^2(\mathbb{R})$
\begin{equation}
	f_N(t) = \sum_{|n|\leq N/2} \langle f, e_n \rangle e_n(t) \in U_N.
\end{equation}
Ainsi $f_N$ est une approximation linéaire de $f$, et $f$ sera rapidement approximée si $f$ n'a pas trop de hautes fréquences. 
Montrons maintenant que la vitesse de décroissance des coefficients de Fourier est liée à la régularité de la fonction.
Tout d'abord, revenons à $L^2(\mathbb{R})$ considérons que $f$ est dérivable, alors on a en intégrant par parties
\begin{equation}
	\hat{f}'(\omega)\int_{-\infty}^{+\infty} f'(t) e^{i \omega t} dt = i \omega \int_{-\infty}^{+\infty} f(t) e^{i\omega t} dt = i \omega \hat{f}(\omega) 
\end{equation}
et en utilisant la formule de Plancherel on a 
\begin{equation}
	||\hat{f}'||_2^2 = \int_{-\infty}^{+\infty} |\omega|^2 |\hat{f}(\omega)|^2 d\omega = \int_{-\infty}^{+\infty} |f'(t)|^2 dt = ||f'||_2^2.
\end{equation}
On est ainsi amenés à définir une régularité dans $L^2(\mathbb{R})$, distincte de la dérivabilité en un point avec la définition suivante :
\begin{definition}
	On dit que $f\in L^2(\mathbb{R})$ est différentiable au sens de Sobolev si
	\begin{equation*}
		\int_{-\infty}^{+\infty} |\omega|^2|\hat{f}(\omega)|^2 d\omega < \infty. 
	\end{equation*}
\end{definition}
Et avec cette définition on peut définir pour n'importe quel $r>0$ l'espace des fonction $r$-différentiables de Sobolev:
\begin{equation}
	W^r(\mathbb{R}) =\{ f \in L^2(\mathbb{R}) : \int_{\mathbb{R}} |\omega|^{2r} |\hat{f}(\omega)|^2 d\omega < \infty \}.
\end{equation}
Ainsi d'après le théorème sur la vitesse d'approximation de Mallat, l'approximation linéaire dans la base de Fourier d'une application $r$-différentiable décroit plus vite que $N^{-2r}$.
\subsection{Décroissance des coefficients d'ondelettes}
On va montrer dans cette section que en imposant certaines conditions sur les ondelettes, alors il est possible de démontrer que la régularité au sens de Lipschitz, implique une décroissance des coefficients d'ondelettes.
On pourra ensuite relier cette décroissance aux discussions de la fin de la partie précédente.
\newline
Tout d'abord posons les définitions dont nous aurons besoin dans cette partie,
\begin{definition}
	Soit $\alpha$ tel qu'il existe un entier strictement positif $r$ tel que $r-1 \leq \alpha <r$. On dit qu'une fonction $f$ est $\alpha$-Lipschitz en $t_0$ si il existe une constante $C>0$ et un polynôme $P_{t_0}$ de degré strictement inférieur à $r$, tels que pour tout $t$ qui appartient à un voisinage $T_0$ de $t_0$, on a 
	\begin{equation*}
		|f(t) - P_{t_0}(t)| \leq C|t-t_0|^{\alpha}
	\end{equation*}
\end{definition}
Avec cette définition on vérifie immédiatement que si on considère un signal $r$-dérivable au sens classique, alors en utilisant l'approximation avec un polynôme de Taylor du signal, on a pour tout $0 < \alpha \leq r$, que le signal est partout $\alpha$-Lipschitz.
\newline
On peut facilement relier cette définition avec les ondelettes, en considérant les moments d'ondelette.
On considère ainsi $\{\psi_{j, k} = \psi( 2^{\frac{j}{2}}\psi(2^j\cdot -k)\}_{j,k}$ une base orthonormale d'ondelettes de $L^2([0, 1])$, et on a le coefficient d'ondelette à l'échelle $j$ et à l'instant $k$ donnée par
\begin{equation*}\label{eq:defWx}
	Wf (j, k) = \langle f, \psi_{j,k} \rangle = \int f(t) \psi_{j,k}(t)dt.
\end{equation*}
\begin{remarque}
Remarquons ici que l'on peut exprimer cette projection à partir de l'ondelette prise à l'instant $0$
\begin{equation*}
	\widecheck{\psi_j}(t) = \psi_{j, 0}(-t) = \psi(-2^jt),
\end{equation*}
et ainsi en faisant un changement de variable dans \ref{eq:defWx}, on obtient
\begin{equation}
	Wf(j,k) = f \star \widecheck{\psi_j}(k).
\end{equation}
On peut ainsi interpreter le coefficient d'ondelette pris en $(j,k)$ comme la corrélation entre le signal pris en $k$ avec une ondelette à l'échelle $j$.
C'est à dire qu'un coefficient avec une grande valeur indique une grande similitude entre l'ondelette et le signal (l'ondelette approxime bien le signal) alors qu'un petit coefficient indique que l'ondelette et le signal ont peu en commun\footnote{L'ondelette et le signal ont peu en commun au sens où ils sont de façon équivalente, presque orthogonaux, et donc ce coefficient à un poids faible dans la formule de reconstruction.}.
\end{remarque}
Rappelons aussi qu'un condition nécessaire pour que $\psi$ soit une ondelette génératrice est
\begin{equation*}
	\int \psi (t)dt = 0.
\end{equation*}
\begin{definition}
	Soit $m$ un entier strictement positif.
	On dit que $\psi$ a $m$-moments nuls si
	\begin{equation*}
		\int t^k \psi(t)dt = 0 \quad, \forall k < m.
	\end{equation*}
\end{definition}
De cette définition on déduit que si $P$ est un polynôme de degré strictement inférieur à $m$ et si l'ondelette a $m$-moments nuls, alors
\begin{equation}\label{eq:momPol}
	\int \psi(t) p(t) dt = 0.
\end{equation}
Ainsi si $f(t) = P(t) +\epsilon$ où $\epsilon$ représente un bruit, on a 
\begin{equation*}
	\langle f, \psi_{j,k} \rangle = o(\epsilon)
\end{equation*}
ainsi les coefficients de l'ondelette d'échelle suffisent à reconstruire $f$.
C'est ainsi que si on considère une ondelette avec un certain nombre de moments nuls, alors dans les parties régulières du signal, les coefficient d'ondelettes seront petits, alors que dans les zones avec des irrégularités ou des discontinuités, les coefficients resteront grands.
On peut aussi remarquer que si les irrégularités sont séparées, alors en affinant l'échelle d'analyse, le support des ondelettes diminue et alors de moins en moins de coefficients auront une valeur importante, ainsi les seuls coefficients qui resteront grand en changeant d'échelle sont ceux qui contiennent une zone irrégulière. 
\newline
On va maintenant démontrer le théorème suivant de Jaffard qui permet de préciser cela,
\begin{theoreme}\label{th:Jaffard}
	Si $f$ est $\alpha$-Lipschitz en $t_0$ avec $0 < \alpha \leq m$ où $m$ est un entier.
	Alors, il existe une constante $C >0$ telle que 
	\begin{equation*}
		|Wf(j, k)| \leq C2^{-j(\alpha + \frac{1}{2})}(1 + |2^{-j}t - t_0|2^{j\alpha}).
	\end{equation*}
\end{theoreme}
\begin{proof}
	Remarquons tout d'abord que, soit $P$ un polynôme de degré strictement inférieur à $m$, et $\psi$ une ondelette a $m$-moments nuls, alors, par changement de variable et d'après \ref{eq:momPol} on a :
	\begin{equation}\label{eq:polAnn}
		WP(j, k) = \int 2^{j/2} \psi(2^jt - 2^{-j}k) P(t)dt 
		= \int 2^{j/2}\psi(t')P(2^jt' - 2^{-j} k) 2^{-j} dt' = 0.
	\end{equation}
	Par hypothèse, $x$ est $\alpha$-Lipschitz en $t_0$, donc il existe $P_{t_0}$ un polynôme de degré inférieur à $m$ tel que $|f(t) - P_{t_0}(t)| \leq C|t - t_0|^{\alpha}$.
	En utilisant la linéarité de l'intégrale et en appliquant une inégalité triangulaire on obtient
	\begin{align*}
		|Wf(j,k)| &= \left|\int (f(t) -P_{t_0}(t) + P_{t_0}(t)) \psi_{j, k}(t) dt\right| \\
			&\leq \left|\int (f(t) - P_{(t_0}(t)) \psi_{j,k}(t)dt\right| + \left|\int P_{t_0}(t) \psi_{j,k}(t)dt\right| \\ 
			&\leq \int| f(t) - P_{t_0}(t)| |\psi_{j,k}(t)|dt  
	\end{align*}
	la dernière inégalité étant obtenue en utilisant \ref{eq:polAnn} sur le terme de droite et en faisant entrer la valeur absolue dans la première intégrale.
	On utilise maintenant le fait que $f$ est $\alpha$-Lipschitz et on fait un changement de variable, on obtient ainsi
	\begin{align*}
		| Wf(j,k)| \leq \int |f(t) - P_{t_0}(t)||\psi_{j,k}(t)|dt &\leq C\int |t - t_0|^\alpha 2^{j/2} |\psi(2^jt - k)|dt \\
		&\leq C \int |2^{-j} t' + 2^{-j}k - t_0|^\alpha 2^{-j/2} |\psi(t')| dt'. 
	\end{align*}
	Pour obtenir l'inégalité suivante on utilise
	\begin{equation*}
		|a + b|^\alpha \leq |2*\max(|a|, |b|)|^\alpha \leq 2^\alpha (|a|^\alpha + |b|^\alpha)
	\end{equation*}
	et on a ainsi
	\begin{align*}
		|Wf(j,k)| &\leq 2^\alpha C \int (|2^{-j}t'|^\alpha + |2^{-j}k-t_0|^\alpha)2^{-j/2} |\psi(t')|dt'\\
		&\leq 2^\alpha C 2^{-j(\alpha + 1/2)}\left( \int|t'|^\alpha |\psi(t')|dt' + |2^{-j}k-t_0|^\alpha 2^{\alpha j}\int |\psi(t')|dt' \right)
	\end{align*}
	Ce qui donne le résultat dès que les intégrales considérées sont définies, ce qui est le cas par exemple si l'ondelette est à support compact ou bien à décroissance suffisamment rapide. 
\end{proof}
On peut combiner le théorème de Jaffard avec une analyse multi-échelle d'ondelettes avec la proposition suivante :
\begin{proposition}
	Soit $f:]0,1[ \to \mathbb{R}$ une fonction $\alpha$-Lispchitzienne avec $\alpha>1$, alors il existe une constante $C>0$ et une base d'ondelette orthonormales associée à une multiresolution $\{(\psi_{j,k})_{(i,j): j\geq J, 2^{j} > k\geq 0}\}, \{\varphi_{J,k}\}_k$ avec $m>\alpha$ moments nuls telle que  
	\begin{equation}
		\varepsilon_l(f, 2^J) = ||f - \sum_{k=0}^{2^J -1} \langle f, \varphi_{J,k}\rangle \tilde{\varphi}_{J,k}||_2^2 \leq C 2^{-2J\alpha} = CN^{-2\alpha}
	\end{equation}
	avec $N=2^J$.
\end{proposition}
\begin{proof}
	L'existence d'une telle base d'ondelette n'est pas démontrée ici, des constructions peuvent être trouvées dans (ajouter ref) pour obtenir des bases de $L^2(\mathbb{R})$, on peut ainsi considérer une telle multirésolution donnée par une ondelette de Daubechies ou bien une coiflet à $m>\alpha$ moments nuls.
	Il est ensuite possible, avec quelques difficultés d'obtenir depuis ces ondelettes, une base orthonormale de $L^2(]0,1[)$ (ajouter ref).
	Soit $f$ une fonction $\alpha$-Lipschitzienne sur $]0,1[$, on a ainsi d'après la partie sur les frames et l'existence de la base d'ondelette précédente admise, une formule de reconstruction
	\begin{equation*}
		f = \sum_{k=0}^{2^J -1} \langle f, \varphi_{J,k}\rangle \tilde{\varphi}_{J,k} + \sum_{j=J+1}^{+\infty}\sum_{k=0}^{2^j-1} \langle f, \psi_{j,k} \rangle \tilde{\psi}_{j,k}. 
	\end{equation*}
	On a ainsi, en réécrivant l'équation et en prenant la norme
	\begin{equation*}
		\varepsilon(f, 2^J) = ||f - \sum_{k=0}^{2^J -1} \langle f, \varphi_{J,k}\rangle \tilde{\varphi}_{J,k}||_2^2 =|| \sum_{j=J+1}^{+\infty}\sum_{k=0}^{2^j-1} \langle f, \psi_{j,k} \rangle \tilde{\psi}_{j,k}||_2^2. 
	\end{equation*}
	On peut alors majorer le terme de droite en utilisant le fait que la famille d'analyse est génératrice, on obtient
	\begin{equation*}
		\varepsilon(f, 2^J) \leq \sum_{j=J+1}^{+\infty} ||\sum_{k=0}^{2^j -1} \langle f, \psi_{j,k} \rangle \tilde{\psi}_{j,k} ||_2^2
	\end{equation*}
	et en utilisant le fait que les ondelettes sont normalisées on a
	\begin{equation*}
		\varepsilon(f, 2^J) \leq \sum_{j=J+1}^{+\infty} \sum_{k=0}^{2^j -1} |\langle f, \psi_{j,k} \rangle|^2
	\end{equation*}
	on utilise maintenant le théorème \ref{th:Jaffard} et on obtient\footnote{Le théorème de Jaffard est pour une fonction ponctuellement Lipschitzienne, on considère ici une fonction $\alpha$-Lipschitzienne en tout point, donc le terme en $(1 + \frac{|2^{-j}k -t_0|}{2^j})$ n'apparait pas.}
	\begin{align*}
		\varepsilon(f, 2^J) &\leq \sum_{j=J+1}^{+\infty} \sum_{k=0}^{2^j -1} C^2 2^{-j(2\alpha + 1)} \\
		&\leq \sum_{j=J+1}^{+\infty} C^2 2^{-j(2\alpha + 1)} 2^{j} = \sum_{j=J+1}^{+\infty} C^2 2^{-j2\alpha} \\
		&\leq \frac{C^2}{1 - 4^{-\alpha}} 2^{-2J\alpha}	
	\end{align*}
	ce qui prouve la proposition.
\end{proof}
\section{Résolution de (P0)}
Dans ce qui précède, nous nous sommes intéressés aux propriétés qui font qu'une famille de vecteurs permet de reconstruire une famillle de signaux.
Nous avons vu différentes bases (Fourier et ondelettes) et nous avons vu que ces bases permettent de reconstruire des signaux présentant un certain type de régularité avec des coefficients qui suivent une décroissance assez rapide.
\newline
On a par exemple vu que l'on pouvait reconstruire les fonctions Lipschitziennes avec une bonne précision en utilisant une base d'ondelettes orthonormale avec un certain nombre de moments nuls.
De plus, on a remarqué que si la fonction se comporte comme un polynôme d'un degré inférieur au nombre de moments nuls au voisinage d'un point, alors les coefficients d'ondelettes dans ce voisinage seront nuls.
De même, les seuls coefficients d'ondelette qui seront grands seront ceux au voisinage d'un point où aucune approximation par un polynôme de petit degré n'est efficace\footnote{Une analyse du théorème de Jaffard \ref{th:Jaffard} montre que les coefficients affectés par une discontinuité forment un cône dans les coefficients d'ondelette autour du point de discontinuité. Ce cône se visualise dans la représentation temps-fréquence des coefficients d'ondelette, il part du point de discontinuité et s'élargit en diminuant le coefficient d'échelle $j$. La largeur de ce cône dépend de la régularité $\alpha$ de la fonction.}.
Ainsi, la représentation avec ces ondelettes d'une fonction ne possédant que quelques points où elle est irrégulière sera approximée avec peu de coefficients.
Afin d'insister, l'intérêt de cela est que de façon naive, afin de déterminer une fonction, il faut connaitre sa valeur en chaque point, ainsi, si l'on souhaite faire un traitement par ordinateur de cette fonction, il faut stocker chacun des points de la fonction.
Avec ce que l'on a fait, on sait qu'en fait on peut reconstruire la fonction avec un plus petit nombre de coefficients que la fonction n'a de points.
En ce sens, la représentation en ondelettes d'une fonction Hölderienne est parcimonieuse (peu de coefficients non nuls), alors que la représentation par la valuation d'une fonction Hölderienne n'est pas parcimonieuse.
\newline
Nous allons maintenant nous intéresser à l'autre direction de ce problème, c'est à dire que nous allons supposer que l'on dispose d'une famille de vecteurs et que la fonction que l'on cherche à reconstruire est une somme parcimonieuse de vecteurs de cette famille.
Cependant on connait seulement la valuation de cette fonction et pas les vecteurs sous-jacents qui permettent de représenter la fonction de façon parcimonieuse.
Aussi, on n'a pas supposé que cette famille est libre donc il n'y a pas une unique façon d'obtenir cette solution, en fait il y a une infinité de solutions dès que la famille n'est pas libre.
On va voir cependant que l'hypothèse de parcimonie est cruciale et qu'elle nous permettra de récupérer exactement les coefficients qui permettent l'écriture parcimonieuse de cette fonction.
\subsection{Définition de (P0)}
Formalisons maintenant ce que nous avons dit ci-dessus. 
On considère $\mathcal{F}$ un espace vectoriel et utilisons un dictionnaire $\Phi = \Phi_1 \cup \cdots \cup \Phi_D$ de bases, où chaque $\Phi_d$ est une base de $\mathcal{F}$. 
Ainsi $\Phi$ est une concaténation de bases\footnote{Ainsi $\Phi$ est un frame équilibré d'après la première partie (TODO : ajouter ref)} et on s'intéresse aux façon d'écrire un signal $f\in \mathcal{F}$ dans $\Phi$, c'est à dire aux façon d'écrire
\begin{equation}\label{eq:defSSum}
	f = \sum_\gamma c_\gamma \phi_\gamma
\end{equation}
où l'indice $\gamma = (d, i)$ indique le dictionnaire $\Phi_d$ correspondant ainsi que le vecteur $\phi_{d, i} \in \Phi_d$.
On peut aussi écrire \ref{eq:defSSum} sous forme matricielle en posant $F_\Phi$ la matrice ayant pour lignes les vecteurs $\phi_\gamma$ et en posant $x = (c_\gamma)_\gamma$ la notation sous forme de vecteur de $x$, on utilisera aussi la notation $x = (x_d)_{d=1, \cdots, D}$
On s'intéresse ainsi aux solutions de 
\begin{equation}
	f = F_\Phi x.
\end{equation}
Comme discuté précedemment, le choix des coefficients $c_\gamma$ n'est pas unique dès que $D>1$, cependant notre objectif n'est pas simplement de reconstruire $f$ (car n'importe quelle base $\Phi_i$ permet déjà cela), mais de trouver l'écriture de $f$ avec le minimum de coefficients non nuls.
Ainsi, le problème que l'on cherche à résoudre est 
\begin{equation}\label{P0}\tag{P0}
	\min ||x||_0\quad \text{tel que } f = F_\Phi x,
\end{equation}
où $||x||_0 = \#\{\gamma : c_\gamma \neq 0\}$ est le nombre de coefficients non nuls de $x$.
Cependant, la résolution en toute généralité de ce problème n'est pas faisable, en effet résoudre ce problème nécessite de résoudre (P0) pour chaque combinaison de vecteurs du dictionnaire si $x$ est dans l'image.
Ainsi, le nombre de combinaisons possibles parmi tous les vecteurs croit bien trop vite pour être calculable en pratique, nous verrons donc comment résoudre ce problème en utilisant une autre méthode.
\newline
Il est important de noter qu'à ce stade il n'y a aucune raison de supposer que chercher une unique solution à (P0) a un sens.
En effet, quand on a choisi le dictionnaire $\Phi$ rien ne nous interdisait de prendre à chaque fois la même base et on aurait ainsi $D$ solutions identiques, ayant chacune la même parcimonie.
On a ainsi $D$ solutions, et si on prend une paire de solutions $x_1, x_2$, alors $F_\Phi(x_1 - x_2) = 0$, d'où on obtient qu'à n'importe laquelle des $D$ solutions, on peut ajouter, par exemple $x_1 - x_2$, et on obtient une nouvelle solution. 
Cependant, cette solution ne sera jamais moins parcimonieuse que l'une des $D$ solutions initiales.
Il est donc clair qu'il est nécessaire d'imposer des conditions sur les bases qui constituent le dictionnaire si l'on souhaite obtenir une solution unique.
Afin d'étudier cela commençons par un cadre simple dans lequel résoudre $P_0$ a un sens.
\begin{exemple}
	On étudie les signaux dans $\mathbb{R}^N$ et on choisi un dictionnaire constitué de la concaténation de la base de Fourier $ W = \{e_k(t) = \frac{1}{\sqrt{N}} e^{\frac{i 2\pi k t}{N}}\}_{0 \leq k \leq N -1}$ et de la base canonique de Diracs\footnote{Chaque vecteur de cette base vérifie $\delta_{k,i} = 1$ si $i = k$ et 0 sinon.} $T = \{\delta_k\}_{0 \leq k \leq N-1}$.
	Ainsi, avec ce choix $F_W$ est la matrice de Fourier discrète et $F_T$ est la matrice identité de taille $N$.
	Avec le théorème suivant, on va obtenir un principe d'incertitude, qui nous garantira qu'un signal ne peut pas être parcimonieux à la fois dans la base de Dirac, et dans la base de Fourier.
	\begin{theoreme}\label{th:Incert1}
		Soit un signal $f\in \mathbb{R^N}$ non nul, alors
		\begin{equation}
			||F_W f||_0 ||F_T f||_0 \geq N 	
		\end{equation}
		et ainsi
		\begin{equation}
			||F_W f||_0 + ||F_T f||_0 \geq 2 \sqrt{N}. 	
		\end{equation}
	\end{theoreme}
	\begin{proof}
	TODO :Ajouter Ref Tao uncertainty principle for cyclic....
		Soit $0 \leq \omega \leq N-1$ un entier, alors 
		\begin{align}
			|F_W(\omega)| &= |\hat{f}(\omega)| = \frac{1}{\sqrt{N}}|\sum_t f(t)e_\omega(t)| \\
				&\leq \frac{1}{\sqrt{N}}\sum_t |f(t)|,
		\end{align}
		d'où $\sup_\omega |F_W(\omega)| \leq \frac{1}{\sqrt{N}} \sum_t |f(t)|$.
		On pose maintenant $sign(f) = (\frac{f(t)}{|f(t)|})_t$ pour tous les $t$ tels que $f(t) \neq 0$ et 0 si $f(t) = 0$ et on a ainsi $\langle sign(f), sign(f) \rangle = ||F_T f||_0$, on va ainsi pouvoir montrer le théorème en utilisant successivement l'inégalité de Cauchy-Schwarz puis l'égalité de Parseval,
		\begin{align}
			\sup_\omega|F_W(\omega)| &\leq \frac{1}{\sqrt{N}}\sum_t |F_Tf (t)| = \frac{1}{\sqrt{N}}\langle sign(f), |f| \rangle \\
			&\leq  \frac{1}{\sqrt{N}} ||F_T||_0^{\frac{1}{2}} \langle |f|, |f| \rangle ^{\frac{1}{2}} =  \frac{1}{\sqrt{N}} ||F_T||_0^{\frac{1}{2}} \langle |F_W f|, |F_Wf| \rangle ^{\frac{1}{2}} \\
			&\leq   \frac{1}{\sqrt{N}} ||F_T||_0^{\frac{1}{2}} \langle |f|, |f| \rangle ^{\frac{1}{2}} =  \frac{1}{\sqrt{N}} ||F_T||_0^{\frac{1}{2}} \left( \sum_\omega|\hat{f}(\omega)|^2 \right)^{\frac{1}{2}} \leq \frac{1}{\sqrt{N}}||F_T||_0^{\frac{1}{2}} ||F_W f||_0^{\frac{1}{2}} \sup_\omega |F_W(\omega)|.
		\end{align}
		On a ainsi montré la première partie du théorème, la deuxième partie provient directement de l'inégalité entre la moyenne arithmétique et la moyenne géométrique.
		En effet, on a 
		\begin{equation}
			\sqrt{||F_T f||_0 ||F_W f||_0} \leq \frac{||F_T f||_0 + ||F_W f||_0}{2}
		\end{equation}
	et on a déjà montré que le terme de gauche est supérieur ou égal à $\sqrt{N}$.
	\end{proof}
	Observons que sans restrictions sur $N$, l'inégalité obtenue ne peut pas être améliorée, comme observé dans (Donoho-Stark 89 /Donoho-Huo TODO: ajouter ref), si $N$ est un carré, alors, la fonction avec des 1 seulement aux coefficients multiples de $\sqrt{N}$ et 0 ailleurs est sa propre transformée de Fourier et ainsi elle a $2\sqrt{N}$ coefficients dans le dictionnaire $(T,W)$ et ainsi l'inégalité est atteinte.
	Une conséquence de cela est qu'une condition sur la parcimonie de la forme $||F_T f||_0 + ||F_W||_0 < K$ avec $K>\sqrt{N}$ ne pourra pas garantir l'unicité de la solution de (P0).
	Montrons que si $K = \sqrt{N}$ alors on a l'unicité de la solution de (P0).
	\begin{theoreme}\label{th:Incert2}
		Soit $N$ un entier positif et un signal $f \in \mathbb{R}^N$, alors n'importe quel $x$ vérifiant $f = F_\Phi x = F_W x_W + F_T x_T$ et 
		\begin{equation}\label{eq:Incert1}
			||F_W x_W||_0 + ||F_T x_T||_0 < \sqrt{N}
		\end{equation}
		est l'unique solution de (P0).
	\end{theoreme}
	Supposons que pour $f$ donné non nul et supposons que l'on ait deux solutions de (P0), $x_1$ et $x_2$, ainsi $f = F_\Phi x_1, f = F_\Phi x_2$ et on a aussi $||x_1||_0 < \sqrt{N}, ||x_2||_0 < \sqrt{N}$.
	On a par linéarité de l'opérateur $F_\Phi$, 
	\begin{equation}
		F_\Phi( x_1 - x_2) = 0.
	\end{equation}
	Etudions ainsi les éléments du noyau de $F_\Phi$, posons $\mathcal{N} = \{\delta : F_\Phi \delta = 0\}$, et pour tout $\delta \in \mathcal{N}$, écrivons $\delta = (\delta_T, \delta_W)$, on a
	\begin{equation}
		F_T \delta_T + F_W \delta_W = 0
	\end{equation}
	ainsi, en utlilisant que les colonnes de $F_W$ forment une base, donc $F_W$ est une matrice orthogonale, on a 
	\begin{equation}\label{eq:structN}
		\delta_W = -F_W^t F_T \delta_T .
	\end{equation}
	On a donc montré que les éléments de $\mathcal{N}$ sont de la forme $\delta = (\delta_T, -F_W^t F_T \delta_T)$ et d'après le théorème \ref{th:Incert1}, on a que $\delta$ a au moins $2\sqrt{N}$ coefficients non nuls si $\delta$ est non nul. 
	En revenant a la situation initiale $\delta = x_1 - x_2$, on a une contradiction car à la fois $x_1$ et $x_2$ ont chacun moins de $\sqrt{N}$ coefficients, donc $\delta = 0$.
	Ainsi, si une solution existe avec moins de $\sqrt{N}$ coefficients, alors c'est la solution de (P0) et elle est unique.
	\newline
	Cependant, en choisissant $N = p$, où $p$ est un nombre premier\footnote{L'hypothèse $p$ premier est essentielle, la preuve reposant sur la non-existence de sous-groupes propres du groupe cyclique $\mathbb{Z}/p\mathbb{Z}$}, Tao 2005 (TODO: ajouter ref) a montré que l'on obtient l'inégalité
	\begin{equation}
		||F_T f||_0 + ||F_W f||_0 \geq p + 1
	\end{equation}
	et que l'inégalité est atteinte\footnote{L'inégalité est atteinte en ce sens que si $A\subset T$ et $B\subset W$ tels que $|A| + |B| \geq p+1$ alors il existe une fonction $f$ telle que $Supp F_T f = A$ et $Supp F_W f = B$}.
	Grâce à ce principe d'incertitude plus fort que \ref{th:Incert1}, on obtient avec le même type de preuve\footnote{Voir Candes-Romberg-Tao pour les détails, un lemme sur l'injectivité d'un opérateur similaire $F_W$ est tout de même nécessaire pour conclure la preuve.} le résultat suivant
	\begin{theoreme}
		Soit $N$ un nombre premier et un signal $f \in \mathbb{R}^N$, alors n'importe quel $x$ vérifiant $f = F_\Phi x = F_W x_W + F_T x_T$ et
		\begin{equation}\label{eq:Incert2}
		||F_T x_T||_0 \leq \frac{N}{2}
		\end{equation}
		est l'unique solution de (P0).
	\end{theoreme}
	On a ainsi vu qu'avec un dictionnaire constitué de Fourier et de Dirac la solution de (P0) est unique, on a également vu brièvement, qu'en renforçant le principe d'incertitude sur les deux familles, alors on peut certifier qu'on a bien obtenu \it{la} solution de (P0) pour des signaux avec un support plus grand.
\end{exemple}
\subsection{Solution optimale combinatoire}
\subsection{Résolution dans un dictionnaire pics/Fourier}
\subsection{Principe d'incertitude}

\section{Résolution de (P1)}
On a ainsi vu dans la section précedente que le problème (P0) de minimisation de la solution par rapport à la parcimonie admet une solution unique dès qu'une solution existe et que cette solution vérifie une condition de la forme \ref{eq:Incert1} ou \ref{eq:Incert2}.
Cependant, on a aussi vu au début de la section précédente que le problème (P0) est un problème de nature combinatoire et le nombre de combinaisons possibles augmentant très vite par rapport à $N$, sa résolution n'est pas faisable et ainsi il est nécessaire d'avoir une autre approche à ce problème.
\newline
La découverte qui a permis de rendre la résolution faisable, et par là permis par exemples les avancées du compressed sensing qui ont eu de nombreuses applications et dont la théorie sera étudiée dans le prochain chapitre, est que l'on peut résoudre un autre problème pour lequel des méthodes de résolution efficaces existaient déjà.
En effet nous allons voir que résoudre le problème \ref{P1},
\begin{equation}\label{P1}\tag{P1}
	min_x ||x||_1 \text{tel que } f = Fx 
\end{equation}
permet sous certaines conditions de résoudre \ref{P0}.
L'intéret de \ref{P1} est que c'est un problème de programmation linéaire et de nombreuses méthodes permettent de le résoudre. (TODO ajouter refs et détails).
Précisons donc ce que nous avons affirmé, dans le même cadre que précedemment, c'est à dire dans le cas d'un dictionnaire $\Phi = (T,W)$ temps fréquence composé de la base de Fourier et de Dirac dans $\mathbb{R}^N$.
\begin{theoreme}\label{th:DiracFourier}
	Soit $N$ un entier positif, $\Phi = (T, W)$ est la concaténation des bases de Dirac et de Fourier et un signal $f\in \mathbb{R}^N$, alors n'importe quel $x = (x_T, x_W)$ vérifiant $f = F_T x_T + F_W x_W$ et
	\begin{equation}\label{eq:cond1}
		||x_T||_0 < \frac{\sqrt{N}}{2} \quad \text{et} \quad   ||x_W||_0 < \frac{\sqrt{N}}{2}
	\end{equation}
	est l'unique solution de \ref{P1}, et c'est la solution de \ref{P0}.
\end{theoreme}
\begin{remarque}
	Le théorème précédent a été obtenu en cherchant une preuve alternative à la preuve qui est faite par Donoho et Huo (TODO: ajouter ref), leur preuve, comme une grande partie de la section précédente, utilise le même schéma que celle qui est faite ici.
	Leur théorème est le suivant :
\begin{theoreme}
	Soit $N$ un entier positif, $\Phi = (T, W)$ est la concaténation des bases de Dirac et de Fourier et un signal $f\in \mathbb{R}^N$, alors n'importe quel $x = (x_T, x_W)$ vérifiant $f = F_T x_T + F_W x_W$ et
	\begin{equation}
		||x_T||_0 +  ||x_W||_0 < \frac{\sqrt{N}}{2}
	\end{equation}
	est l'unique solution de \ref{P1}, et c'est la solution de \ref{P0}.
\end{theoreme}
	La preuve qui est présentée utilise un lemme qui est une version affaiblie d'un résultat présenté dans l'article.
	Dans l'article, l'inégalité plus forte qui est utilisée est obtenue à l'aide d'un principe variationnel (TODO: préciser), mais comme les auteurs le remarquent, leur résultat ne semblait pas exact au sens où même lorsque l'inégalité est atteinte il n'y avait aucun contre-exemple apparent.
	En effet, dans le cas de la base de Fourier-Dirac, le peigne de Dirac, fournit dans certains cas un exemple de signal qui est supporté sur $\sqrt{N}$ coefficients soit dans la base de Fourier, soit dans la base de Dirac, ainsi le problème \ref{P0} a plusieurs solutions et donc une condition nécessaire pour résoudre simultanément \ref{P1} et \ref{P0} est $||x||_0 < \sqrt{N}$.
	Or, les hypothèses du théorème ne sont plus vérifiées dès que $||x||_0 = \sqrt{N}$ (car au moins, soit $x_T$, soit $x_W$ est supporté sur au moins $\frac{\sqrt{N}}{2}$ coefficients). 
\end{remarque}
\begin{proof}
	La preuve de ce théorème se fait en plusieurs parties.
	Tout d'abord, remarquons que si $x$ vérifie \ref{eq:cond1}, alors $x$ vérifie \ref{eq:Incert1} et donc d'après le théorème \ref{th:Incert2} $x$ est donc l'unique condition de \ref{P0}.
	Il nous faut donc vérifier que cette solution est bien la solution de (P1).
	On montre ensuite un lemme qui permet de donner une condition suffisante pour qu'une paire de bases vérifie que la solution obtenue est bien celle de \ref{P1}.
	On vérifiera ensuite que dans la paire de bases Fourier-Dirac, les conditions du lemme sont vérifiées et cela permettra de conclure la preuve du théorème.
	Avant d'énoncer le lemme, définissons une quantité $\mu$ qui mesure dans une paire de bases $\Phi=(T,W)$ à quel point un élément dans le noyau de $F_\Phi$ peut être supporté à la fois sur $T$ et sur $W$.
	\begin{definition}
		Soit $\Phi = (T,W)$ une paire de bases, on note $\mathcal{N} = \{\delta = (\delta_T, \delta_W): F_\Phi \delta = 0\}$, soit $\Gamma_T$ (resp. $\Gamma_W$) un ensemble d'indices de $T$ (resp. $W$), alors on pose
		\begin{equation}
			\mu(\Gamma_T, \Gamma_W) = \sup_{\delta \in \mathcal{N}} \frac{\sum_{t \in \Gamma_T} |\delta_{T,t}| + \sum_{\omega \in \Gamma_W} |\delta_{W,\omega}|  }{||\delta_T||_1 + ||\delta_W||_1 }
		\end{equation}
	\end{definition}
	\begin{lemme}\label{th:muP1}
		Soit un signal $f \in \mathbb{R}^N$ et $\Phi=(T,W)$ une paire de bases de $\mathbb{R}^N$, alors n'importe quel $x = (x_T, x_W)$, où $\Gamma_T$ est le support de $x_T$ et $\Gamma_W$ est le support de $x_W$, 	vérifiant $f = F_T x_T + F_W x_W$ et
		\begin{equation}\label{eq:condmu}
			\mu(\Gamma_T, \Gamma_W) < \frac{1}{2}
		\end{equation}
		est l'unique solution de \ref{P1}.
	\end{lemme}
	Pour prouver le théorème on vérifiera donc dans la base de Fourier-Dirac que pour n'importe quelle paires d'indices vérifiant les conditions du théorème alors l'inégalité \ref{eq:condmu} sera vérifiée, et ainsi la solution de \ref{P0} sera bien la même que celle de \ref{P1} ce qui permettra de conclure la preuve du théorème.
	Enonçons donc cela sous la forme d'un autre lemme
	\begin{lemme}\label{th:muFD}\footnote{C'est ce lemme dont il est fait mention dans la remarque précédant la preuve et qui permet la généralisation du théorème de Donoho et Huo.}
		Soit $\Phi=(T,W)$ la paire de bases Fourier-Dirac et soient $\Gamma_T$  et $\Gamma_W$ des sous ensembles d'indices de $T$ et respectivement de $W$ vérifiant
		\begin{equation}
			|\Gamma_T| < \frac{\sqrt{N}}{2} \quad \text{et} \quad |\Gamma_W| < \frac{\sqrt{N}}{2},
		\end{equation}
		alors on a,
		\begin{equation}
			\mu(\Gamma_T, \Gamma_W) < \frac{1}{2}.
		\end{equation}
	\end{lemme}
	Ainsi, une fois les lemmes démontrés, le théorème le sera aussi.
	\end{proof}
	Commençons par la preuve du lemme \ref{th:muP1}.
	\begin{proof}
		Supposons que $x$ vérifie les conditions du lemme, c'est à dire, $x$ est effectivement une solution de l'équation $f=F_\Phi x$ et la condition \ref{eq:condmu} est vérifiée sur $\Phi$, alors on doit donc montrer que $x$ est l'unique solution de (P1), on doit donc montrer que pour tout $x_1$ différent de $x$ qui vérifie $f = F_\Phi x_1$ alors $||x_1||_1 > ||x||_1$. 
		Donc de façon équivalente, pour tout $\delta \in \mathcal{N} = \{\delta : F_\Phi \delta = 0\}$ non nul, on doit vérifier que
		\begin{equation}\label{eq:ineqdelta3}
			||x + \delta||_1 - ||x|| > 0.
		\end{equation}
		Notons $\Gamma = \{\gamma : c_\gamma \neq 0\} = \Gamma_T \cup \Gamma_W \subset [0, 2N-1]$ l'ensemble des indices non nuls de $x = (c_\gamma)_\gamma$, 
		on peut donc décomposer la somme
		\begin{equation}
			||x + \delta||_1 - ||x||_1 = \sum_{\gamma \in \Gamma^c} |\delta_\gamma| + \sum_{\gamma \in \Gamma} |c_\gamma + \delta_\gamma| - |c_\gamma|.
		\end{equation}
		Par l'inégalité triangulaire on a $|c_\gamma| \leq |c_\gamma + \delta_\gamma| + |\delta_\gamma|$ quel que soit $\gamma$.
		On a ainsi 
		\begin{equation}
			|c_\gamma + \delta_\gamma| - |c_\gamma| \geq -|\delta_\gamma|
		\end{equation}
		et en insérant cette inégalité dans la somme on obtient
		\begin{equation}
			||x + \delta||_1 - ||x||_1 \geq \sum_{\gamma \in \Gamma^c} |\delta_\gamma| - \sum_{\gamma \in \Gamma} |\delta_\gamma|,
		\end{equation}
		ainsi une condition suffisante pour obtenir l'unicité est que pour $\delta \in \mathcal{N}$ non nul on ait
		\begin{equation}\label{eq:ineqdelta}
			\sum_{\gamma \in \Gamma} |\delta_\gamma| < \sum_{\gamma \in \Gamma^c} |\delta_\gamma|. 
		\end{equation}
		Avec des mots cela revient à dire que si $\delta$ est dans $\mathcal{N}$ et non nul, alors $\delta$ a plus de poids hors du support de $x$ que sur le support de $x$.
		En ajoutant le terme de gauche de l'inégalité précédente des deux côtés on obtient
		\begin{equation}
			\sum_{\gamma \in \Gamma} |\delta_\gamma| < \frac{1}{2} \left(\sum_{t \in T} |\delta_{T, t}| + \sum_{\omega \in W} |\delta_{W, \omega}|\right) = \frac{||\delta_T||_1 + ||\delta_W||_1}{2}.
		\end{equation}
		Donc l'inégalité précédente est aussi une condition suffisante pour que \ref{eq:ineqdelta3} soit vérifiée et on peut réécrire cette inégalité sous la forme
		\begin{equation}\label{eq:ineqdelta4}
			\frac{\sum_{t \in \Gamma_T} |\delta_{T,t}| + \sum_{\omega \in \Gamma_W} |\delta_{W,\omega}|  }{||\delta_T||_1 + ||\delta_W||_1 } < \frac{1}{2}.
		\end{equation}
		On veut que l'inégalité soit vérifiée pour n'importe quel delta, donc en vérifiant la condition sur le suprémum des $\delta$ dans le noyau de $F_\Phi$ le lemme sera vrai.
		C'est exactement la condition \ref{eq:ineqmu} du lemme
		\begin{equation}
			\mu(\Gamma_T, \Gamma_W) := \sup_{\delta \in \mathcal{N}} \frac{\sum_{t \in \Gamma_T} |\delta_{T,t}| + \sum_{\omega \in \Gamma_W} |\delta_{W,\omega}|  }{||\delta_T||_1 + ||\delta_W||_1 } < \frac{1}{2}.
		\end{equation}
		Le lemme \ref{th:muP1} est donc bien démontré.
		
		
		On peut au passage remarquer qu'on peut utiliser la structure du noyau de $F_\Phi$ de la façon suivante afin d'obtenir une écriture équivalente de \ref{eq:ineqmu} mais qui utilise le fait qu'un élément du noyau de $F_\Phi$ est entièrement déterminé par ses coefficients dans l'une des deux bases.
		On avait vu avec \ref{eq:structN} que les éléments $\delta$ de $\mathcal{N}$ sont de la forme $(\delta_T, -F_W^t F_T \delta_T) =: (\delta_T, -\widehat{\delta_T})$, donc \ref{eq:ineqdelta4} devient
		\begin{equation}
			\frac{\sum_{t\in \Gamma_T} |\delta_{T, t}| + \sum_{\omega \in \Gamma_W} |\widehat{\delta_T}_\omega|}{||\delta_T||_1 + ||\widehat{\delta_T}||_1} < \frac{1}{2}.
		\end{equation}
	\end{proof}	
	On peut maintenant passer à la preuve du lemme \ref{th:muFD}
	\begin{proof}
		\begin{equation}\label{eq:ineqmu}
			\mu(\Gamma_T, \Gamma_W) \leq \frac{\sum_{t \in \Gamma_T} |\delta_{T, t}| + \sum_{\omega \in \Gamma_W} |\widehat{\delta_{T}}_\omega|}{||\delta_T||_1 + ||\delta_W||_1}. 
		\end{equation}
		Maintenant majorons le numérateur avec 
		\begin{equation}\label{eq:ineqnum}
			\sum_{\omega \in \Gamma_W} |\widehat{\delta_T}_\omega| = ||R_{\Gamma_W} F_W^t F_T \delta_T||_1 \leq ||R_{\Gamma_W} F_W^t F_T ||_1 ||\delta_T||_1 
		\end{equation}
		où $||A||_1 = \sup_i ||c_i||_1$ avec $c_i$ les colonnes de la matrice, et $R_{\Gamma_W}$ est la matrice de projection dans l'espace engendré par les vecteurs indexés par $\Gamma_W$.
		Donc $R_{\Gamma_W} F_W^t F_T$ est une matrice à $|\Gamma_W|$ lignes et $N$ colonnes, la norme $\ell_1$ de chaque colonne est égale à $\frac{|\Gamma_W|}{\sqrt{N}}$, ainsi, on a\footnote{C'est ici que le choix de la paire de bases a une importance, la matrice $F_W^t F_T$ contient tous les produits scalaires des vecteurs de $W$ et de $T$, dans le dictionnaire de Fourier-Dirac, chacun des coefficients vaut $1/\sqrt{N}$} :
		\begin{equation}\label{eq:ineqdelta1}
			||R_{\Gamma_W} F_W^t F_T||_1 = \frac{|\Gamma_W|}{\sqrt{N}}.
		\end{equation}	
			Maintenant appliquons la même chose à $\delta_T = -R_{\Gamma_T}F_T^tF_W \delta_W$:
			\begin{equation}
				\sum_{t \in \Gamma_T} |\delta_{T,t}| = ||R_{\Gamma_T} F_T^t F_W \delta_W||_1 \leq ||R_{\Gamma_T} F_T^t F_W ||_1 ||\delta_T||_1 
			\end{equation}
		ainsi que
		\begin{equation}
			||R_{\Gamma_T} F_T^t F_W||_1 = \frac{|\Gamma_T|}{\sqrt{N}}.
		\end{equation}
		On peut maintenant rassembler les résultats:
		\begin{equation}
			\sum_{t \in \Gamma_T} |\delta_{T, t}| + \sum_{\omega \in \Gamma_W} |\widehat{\delta_T}_\omega| 
			\leq ||\delta_{W}||_1 \frac{|\Gamma_T|}{\sqrt{N}} + ||\delta_{T}||_1 \frac{|\Gamma_W|}{\sqrt{N}}. 
		\end{equation}
			On utilise maintenant les hypothèses $|\Gamma_T| < \sqrt{N}/2$ et $|\Gamma_W| < \sqrt{N}/2$, on obtient ainsi :
		\begin{equation}
			\sum_{t \in \Gamma_T} |\delta_{T, t}| + \sum_{\omega \in \Gamma_W} |\widehat{\delta_T}_\omega| 
			< \frac{||\delta_{T}||_1 + ||\delta_{W}||_1 }{2}.
		\end{equation}
			Il nous reste maintenant à appliquer la majoration que l'on vient de trouver à \ref{eq:ineqmu} et on obtient
		\begin{equation}
			\mu(\Gamma_T, \Gamma_W) < \frac{1}{2} \frac{||\delta_T||_1 + ||\delta_W||_1}{||\delta_T||_1 + ||\delta_W||_1} = \frac{1}{2} 
		\end{equation}
			Ce qui conclut la preuve du lemme \ref{th:muFD} et donc du théorème \ref{th:DiracFourier}.
	\end{proof}


\subsection{Définition de (P1)}
\subsection{Propriétés du minimiseur}

\section{Lien géométrique entre (P0) et (P1)}
\subsection{Boules unité en grande dimension}
\subsection{Unicité de la solution de (P0) et (P1)}


