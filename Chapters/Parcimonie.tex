\section{Signaux ayant une représentation parcimonieuse}
\subsection{Définition d'une représentation parcimonieuse}
\subsection{Importance de la base}

\section{Décroissance des coefficients et régularité}
\subsection{Approximation linéaire et régularité}
On s'intéresse dans cette partie au lien entre une fonction $f\in L^2(]0, 1[)$ et son approximation dans une base. On verra un résultat reliant la décroissance des coefficients de la fonction dans une base fixée et la vitesse de convergence de la reconstruction. 
On verra ensuite à l'aide de ce résultat, que pour la base de Fourier (et resp. certaines bases d'ondelettes), on obtient des formules de reconstruction pour les fonctions dérivables (et resp. pour les fonctions Lipschitziennes) avec une erreur de reconstruction qui décroit rapidement.
\newline 
On considère ainsi un espace d'approximation de fonctions $U_N \subset L^2([0, 1])$.
Par construction, la meilleure approximation linéaire de $f$ dans $U_N$, est la projection orthogonale $f_N$ de $f$ dans $U_N$, qui peut être obtenue à l'aide de la base biorthogonale de synthèse associée $(\tilde{\phi}_k)_{k=1, \cdots, N}$ et la formule de reconstruction :
\begin{equation}
	f_N = \sum_{k=0}^{N-1}\langle f, \phi_k \rangle \tilde{\phi}_k.
\end{equation}
Afin de mesurer l'erreur d'approximation par rapport à $f$, on considère une base $\mathcal{B} = \{g_k\}_{k\in \mathbb{N}}$ de l'espace $L^2([0, 1])$ entier à laquelle on ajoute la condition de contenir une famille $(g_k)_{k\in I}$, avec $\#I = N$ qui forme une base de l'espace d'approximation $U_N$.
On peut ainsi écrire, en réordonnant la famille $(g_k)$,  $f_N \in U_N$ dans cette base :
\begin{equation}
	f_N = \sum_{k=0}^{N-1} \langle f, g_k \rangle g_k 
\end{equation}
et les $\{g_k\}_{\mathbb{N}}$ formant une base de $L^2([0, 1])$, on peut écrire $f$ dans cette base:
\begin{equation}
	f = \sum_{k=0}^{+\infty} \langle f, g_k \rangle g_k.
\end{equation}
On obtient donc que la partie orthogonale à la famille $\{\phi_k\}_{k=0, \cdots, N-1}$, est celle analysée par $\{g_k\}_{k\geq N}$.
C'est à dire, 
\begin{equation}
	f - f_N = \sum_{k=N}^{+\infty} \langle f, g_k \rangle g_k 
\end{equation}
et la mesure de l'erreur d'approximation avec $N$ coefficients est donc 
\begin{equation}
	\varepsilon_l(N, f) = ||f-f_N||^2 = \sum_{k=N}^{+\infty} |\langle f, g_k\rangle|^2.
\end{equation}
Comme on a supposé que $f\in L^2([0, 1])$ et que la famille $(g_k)$ est génératrice, on a que l'erreur d'approximation tend vers $0$ lorsque $N$ augmente.
On va maintenant s'intéresser au théorème suivant de Stéphane Mallat qui relie la décroissance des coefficients de $f$ dans la base de $L^2([0, 1])$ à la vitesse de décroissance de l'erreur d'approximation de la fonction.
\begin{theoreme}
	Soit $r > 1/2$, il existe des constantes $A, B > 0$ telles que si 
	\begin{equation}
		\sum_{k=0}^{+\infty} |k|^{2r}|\langle f, g_k \rangle |^2 < \infty,
	\end{equation}
	alors on a 
	\begin{equation}\label{eq:regframe}
		A \sum_{k=0}^{+\infty} k^{2r}|\langle f, g_k \rangle |^2 \leq 
		\sum_{N=0}^{+\infty} N^{2r-1} \varepsilon_l(N, f) \leq
		B \sum_{k=0}^{+\infty} k^{2r} |\langle f, g_k\rangle |^2
	\end{equation}
	et ainsi on a $\varepsilon_l(N, f) = o(N^{-2r})$.
\end{theoreme}
\begin{preuve}
	On développe le terme au centre de l'égalité dela façon suivante 
	\begin{equation*}
		\sum_{N=0}^{\infty} N^{2r-1} \varepsilon_l(N, f) =
		\sum_{N=0}^{\infty} \sum_{k=N}^{\infty} N^{2r-1} |\langle f, g_k \rangle |^2 =
		\sum_{k=0}^{+\infty} |\langle f, g_k \rangle |^2 \sum_{N=0}^{k} N^{2r-1}.
	\end{equation*}
	Puis on majore des deux côtés avec 
	\begin{equation}
		\int_{0}^M x^{2r-1} dx \leq \sum_{N=0}^m N^{2r-1} \leq \int_{1}^{m+1} x^{2r-1}dx.
	\end{equation}
	En calculant les deux intégrales on déduit 
	\begin{equation}
		A m^{2r} \leq \sum_{N=0}^{m} N^{2r-1} \leq B m^{2r}
	\end{equation}
	où $A$ et $B$ dépendent seulement de $r$, ce qui nous donne \ref{eq:regframe}.
	Montrons maintenant $\varepsilon_l(N, f) = o(N^{-2r})$, remarquons tout d'abord que $\varepsilon_l(N, f)$ est décroissant par rapport à la première variable, on déduit de cela
	\begin{equation*}
		\varepsilon_l(N, f)\sum_{m=N/2}^{N-1} m^{2r-1} \leq \sum_{m=N/2}^{+\infty} m^{2r-1} \varepsilon_l(m, f)
	\end{equation*}
	on a donc avec le calcul précédent que le terme de droite converge quel que soit le choix de $N$, et ainsi
	\begin{equation*}
		\lim_{N \to +\infty} \sum_{m=N/2}^{+\infty} m^{2r-1}\varepsilon_l(m, f) = 0
	\end{equation*}
	donc tous les termes de l'inégalité précédente tendent vers 0. 
	De plus, $\sum_{m = N/2}^{N-1} m^{2r-1} \geq CN^{2r}$ et donc
	\begin{equation*}
		\lim_{N\to \infty} \varepsilon_l(N, f)N^{2r} = 0.
	\end{equation*}
\end{preuve}
On a ainsi démontré que si $f$ appartient à l'espace
\begin{equation}
	W_{\mathcal{B}, r} =\{f : \sum_{m=0}^{+\infty} m^{2r} |\langle f, g_m \rangle |^2 < \infty\}
\end{equation}
alors l'approximation linéaire dans la base $\mathcal{B}$ décroit au moins comme $N^{-2r}$.
On montrera dans la prochaine sections que si $\mathcal{B}$ est une base de Fourier alors $W_{\mathcal{B}, r}$ contient les fonctions $r$-différentiables. 
On montrera ensuite que si $\mathcal{B}$ est une base d'ondelette avec une certaine propriété alors l'espace $W_{\mathcal{B}, r}$ contient les fonctions $\alpha$-Lipschitziennes pour $1 < \alpha < r$.
Des énoncés réciproques existent aussi et des démonstrations de ceux-ci peuvent être trouvés dans ( TODO :Ajouter Ref )
\subsection{Décroissance des coefficients de Fourier}
On considère ici $\mathcal{B} = (e_{n})_{n \in \mathbb{Z}}$ la base de Fourier ( voir 1.3.3) de $L^2(\mathbb{R})$.
On peut ainsi définir l'espace $U_N = \{f \in L^2(\mathbb{R}) : |k| > N \implies \hat{f}(k) = 0\}$ sur lequel $\mathcal{B}_N = (e_{n})_{|n|\leq N}$ est une base. 
Avec des mots, $U_N$ est l'espace des fonctions qui ne sont portées par aucune exponentielle complexe de fréquence supérieure ou égale à $N$.
Le théorème de Shannon nous indique que $2N$ fréquences permettent de séparer n'importe laquelle de ces fonctions, ainsi on  a la formule de reconstruction.
On dispose ainsi d'une formule de projection (et de reconstruction), soit $f \in L^2(\mathbb{R})$
\begin{equation}
	f_N(t) = \sum_{|n|\leq N/2} \langle f, e_n \rangle e_n(t) \in U_N.
\end{equation}
Ainsi $f_N$ est une approximation linéaire de $f$, et $f$ sera rapidement approximée si $f$ n'a pas trop de hautes fréquences. 
Montrons maintenant que la vitesse de décroissance des coefficients de Fourier est liée à la régularité de la fonction.
Tout d'abord, revenons à $L^2(\mathbb{R})$ considérons que $f$ est dérivable, alors on a en intégrant par parties
\begin{equation}
	\hat{f}'(\omega)\int_{-\infty}^{+\infty} f'(t) e^{i \omega t} dt = i \omega \int_{-\infty}^{+\infty} f(t) e^{i\omega t} dt = i \omega \hat{f}(\omega) 
\end{equation}
et en utilisant la formule de Plancherel on a 
\begin{equation}
	||\hat{f}'||_2^2 = \int_{-\infty}^{+\infty} |\omega|^2 |\hat{f}(\omega)|^2 d\omega = \int_{-\infty}^{+\infty} |f'(t)|^2 dt = ||f'||_2^2.
\end{equation}
On est ainsi amenés à définir une régularité dans $L^2(\mathbb{R})$, distincte de la dérivabilité en un point avec la définition suivante :
\begin{definition}
	On dit que $f\in L^2(\mathbb{R})$ est différentiable au sens de Sobolev si
	\begin{equation*}
		\int_{-\infty}^{+\infty} |\omega|^2|\hat{f}(\omega)|^2 d\omega < \infty. 
	\end{equation*}
\end{definition}
Et avec cette définition on peut définir pour n'importe quel $r>0$ l'espace des fonction $r$-différentiables de Sobolev:
\begin{equation}
	W^r(\mathbb{R}) =\{ f \in L^2(\mathbb{R}) : \int_{\mathbb{R}} |\omega|^{2r} |\hat{f}(\omega)|^2 d\omega < \infty \}.
\end{equation}
Ainsi d'après le théorème sur la vitesse d'approximation de Mallat, l'approximation linéaire dans la base de Fourier d'une application $r$-différentiable décroit plus vite que $N^{-2r}$.
\subsection{Décroissance des coefficients d'ondelettes}
On va montrer dans cette section que en imposant certaines conditions sur les ondelettes, alors il est possible de démontrer que la régularité au sens de Lipschitz, implique une décroissance des coefficients d'ondelettes.
On pourra ensuite relier cette décroissance aux discussions de la fin de la partie précédente.
\newline
Tout d'abord posons les définitions dont nous aurons besoin dans cette partie,
\begin{definition}
	Soit $\alpha$ tel qu'il existe un entier strictement positif $r$ tel que $r-1 \leq \alpha <r$. On dit qu'une fonction $f$ est $\alpha$-Lipschitz en $t_0$ si il existe une constante $C>0$ et un polynôme $P_{t_0}$ de degré strictement inférieur à $r$, tels que pour tout $t$ qui appartient à un voisinage $T_0$ de $t_0$, on a 
	\begin{equation*}
		|f(t) - P_{t_0}(t)| \leq C|t-t_0|^{\alpha}
	\end{equation*}
\end{definition}
Avec cette définition on vérifie immédiatement que si on considère un signal $r$-dérivable au sens classique, alors en utilisant l'approximation avec un polynôme de Taylor du signal, on a pour tout $0 < \alpha \leq r$, que le signal est partout $\alpha$-Lipschitz.
\newline
On peut facilement relier cette définition avec les ondelettes, en considérant les moments d'ondelette.
On considère ainsi $\{\psi_{j, k} = \psi( 2^{\frac{j}{2}}\psi(2^j\cdot -k)\}_{j,k}$ une base orthonormale d'ondelettes de $L^2([0, 1])$, et on a le coefficient d'ondelette à l'échelle $j$ et à l'instant $k$ donnée par
\begin{equation*}\label{eq:defWx}
	Wf (j, k) = \langle f, \psi_{j,k} \rangle = \int f(t) \psi_{j,k}(t)dt.
\end{equation*}
\begin{remarque}
Remarquons ici que l'on peut exprimer cette projection à partir de l'ondelette prise à l'instant $0$
\begin{equation*}
	\widecheck{\psi_j}(t) = \psi_{j, 0}(-t) = \psi(-2^jt),
\end{equation*}
et ainsi en faisant un changement de variable dans \ref{eq:defWx}, on obtient
\begin{equation}
	Wf(j,k) = f \star \widecheck{\psi_j}(k).
\end{equation}
On peut ainsi interpreter le coefficient d'ondelette pris en $(j,k)$ comme la corrélation entre le signal pris en $k$ avec une ondelette à l'échelle $j$.
C'est à dire qu'un coefficient avec une grande valeur indique une grande similitude entre l'ondelette et le signal (l'ondelette approxime bien le signal) alors qu'un petit coefficient indique que l'ondelette et le signal ont peu en commun\footnote{L'ondelette et le signal ont peu en commun au sens où ils sont de façon équivalente, presque orthogonaux, et donc ce coefficient à un poids faible dans la formule de reconstruction.}.
\end{remarque}
Rappelons aussi qu'un condition nécessaire pour que $\psi$ soit une ondelette génératrice est
\begin{equation*}
	\int \psi (t)dt = 0.
\end{equation*}
\begin{definition}
	Soit $m$ un entier strictement positif.
	On dit que $\psi$ a $m$-moments nuls si
	\begin{equation*}
		\int t^k \psi(t)dt = 0 \quad, \forall k < m.
	\end{equation*}
\end{definition}
De cette définition on déduit que si $P$ est un polynôme de degré strictement inférieur à $m$ et si l'ondelette a $m$-moments nuls, alors
\begin{equation}\label{eq:momPol}
	\int \psi(t) p(t) dt = 0.
\end{equation}
Ainsi si $f(t) = P(t) +\epsilon$ où $\epsilon$ représente un bruit, on a 
\begin{equation*}
	\langle f, \psi_{j,k} \rangle = o(\epsilon)
\end{equation*}
ainsi les coefficients de l'ondelette d'échelle suffisent à reconstruire $f$.
C'est ainsi que si on considère une ondelette avec un certain nombre de moments nuls, alors dans les parties régulières du signal, les coefficient d'ondelettes seront petits, alors que dans les zones avec des irrégularités ou des discontinuités, les coefficients resteront grands.
On peut aussi remarquer que si les irrégularités sont séparées, alors en affinant l'échelle d'analyse, le support des ondelettes diminue et alors de moins en moins de coefficients auront une valeur importante, ainsi les seuls coefficients qui resteront grand en changeant d'échelle sont ceux qui contiennent une zone irrégulière. 
\newline
On va maintenant démontrer le théorème suivant de Jaffard qui permet de préciser cela,
\begin{theoreme}\label{th:Jaffard}
	Si $f$ est $\alpha$-Lipschitz en $t_0$ avec $0 < \alpha \leq m$ où $m$ est un entier.
	Alors, il existe une constante $C >0$ telle que 
	\begin{equation*}
		|Wf(j, k)| \leq C2^{-j(\alpha + \frac{1}{2})}(1 + |2^{-j}t - t_0|2^{j\alpha}).
	\end{equation*}
\end{theoreme}
\begin{preuve}
	Remarquons tout d'abord que, soit $P$ un polynôme de degré strictement inférieur à $m$, et $\psi$ une ondelette a $m$-moments nuls, alors, par changement de variable et d'après \ref{eq:momPol} on a :
	\begin{equation}\label{eq:polAnn}
		WP(j, k) = \int 2^{j/2} \psi(2^jt - 2^{-j}k) P(t)dt 
		= \int 2^{j/2}\psi(t')P(2^jt' - 2^{-j} k) 2^{-j} dt' = 0.
	\end{equation}
	Par hypothèse, $x$ est $\alpha$-Lipschitz en $t_0$, donc il existe $P_{t_0}$ un polynôme de degré inférieur à $m$ tel que $|f(t) - P_{t_0}(t)| \leq C|t - t_0|^{\alpha}$.
	En utilisant la linéarité de l'intégrale et en appliquant une inégalité triangulaire on obtient
	\begin{align*}
		|Wf(j,k)| &= \left|\int (f(t) -P_{t_0}(t) + P_{t_0}(t)) \psi_{j, k}(t) dt\right| \\
			&\leq \left|\int (f(t) - P_{(t_0}(t)) \psi_{j,k}(t)dt\right| + \left|\int P_{t_0}(t) \psi_{j,k}(t)dt\right| \\ 
			&\leq \int| f(t) - P_{t_0}(t)| |\psi_{j,k}(t)|dt  
	\end{align*}
	la dernière inégalité étant obtenue en utilisant \ref{eq:polAnn} sur le terme de droite et en faisant entrer la valeur absolue dans la première intégrale.
	On utilise maintenant le fait que $f$ est $\alpha$-Lipschitz et on fait un changement de variable, on obtient ainsi
	\begin{align*}
		| Wf(j,k)| \leq \int |f(t) - P_{t_0}(t)||\psi_{j,k}(t)|dt &\leq C\int |t - t_0|^\alpha 2^{j/2} |\psi(2^jt - k)|dt \\
		&\leq C \int |2^{-j} t' + 2^{-j}k - t_0|^\alpha 2^{-j/2} |\psi(t')| dt'. 
	\end{align*}
	Pour obtenir l'inégalité suivante on utilise
	\begin{equation*}
		|a + b|^\alpha \leq |2*\max(|a|, |b|)|^\alpha \leq 2^\alpha (|a|^\alpha + |b|^\alpha)
	\end{equation*}
	et on a ainsi
	\begin{align*}
		|Wf(j,k)| &\leq 2^\alpha C \int (|2^{-j}t'|^\alpha + |2^{-j}k-t_0|^\alpha)2^{-j/2} |\psi(t')|dt'\\
		&\leq 2^\alpha C 2^{-j(\alpha + 1/2)}\left( \int|t'|^\alpha |\psi(t')|dt' + |2^{-j}k-t_0|^\alpha 2^{\alpha j}\int |\psi(t')|dt' \right)
	\end{align*}
	Ce qui donne le résultat dès que les intégrales considérées sont définies, ce qui est le cas par exemple si l'ondelette est à support compact ou bien à décroissance suffisamment rapide. 
\end{preuve}
On peut combiner le théorème de Jaffard avec une analyse multi-échelle d'ondelettes avec la proposition suivante :
\begin{proposition}
	Soit $f:]0,1[ \to \mathbb{R}$ une fonction $\alpha$-Lispchitzienne avec $\alpha>1$, alors il existe une constante $C>0$ et une base d'ondelette orthonormales associée à une multiresolution $\{(\psi_{j,k})_{(i,j): j\geq J, 2^{j} > k\geq 0}\}, \{\varphi_{J,k}\}_k$ avec $m>\alpha$ moments nuls telle que  
	\begin{equation}
		\varepsilon_l(f, 2^J) = ||f - \sum_{k=0}^{2^J -1} \langle f, \varphi_{J,k}\rangle \tilde{\varphi}_{J,k}||_2^2 \leq C 2^{-2J\alpha} = CN^{-2\alpha}
	\end{equation}
	avec $N=2^J$.
\end{proposition}
\begin{preuve}
	L'existence d'une telle base d'ondelette n'est pas démontrée ici, des constructions peuvent être trouvées dans (ajouter ref) pour obtenir des bases de $L^2(\mathbb{R})$, on peut ainsi considérer une telle multirésolution donnée par une ondelette de Daubechies ou bien une coiflet à $m>\alpha$ moments nuls.
	Il est ensuite possible, avec quelques difficultés d'obtenir depuis ces ondelettes, une base orthonormale de $L^2(]0,1[)$ (ajouter ref).
	Soit $f$ une fonction $\alpha$-Lipschitzienne sur $]0,1[$, on a ainsi d'après la partie sur les frames et l'existence de la base d'ondelette précédente admise, une formule de reconstruction
	\begin{equation*}
		f = \sum_{k=0}^{2^J -1} \langle f, \varphi_{J,k}\rangle \tilde{\varphi}_{J,k} + \sum_{j=J+1}^{+\infty}\sum_{k=0}^{2^j-1} \langle f, \psi_{j,k} \rangle \tilde{\psi}_{j,k}. 
	\end{equation*}
	On a ainsi, en réécrivant l'équation et en prenant la norme
	\begin{equation*}
		\varepsilon(f, 2^J) = ||f - \sum_{k=0}^{2^J -1} \langle f, \varphi_{J,k}\rangle \tilde{\varphi}_{J,k}||_2^2 =|| \sum_{j=J+1}^{+\infty}\sum_{k=0}^{2^j-1} \langle f, \psi_{j,k} \rangle \tilde{\psi}_{j,k}||_2^2. 
	\end{equation*}
	On peut alors majorer le terme de droite en utilisant le fait que la famille d'analyse est génératrice, on obtient
	\begin{equation*}
		\varepsilon(f, 2^J) \leq \sum_{j=J+1}^{+\infty} ||\sum_{k=0}^{2^j -1} \langle f, \psi_{j,k} \rangle \tilde{\psi}_{j,k} ||_2^2
	\end{equation*}
	et en utilisant le fait que les ondelettes sont normalisées on a
	\begin{equation*}
		\varepsilon(f, 2^J) \leq \sum_{j=J+1}^{+\infty} \sum_{k=0}^{2^j -1} |\langle f, \psi_{j,k} \rangle|^2
	\end{equation*}
	on utilise maintenant le théorème \ref{th:Jaffard} et on obtient\footnote{Le théorème de Jaffard est pour une fonction ponctuellement Lipschitzienne, on considère ici une fonction $\alpha$-Lipschitzienne en tout point, donc le terme en $(1 + \frac{|2^{-j}k -t_0|}{2^j})$ n'apparait pas.}
	\begin{align*}
		\varepsilon(f, 2^J) &\leq \sum_{j=J+1}^{+\infty} \sum_{k=0}^{2^j -1} C^2 2^{-j(2\alpha + 1)} \\
		&\leq \sum_{j=J+1}^{+\infty} C^2 2^{-j(2\alpha + 1)} 2^{j} = \sum_{j=J+1}^{+\infty} C^2 2^{-j2\alpha} \\
		&\leq \frac{C^2}{1 - 4^{-\alpha}} 2^{-2J\alpha}	
	\end{align*}
	ce qui prouve la proposition.\qedhere
\end{preuve}
\section{Résolution de (P0)}
\subsection{Définition de (PO)}
\subsection{Solution optimale combinatoire}
\subsection{Résolution dans un dictionnaire pics/Fourier}
\subsection{Principe d'incertitude}

\section{Résolution de (P1)}
\subsection{Définition de (P1)}
\subsection{Propriétés du minimiseur}

\section{Lien géométrique entre (P0) et (P1)}
\subsection{Boules unité en grande dimension}
\subsection{Unicité de la solution de (P0) et (P1)}


